%%=============================================================================
%% Methodologie
%%=============================================================================

\chapter{\IfLanguageName{dutch}{Resultaten}{Resultaten}}%
\label{ch:resultaten}

%% TODO: In dit hoofstuk geef je een korte toelichting over hoe je te werk bent
%% gegaan. Verdeel je onderzoek in grote fasen, en licht in elke fase toe wat
%% de doelstelling was, welke deliverables daar uit gekomen zijn, en welke
%% onderzoeksmethoden je daarbij toegepast hebt. Verantwoord waarom je
%% op deze manier te werk gegaan bent.
%% 
%% Voorbeelden van zulke fasen zijn: literatuurstudie, opstellen van een
%% requirements-analyse, opstellen long-list (bij vergelijkende studie),
%% selectie van geschikte tools (bij vergelijkende studie, "short-list"),
%% opzetten testopstelling/PoC, uitvoeren testen en verzamelen
%% van resultaten, analyse van resultaten, ...
%%
%% !!!!! LET OP !!!!!
%%
%% Het is uitdrukkelijk NIET de bedoeling dat je het grootste deel van de corpus
%% van je bachelorproef in dit hoofstuk verwerkt! Dit hoofdstuk is eerder een
%% kort overzicht van je plan van aanpak.
%%
%% Maak voor elke fase (behalve het literatuuronderzoek) een NIEUW HOOFDSTUK aan
%% en geef het een gepaste titel.
\section{Inleiding}
De Google Scholar alerts leverden 23 notificatie e-mails met zoekresultaten op voor de VLIZ zoekopdracht (zie hoofdstuk \ref{ch:googlescholaralert}) sinds 1 februari 2025 tot 16 mei 2025.\\
Figuur \ref{fig:GSAlertsTimeline} toont de tijdslijn van de Google Scholar alerts ontvangen tijdens deze tijdsspanne.
\begin{figure}[h!]
    \centering
    \includegraphics[width=0.8\textwidth]{./9_resultaten/GS_alerts_timeline.png}
    \caption[Tijdslijn Google Scholar alerts.]{\label{fig:GSAlertsTimeline}Tijdslijn Google Scholar alerts.}
\end{figure}
Een volledig overzicht van alle resultaten na web scraping, natural language processing, Crossref en semantic search is weergegeven in bijlage \ref{ch:overzicht_resultaten}.

\section{Web scraping}
Titel, link en tekst worden systematisch goed gevonden door de scraping. Voor auteurs, tijdschrift en jaartal en dat beduidend minder het geval. De reden daarvoor is dat deze 3 laatste parameters binnen dezelfde tag staan en gescheiden worden door specifieke karakters. Dat maakt het een pak moeilijker voor de scraper om die informatie te  ontleden, vooral in het geval wanneer 1 of 2 van die gegevens ook ontbreken in het zoekresultaat zoals te zien is in figuur \ref{fig:GSauteurtijdschriftjaartal}.
\begin{figure}[h!]
    \centering
    \includegraphics[width=0.8\textwidth]{./9_resultaten/GS_alerts_auteurtijdschriftjaartal.jpg}
    \caption[Google Scholar alert zonder auteur, tijdschrift of jaartal.]{\label{fig:GSauteurtijdschriftjaartal}Google Scholar alert zonder auteur, tijdschrift of jaartal.}
\end{figure}

\section{Relevantiescore}
De relevantiescore werd berekend op basis van het voorkomen van de zoektermen in de tekst. Op die manier vallen de zoekresultaten waarin de zoektermen niet voorkomen onmiddellijk op door hun waarde 0 zoals te zien is in figuur  \ref{fig:GSrelevantie}.
\begin{figure}[h!]
    \centering
    \includegraphics[width=0.8\textwidth]{./9_resultaten/GS_alerts_relevantiescore.png}
    \caption[Google Scholar alert relevantiescore.]{\label{fig:GSrelevantie}Google Scholar alert relevantiescore.}
\end{figure}

\section{DOI}
\begin{table}[h!]
    \caption{statistieken linked data}
    \centering
    \begin{tabularx}{\textwidth}{|X|p{4cm}|} 
        \hline
        Percentage aantal keer dat de DOI gevonden is.&48\%\\
        \hline
        Percentage aantal keer dat de DOI in de URL staat.&7\%\\
        \hline
        Percentage aantal keer dat de DOI via Crossref gevonden wordt.&20\%\\
        \hline
        Percentage aantal keer dat de DOI op de webpagina staat.&73\%\\
        \hline
    \end{tabularx}
    \label{table:statistieken_linked_data}
\end{table}

\section{Semantic search}
Voor de resterende 52\% van de zoekresultaten waarvan geen DOI gevonden kon worden, werd semantic search van de titel uitgevoerd in de IMIS databank. Dat vormde een hele uitdaging aangezien er 260.000 records zijn waarin de titel opgezocht moet worden.
De query om de titels op te zoeken is:
\[https://www.vliz.be/nl/imis?module=ref&show=json\]
\[&spcol=39,141,396,536,733,746,747,748,749,750,751,\]
\[752,753,754,755,756,757,758,759,760,761,762,763,764,\]
\[765,766,768,769,770,771,772,773,774,775,776,777,778,\]
\[779,780,781,782,783,784,785,786,787,788,789,790,791,\]
\[802,803,900,911,981,982,992,997,1002,1051,1064\]
Daarbij duiden de ``spcol'' waarden op alle mogelijke collecties binnen IMIS die te maken hebben met het VLIZ (de zoekterm van de Google Scholar alert).
Omdat IMIS niet toelaat om 260.000 records ineens op te vragen, wordt er gewerkt met 26 batches van 10.000 records. De overeenkomstige query is:
\[https://www.vliz.be/nl/imis?module=ref&show=json\]
\[&spcol=39,141,396,536,733,746,747,748,749,750,751,\]
\[752,753,754,755,756,757,758,759,760,761,762,763,764,\]
\[765,766,768,769,770,771,772,773,774,775,776,777,778,\]
\[779,780,781,782,783,784,785,786,787,788,789,790,791,\]
\[802,803,900,911,981,982,992,997,1002,1051,1064\]
\[&count=10000&start=0\]
Daarbij verhoogt de parameter ``start'' telkens met 10.000.\\
Deze manier van werken heeft mogelijks een invloed op het resultaat: de positie van een titel opzoeken in 10.000 resultaten is niet hetzelfde als in 260.000 resultaten.
In figuur \ref{fig:GSsss} is te zien dat de Semantic search score toeneemt in functie van het aantal titels waarin gezocht wordt. Maar de mate van toename neemt sterk af bij grote aantallen. Daarom wordt er aangenomen dat de Semantic search score bij 10.000 titels representatief is voor de score bij 260.000 titels.\\
\begin{figure}[h!]
    \centering
    \includegraphics[width=0.8\textwidth]{./9_resultaten/sss.png}
    \caption[Semantic search score vs. number of titles.]{\label{fig:GSsss}Semantic search score vs. number of titles.}
\end{figure}
Na verwerking is de Semantic search score de hoogste waarde van alle batches.\\\\
Alle gevonden scores liggen onder de 0.9. Dat is de threshold die empirisch bepaald werd om te beslissen of een titel reeds in IMIS zit (zie tabel \ref{table:empirisch}).\\
\begin{table}[h!]
    \caption{Empirische test kritische waarde semantic search}
    \centering
    \begin{tabularx}{\textwidth}{|X|p{4cm}|} 
        \hline
        \textbf{Titel}&\textbf{Semantic search score}\\
        \hline
        Genome-wide association study revealed candidate genes associated with egg-laying time traits in layer chicken&\textcolor{orange}{1.0}\\
        \hline
        Genome wide association study revealed candidate genes associated with egg laying time traits in layer chicken&\textcolor{orange}{0.9958072900772095}\\
        \hline
        Gen. wide assoc. study revealed candidate genes associated with egg time traits in layer chicken&\textcolor{orange}{0.9708424806594849}\\
        \hline
        Genome-wide association study revealed candidate genes associated with egg-laying time traits&\textcolor{orange}{0.9543328285217285}\\
        \hline
        Genome wide association study revealed candidate genes associated with egg laying time traits&\textcolor{orange}{0.9487157464027405}\\
        \hline
        Gen. wide assoc. study revealed candidate genes associated with egg time traits&\textcolor{orange}{0.9242777228355408}\\
        \hline
        with time study revealed associated Genome-wide in chicken genes traits candidate association egg-laying&\textcolor{orange}{0.9624322652816772}\\
        \hline
        Gen.-wide assoc. study reveal cand. gene associate with egg-lay time trait in layer chicken&\textcolor{orange}{0.9395400881767273}\\
        \hline
        Genome-wide association study revealed some new candidate genes associated with flowering and maturity time of soybean in Central and West Siberian regions of Russia&0.8146297931671143\\
        \hline
        Genome- and Exome-Wide Association Studies Revealed Candidate Genes Associated with DaTscan Imaging Features&0.7932324409484863\\
        \hline
        Genome-wide association study revealed candidate genes associated with leaf size in alfalfa&0.8393213748931885\\
        \hline
        Genome-Wide Association Studies Revealed the Genetic Loci and Candidate Genes of Pod-Related Traits in Peanut&0.8185563087463379\\
        \hline
    \end{tabularx}
    \label{table:empirisch}
\end{table}
Bijgevolg is er geen enkel duplicaat aanwezig tussen de zoekresultaten.
