\chapter{\IfLanguageName{dutch}{Stand van zaken}{State of the art}}%
\label{ch:stand-van-zaken}

% Tip: Begin elk hoofdstuk met een paragraaf inleiding die beschrijft hoe
% dit hoofdstuk past binnen het geheel van de bachelorproef. Geef in het
% bijzonder aan wat de link is met het vorige en volgende hoofdstuk.

% Pas na deze inleidende paragraaf komt de eerste sectiehoofding.

ToDo IMIS\\
ToDo Google Scholar\\
Parsen van Google Scholar zoekresultaten:
\autocite{Kim2019} stelt vast dat gebruikers meer interesse hebben in de weergegeven inhoud dan in de andere elementen zoals auteur, aantal citaties, en andere.
Een vooraf gedefinieerde zoekopdracht levert meer relevante zoekresultaten dan een instant zoekopdracht door een gebruiker op de user interface \autocite{Zhang2013}
\textcite{Wu2014} stelt vast dat studenten een voorkeur hebben voor Google Scholar als zoekmachine omwille van de vertrouwde interface.
ToDo check Khabsa M. 2017 The number of scholarly documents on the public web
ToDo check Serenko A. 2015 Citation classics published in knowledge management journals. Part ii: studying research trends and discovering the Google Scholar effect

Dit hoofdstuk bevat je literatuurstudie. De inhoud gaat verder op de inleiding, maar zal het onderwerp van de bachelorproef *diepgaand* uitspitten. De bedoeling is dat de lezer na lezing van dit hoofdstuk helemaal op de hoogte is van de huidige stand van zaken (state-of-the-art) in het onderzoeksdomein. Iemand die niet vertrouwd is met het onderwerp, weet nu voldoende om de rest van het verhaal te kunnen volgen, zonder dat die er nog andere informatie moet over opzoeken \autocite{Pollefliet2011}.

Je verwijst bij elke bewering die je doet, vakterm die je introduceert, enz.\ naar je bronnen. In \LaTeX{} kan dat met het commando \texttt{$\backslash${textcite\{\}}} of \texttt{$\backslash${autocite\{\}}}. Als argument van het commando geef je de ``sleutel'' van een ``record'' in een bibliografische databank in het Bib\LaTeX{}-formaat (een tekstbestand). Als je expliciet naar de auteur verwijst in de zin (narratieve referentie), gebruik je \texttt{$\backslash${}textcite\{\}}. Soms is de auteursnaam niet expliciet een onderdeel van de zin, dan gebruik je \texttt{$\backslash${}autocite\{\}} (referentie tussen haakjes). Dit gebruik je bv.~bij een citaat, of om in het bijschrift van een overgenomen afbeelding, broncode, tabel, enz. te verwijzen naar de bron. In de volgende paragraaf een voorbeeld van elk.

\textcite{Knuth1998} schreef een van de standaardwerken over sorteer- en zoekalgoritmen. Experten zijn het erover eens dat cloud computing een interessante opportuniteit vormen, zowel voor gebruikers als voor dienstverleners op vlak van informatietechnologie~\autocite{Creeger2009}.

Let er ook op: het \texttt{cite}-commando voor de punt, dus binnen de zin. Je verwijst meteen naar een bron in de eerste zin die erop gebaseerd is, dus niet pas op het einde van een paragraaf.

\begin{figure}
  \centering
  \includegraphics[width=0.8\textwidth]{grail.jpg}
  \caption[Voorbeeld figuur.]{\label{fig:grail}Voorbeeld van invoegen van een figuur. Zorg altijd voor een uitgebreid bijschrift dat de figuur volledig beschrijft zonder in de tekst te moeten gaan zoeken. Vergeet ook je bronvermelding niet!}
\end{figure}

\begin{listing}
  \begin{minted}{python}
    import pandas as pd
    import seaborn as sns

    penguins = sns.load_dataset('penguins')
    sns.relplot(data=penguins, x="flipper_length_mm", y="bill_length_mm", hue="species")
  \end{minted}
  \caption[Voorbeeld codefragment]{Voorbeeld van het invoegen van een codefragment.}
\end{listing}

\lipsum[7-20]

\begin{table}
  \centering
  \begin{tabular}{lcr}
    \toprule
    \textbf{Kolom 1} & \textbf{Kolom 2} & \textbf{Kolom 3} \\
    $\alpha$         & $\beta$          & $\gamma$         \\
    \midrule
    A                & 10.230           & a                \\
    B                & 45.678           & b                \\
    C                & 99.987           & c                \\
    \bottomrule
  \end{tabular}
  \caption[Voorbeeld tabel]{\label{tab:example}Voorbeeld van een tabel.}
\end{table}

