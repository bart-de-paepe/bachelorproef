\chapter{\IfLanguageName{dutch}{Stand van zaken}{State of the art}}%
\label{ch:stand-van-zaken}

% Tip: Begin elk hoofdstuk met een paragraaf inleiding die beschrijft hoe
% dit hoofdstuk past binnen het geheel van de bachelorproef. Geef in het
% bijzonder aan wat de link is met het vorige en volgende hoofdstuk.

% Pas na deze inleidende paragraaf komt de eerste sectiehoofding.
Het huidige digitale tijdperk dat hoofdzakelijk gekenmerkt wordt door de toenemende belangstelling in AI, levert eveneens een gigantische hoeveelheid aan online data. Met deze grote berg van informatie die beschikbaar is op de pagina's van het world wide web, is het bijzonder relevant voor businesses om te begrijpen hoe ze deze data kunnnen ontginnen teneinde er bruikbare informatie uit te filteren \autocite{Lofti2021}. Een belangrijk begrip dat daarbij prominent op de voorgrond treedt, is 'web scraping' of 'web crawling'. <<Web scraping is een geautomatiseerd data extractie proces van websites met gebruik van gespecialiseerde software>> \textcite{Bhatt2023}. Maar web scraping is lang niet de enige techniek die gebruikt wordt om data van een webpagina te halen \autocite{Gray2012}. Andere technieken zijn het gebruik van API's, screen readers en het ontleden van online PDF documenten. Het voordeel van web scraping is dat de website niet over een API met ruwe data hoeft te beschikken. De HTML code waarmee de website opgebouwd is, wordt gebruikt door de scraper om de eigenlijke inhoud eruit te filteren.\\
Web scrapers kunnen gebaseerd zijn op verschillende technologieën zoals 'spidering' en 'pattern matching'. Daarnaast biedt de programmeertaal waarin ze geïmplenteerd zijn ook telkens andere mogelijkheden \autocite{Bhatt2023}.
\textcite{Lotfi2021} onderscheidt web scrapers zowel op basis van hun toepassing (vb. medisch, social media, financieel, marketing, onderzoek) als op basis van hun  methodiek (vb. copy \& paste, HTML parsing, DOM parsing, HTML DOM, reguliere expressies, XPath, vetical aggregation platform, semantic annotation recognizing, computer vision web page analyzer). Hoewel de methodologie kan wijzigen, blijft het uitgangspunt van integriteit, correctheid en betrouwbaarheid van de data steeds van primordiaal belang \autocite{Lofti2021}.
Verder duidt \textcite{Lofti2021} ook nog op het iteratieve karakter van een web scaper. Het programma wordt gevoed met een url, en zal dan op zijn beurt nieuwe urls zoeken op de pagina zodat die ook bezocht kunnen worden.
\textcite{Singrodia2019} vult bovenstaande kenmerken van web scraping nog verder aan met de systematische omzetting van ongestructureerde gegevens op webpagina's naar gestructureerde data in een databank. De gegenereerde data heeft aanleiding tot filtering of statistieken. Het biedt vele voordelen aangezien de data opgeschoond is en daardoor als vrij van fouten kan beschouwd worden. Andere voordelen zijn tijdswinst en centrale opslag wat verdere verwerking ten goede komt.
\\
Na zo uitvoerig web scrapers te bespreken, is het niet onbelangrijk even stil te staan bij de vraag of web scrapen wel legaal is? Volgens \textcite{EPSI2015} is daar geen straightforward antwoord op, de situatie is afhankelijk van het betrokken land. Maar over het algemeen zijn de meest reguleringen wel in het voordeel van web scraping. Men moet vooral opletten met het repsecteren van het eigendomsrecht wanneer inhoud verworven wordt door scaping en vervolgens verder gebruikt wordt.\\
Tot slot van de stand van zake omtrent web scraping nog een pleidooi voor Python. Al is het perfect mogelijk om web scraping te ontwikkelen in Php, Java, en andere, toch biedt Python de meest gebruiksvriendelijke tools aan \textcite{Kumar2023}. <<Beautiful Soup is een van de eenvoudigste bibliotheken om data te scrapen van websites. Een simpele find\_all() in Beautiful Soup, is krachtig genoeg om de data uit het gehele document te doorzoeken. Daarna is de taak om de data te structureren. Dat kan in Python aan de hand van Pandas die in staat is om de data in een geordend formaat te presenteren.>>
De website in de spotlight voor dit onderzoek is Google Scholar (GS), de grootste bron van wetenschappelijke publicaties op heden. De beta versie verscheen in 2004 en sindsdien wordt het systeem voornamelijk door academici gebruikt om een persoonlijke bibliotheek aan te leggen tezamen met statistieken omtrent citaties en h-indexen \footnote{De h-index van een wetenschappelijk onderzoeker komt overeen met de grootste h van het aantal publicaties die minstens h keer geciteerd zijn in ander werk.}. Volgens \autocite{Oh2019} is er een symbiose tussen GS and de academische wereld waarbij academici gratis hun werk kunnen aabieden op GS. GS op zijn beurt zorgt voor goede visibiliteit van dat werk door onder andere citaten, referenties en gelijkaardige onderwerpen. In tegenstelling tot de baanbrekende vernieuwing die GS introduceerde, gaat het raadplegen van deze gegevens wel gepaard met een aantal hindernissen, in het bijzonder op grote schaal.\\
Web scraping is 1 van de technieken die aangewend worden om dit probleem te omzeilen. Aangezien deze techniek inwerkt op de HTML structuur van de webpagina, kan ze gebruikt worden voor alle websites en is ze niet afhankelijk van de structuur van de onderliggende ruwe data. Daarenbover vergroot web scraping de snelheid en het volume van de data die verwerkt kan worden aanzienlijk, en vermindert het ook het aantal fouten die zouden optreden door menselijke verwerking \autocite{Bhatt2023}. Uiteindelijk zet web scraping de online informatie om in business intelligence afhankelijk van het uitgangspunt van de eindgebruiker.\\
Het is evident dat HTML pagina's verwerkt kunnen worden door web scrapers, maar daarnaast zijn ongeveer 70\% van de feiten die op het internet gepresenteerd worden, verkregen uit PDF-documenten \autocite{Singrodia2019}. Dit verklaart de noodzaak om zowel HTML pagina's als PDF documenten te scrapen.\\



Efficiënt gebruik van gepubliceerde gegevens is cruciaal voor het onderbouwen van data analyse en beslissingen gebaseerd op die data \autocite{NurDalila2023}. Dergelijke tools verbeteren niet alleen de efficiëntie van het collecteren van data, maar verruimen ook de capaciteit van de gebruikers om de data te beheren en te analyseren \textcite{Toth2024}\\


Om verder gebruik te maken van deze data moet deze eerst gestuctureerd opgeslaan worden. \textcite{Mitchell2015} reikt een aantal methodes aan waarop dit gedaan kan worden, steeds afhankelijk van het beoogde resultaat.\\
Meerdere studies (\autocite{Pratiba2018},\autocite{Rafsanjani2022},\autocite{Amin2024},\autocite{Sulistya2024}) gebruiken bovenstaande concepten van web scrapers en gestructureerde opslag om automatisch data te ontleden van GS aan de hand van gebruiksvriendelijke interfaces. Er zijn onderling steeds wel verschillen tussen de gebruikte tools en het gekozen formaat die beide afhankelijk zijn van het beoogde resultaat. Maar het omliggend kader is steeds hetzelfde met gebruik van web scraping en gestructureerde opslag.\\
\textcite{Yang2017} beschrijft welke HTML elementen belangrijk zijn bij het ontleden van een GS pagina.

ToDo Google Scholar\\


Parsen van Google Scholar zoekresultaten:
\autocite{Kim2019} stelt vast dat gebruikers meer interesse hebben in de weergegeven inhoud dan in de andere elementen zoals auteur, aantal citaties, en andere.
Een vooraf gedefinieerde zoekopdracht levert meer relevante zoekresultaten dan een instant zoekopdracht door een gebruiker op de user interface \autocite{Zhang2013}
\textcite{Wu2014} stelt vast dat studenten een voorkeur hebben voor Google Scholar als zoekmachine omwille van de vertrouwde interface.
ToDo check Khabsa M. 2017 The number of scholarly documents on the public web
ToDo check Serenko A. 2015 Citation classics published in knowledge management journals. Part ii: studying research trends and discovering the Google Scholar effect

Dit hoofdstuk bevat je literatuurstudie. De inhoud gaat verder op de inleiding, maar zal het onderwerp van de bachelorproef *diepgaand* uitspitten. De bedoeling is dat de lezer na lezing van dit hoofdstuk helemaal op de hoogte is van de huidige stand van zaken (state-of-the-art) in het onderzoeksdomein. Iemand die niet vertrouwd is met het onderwerp, weet nu voldoende om de rest van het verhaal te kunnen volgen, zonder dat die er nog andere informatie moet over opzoeken \autocite{Pollefliet2011}.

Je verwijst bij elke bewering die je doet, vakterm die je introduceert, enz.\ naar je bronnen. In \LaTeX{} kan dat met het commando \texttt{$\backslash${textcite\{\}}} of \texttt{$\backslash${autocite\{\}}}. Als argument van het commando geef je de ``sleutel'' van een ``record'' in een bibliografische databank in het Bib\LaTeX{}-formaat (een tekstbestand). Als je expliciet naar de auteur verwijst in de zin (narratieve referentie), gebruik je \texttt{$\backslash${}textcite\{\}}. Soms is de auteursnaam niet expliciet een onderdeel van de zin, dan gebruik je \texttt{$\backslash${}autocite\{\}} (referentie tussen haakjes). Dit gebruik je bv.~bij een citaat, of om in het bijschrift van een overgenomen afbeelding, broncode, tabel, enz. te verwijzen naar de bron. In de volgende paragraaf een voorbeeld van elk.

\textcite{Knuth1998} schreef een van de standaardwerken over sorteer- en zoekalgoritmen. Experten zijn het erover eens dat cloud computing een interessante opportuniteit vormen, zowel voor gebruikers als voor dienstverleners op vlak van informatietechnologie~\autocite{Creeger2009}.

Let er ook op: het \texttt{cite}-commando voor de punt, dus binnen de zin. Je verwijst meteen naar een bron in de eerste zin die erop gebaseerd is, dus niet pas op het einde van een paragraaf.

\begin{figure}
  \centering
  \includegraphics[width=0.8\textwidth]{grail.jpg}
  \caption[Voorbeeld figuur.]{\label{fig:grail}Voorbeeld van invoegen van een figuur. Zorg altijd voor een uitgebreid bijschrift dat de figuur volledig beschrijft zonder in de tekst te moeten gaan zoeken. Vergeet ook je bronvermelding niet!}
\end{figure}

\begin{listing}
  \begin{minted}{python}
    import pandas as pd
    import seaborn as sns

    penguins = sns.load_dataset('penguins')
    sns.relplot(data=penguins, x="flipper_length_mm", y="bill_length_mm", hue="species")
  \end{minted}
  \caption[Voorbeeld codefragment]{Voorbeeld van het invoegen van een codefragment.}
\end{listing}

\begin{table}
  \centering
  \begin{tabular}{lcr}
    \toprule
    \textbf{Kolom 1} & \textbf{Kolom 2} & \textbf{Kolom 3} \\
    $\alpha$         & $\beta$          & $\gamma$         \\
    \midrule
    A                & 10.230           & a                \\
    B                & 45.678           & b                \\
    C                & 99.987           & c                \\
    \bottomrule
  \end{tabular}
  \caption[Voorbeeld tabel]{\label{tab:example}Voorbeeld van een tabel.}
\end{table}

