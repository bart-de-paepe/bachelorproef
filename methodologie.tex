%%=============================================================================
%% Methodologie
%%=============================================================================

\chapter{\IfLanguageName{dutch}{Methodologie}{Methodology}}%
\label{ch:methodologie}

%% TODO: In dit hoofstuk geef je een korte toelichting over hoe je te werk bent
%% gegaan. Verdeel je onderzoek in grote fasen, en licht in elke fase toe wat
%% de doelstelling was, welke deliverables daar uit gekomen zijn, en welke
%% onderzoeksmethoden je daarbij toegepast hebt. Verantwoord waarom je
%% op deze manier te werk gegaan bent.
%% 
%% Voorbeelden van zulke fasen zijn: literatuurstudie, opstellen van een
%% requirements-analyse, opstellen long-list (bij vergelijkende studie),
%% selectie van geschikte tools (bij vergelijkende studie, "short-list"),
%% opzetten testopstelling/PoC, uitvoeren testen en verzamelen
%% van resultaten, analyse van resultaten, ...
%%
%% !!!!! LET OP !!!!!
%%
%% Het is uitdrukkelijk NIET de bedoeling dat je het grootste deel van de corpus
%% van je bachelorproef in dit hoofstuk verwerkt! Dit hoofdstuk is eerder een
%% kort overzicht van je plan van aanpak.
%%
%% Maak voor elke fase (behalve het literatuuronderzoek) een NIEUW HOOFDSTUK aan
%% en geef het een gepaste titel.
Om nieuwe publicaties die gerelateerd zijn aan wetenschappelijke projecten toe te voegen aan IMIS, moet er natuurlijk een signaal zijn wanneer dergelijke publicaties beschikbaar zijn. Indexen van academische literatuur zijn daarvoor een geschikte bron en zoals eerder beschreven is Google Scholar een waardevolle optie. Het is de bedoeling om enkel nieuwe resultaten te ontvangen en bijgevolg vervalt dus de optie om publicaties op te zoeken aan de hand van de Google Scholar zoekpagina. Google Scholar laat echter toe om meldingen aan te maken voor een bepaalde zoekopdracht. Die stuurt automatisch nieuwe resultaten door per e-mail onder de vorm van de Google Scholar SERP.\\
Google Scholar meldingen zijn e-mails in HTML formaat. Ze bevatten inhoud die overeenkomt met de Google Scholar SERP overeenkomstig de zoekopdracht. De SERP bevat een vaste structuur. Het is een lijst met zoekresultaten die telkens dezelfde elementen bevatten:
\begin{itemize}
    \item titel
    \item link naar de webpagina van de publicatie
    \item auteurs
    \item tijdschrift
    \item jaartal
    \item abstract van de publicatie of een fragment ervan
\end{itemize}
Die HTML moet omgezet worden in gestructureerde data door middel van HTML scraping technieken:
\begin{itemize}
    \item HTML scraping door een LLM
    \begin{itemize}
        \item online model
        \begin{itemize}
            \item betalend model
            \item niet betalend model
        \end{itemize}
        \item lokaal model
    \end{itemize}
    \item HTML scraping door het ontleden van de DOM
    \begin{itemize}
        \item online scraper
        \item lokale scraper
    \end{itemize}
\end{itemize}\\
Om te weten of een publicatie echt nieuw is, moet ze ondubbelzijdig geïdentificeerd kunnen worden. Titel, auteurs, tijdschrift, enz. maken een publicatie echter niet uniek. Er kunnen namelijk variante benamingen voorvallen van titels, auteurs, enz. Voor literatuur is het de DOI die een publicatie uniek maakt. Er wordt op zoek gegaan naar de DOI aan de hand van een stapsgewijze procedure:
\begin{itemize}
    \item DOI opzoeken in de link naar de webpagina van de publicatie
    \item DOI opzoeken in Crossref op basis van titel, auteurs, tijdschrift en jaartal
    \item DOI opzoeken in de webpagina van de publicatie
    \begin{itemize}
        \item dit kan een HTML pagina zijn
        \item dit kan een PDF document zijn
        \item dit kan een HTML pagina zijn met een embedded PDF document
    \end{itemize}
\end{itemize}\\
Dan resteert de vraag of de publicatie echt nieuw is, of dat ze reeds aan IMIS toegevoegd werd? Het antwoord daarop is afhankelijk van de beschikbare informatie die tijdens de voorgaande stappen gevonden werd.
\begin{itemize}
    \item De DOI is gevonden: er kan met 100\% zekerheid opgezocht worden of de publicatie reeds in IMIS zit of niet.
    \item De DOI is niet gevonden: omwille van de variante benamingen kan niet met volledige zekerheid opgezocht worden of een publicatie reeds in IMIS zit of niet. Wel kan door middel van semantic search de waarschijnlijkheid berekend worden dat de publicatie reeds in IMIS zit op basis van de titel, auteurs, tijdschrift en jaartal. Daarvoor moeten de embeddings van deze gegevens van de publicatie vergeleken worden met de embeddings van alle publicatie in IMIS. De mate van overeenkomst tussen de embeddings geeft een probabiliteitswaarde voor de aanwezigheid van de publicatie in IMIS.
\end{itemize}
\section{Google Scholar zoekopdracht}
GS biedt een aantal filtermogelijkheden om de meest relevante resultaten te bekomen voor de zoekopdracht. De filter moet manueel geconfigureerd worden voor elke zoekopdracht en vervolgens kan er een e-mail alert ingesteld worden per zoekopdracht.
\section{Parsen van Google Scholar zoekresultaat}
Er moet een geautomatiseerd systeem komen dat 
\begin{itemize}
  \item de binnenkomende e-mail alerts in een mailbox leest
  \item vervolgens de zoekresultaten in de body parset
  \item tenslotte voor elk zoekresultaat de overeenkomstige link bezoekt teneinde de Digital Object Indentifier (DOI) op te zoeken.
\end{itemize}
Daarvoor moet er custom code geschreven worden en bijhorend moet de vraag gesteld worden welke technology stack er gebruikt zal worden? Daarbij zijn 2 aspecten belangrijk:
\begin{itemize}
    \item Wat is de bestaande technology stack van de klant en kan de te ontwikkelen software daarin ondergebracht worden?
    \item Welke technologie is het meest geschikt om het gestelde probleem op te lossen?
\end{itemize}
Voor het eerste punt zijn er bij de klant 2 pijlers:
\begin{itemize}
    \item Alle websites en datasystemen zijn geschreven in Php, al dan niet gebruik makend van het Symfony framework. Voor hun data steunen ze voornamelijk op SqlServer en op PostgreSQL. Voor hun user interface steunen ze hoofdzakelijk op Twig en op NextJS.
    \item Veel data processing scripts draaien in hun eigen docker container. Ze gebruiken vooral R en Python als programmeertaal.
\end{itemize}
\begin{figure}[H]
    \centering
    \includegraphics[width=0.8\textwidth]{./2_parse_zoekresultaat/technology-stack.jpg}
    \caption[Technology Stack.]{\label{fig:Technology Stack}Technology Stack.}
\end{figure}
Onze opdracht past het best binnen de tweede pijler. Het is niet opportuun om een webserver te belasten met het systematisch herhalen van een opdracht die los staat van de websites die hij host. De GS alerts zijn op zich ook data die in hun eigen container verwerkt zullen worden.\\
Voor het tweede punt is de technologie zeer duidelijk. Python (maar ook R) beschikken over hele grote bibliotheken die toelaten om uiteenlopende soorten data te verwerken.\\
Het programmeerwerk zal uitgevoerd worden in Python. Op het moment van schrijven is Python versie 3.13.1 de meest recente stabiele versie.
\begin{figure}[H]
    \centering
    \includegraphics[width=0.8\textwidth]{./2_parse_zoekresultaat/1_python_release_cycle.jpg}
    \caption[Python release cycle.]{\label{fig:Python release cycle}Python release cycle.\autocite{pythonreleasecycle2025}}
\end{figure}
\FloatBarrier
De geparsete e-mail alerts worden opgeslagen in een MongoDB database, dit is een NoSQL document store databank. Een RabbitMQ broker zorgt ervoor dat voor elke e-mail iedere stap in de flow doorlopen wordt. Tenslotte wordt alles ondergebracht in zijn eigen Docker container die binnen het project met elkaar communiceren.
\section{Crossref}
Crossref biedt verschillende API's aan waar de metadata van wetenschappelijke publicaties opgevraagd kan worden op basis van de DOI.
\section{Natural language processing}
Wanneer het abstract van de publicatie beschikbaar is, wordt die gebruikt om het artikel te classificeren binnen 1 van de collecties van IMIS. Dat kan met behulp van een vector databank. Daarvoor moeten eerst alle bestaande publicaties van een collectie omgezet worden naar een vector. Dat gebeurt op basis van een LLM. Die vectors worden opgeslaan in een vector database. Voor elk nieuw artikel wordt de overeenkomstige vector berekend van het abstract aan de hand van hetzelfde model. Vervolgens vertelt de positie van de vector ten opzichte van andere vectoren ons tot welke collectie het artikel behoort.
\section{IMIS-input}
\lipsum[1-1]



