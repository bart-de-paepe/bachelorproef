%%=============================================================================
%% Methodologie
%%=============================================================================

\chapter{\IfLanguageName{dutch}{Methodologie}{Methodology}}%
\label{ch:methodologie}

%% TODO: In dit hoofstuk geef je een korte toelichting over hoe je te werk bent
%% gegaan. Verdeel je onderzoek in grote fasen, en licht in elke fase toe wat
%% de doelstelling was, welke deliverables daar uit gekomen zijn, en welke
%% onderzoeksmethoden je daarbij toegepast hebt. Verantwoord waarom je
%% op deze manier te werk gegaan bent.
%% 
%% Voorbeelden van zulke fasen zijn: literatuurstudie, opstellen van een
%% requirements-analyse, opstellen long-list (bij vergelijkende studie),
%% selectie van geschikte tools (bij vergelijkende studie, "short-list"),
%% opzetten testopstelling/PoC, uitvoeren testen en verzamelen
%% van resultaten, analyse van resultaten, ...
%%
%% !!!!! LET OP !!!!!
%%
%% Het is uitdrukkelijk NIET de bedoeling dat je het grootste deel van de corpus
%% van je bachelorproef in dit hoofstuk verwerkt! Dit hoofdstuk is eerder een
%% kort overzicht van je plan van aanpak.
%%
%% Maak voor elke fase (behalve het literatuuronderzoek) een NIEUW HOOFDSTUK aan
%% en geef het een gepaste titel.
Om nieuwe publicaties die gerelateerd zijn aan wetenschappelijke projecten toe te voegen aan IMIS, moet er natuurlijk een signaal zijn wanneer dergelijke publicaties beschikbaar zijn. Indexen van academische literatuur zijn daarvoor een geschikte bron en zoals eerder beschreven is Google Scholar een waardevolle optie. Het is de bedoeling om enkel nieuwe resultaten te ontvangen en bijgevolg vervalt dus de optie om publicaties op te zoeken aan de hand van de Google Scholar zoekpagina. Google Scholar laat echter toe om meldingen aan te maken voor een bepaalde zoekopdracht. Die stuurt automatisch nieuwe resultaten door per e-mail onder de vorm van de Google Scholar SERP. Dit wordt verder uitgewerkt in Hoofdstuk~\ref{ch:googlescholaralert}.\\
Google Scholar meldingen zijn e-mails in HTML formaat. Ze bevatten inhoud die overeenkomt met de Google Scholar SERP overeenkomstig de zoekopdracht. De SERP bevat een vaste structuur. Het is een lijst met zoekresultaten die telkens dezelfde elementen bevatten:
\begin{itemize}
    \item titel
    \item link naar de webpagina van de publicatie
    \item auteurs
    \item tijdschrift
    \item jaartal
    \item abstract van de publicatie of een fragment ervan
\end{itemize}
Die HTML moet omgezet worden in gestructureerde data door middel van HTML scraping technieken:
\begin{itemize}
    \item HTML scraping door een LLM \footnote{Large Language Model}
    \begin{itemize}
        \item online model
        \begin{itemize}
            \item OpenAI
            \item Generieke procedure onafhankelijk van het model
        \end{itemize}
        \item lokaal model
    \end{itemize}
    \item HTML scraping door het parsen van de DOM
    \begin{itemize}
        \item Beautifulsoup
        \item SerpAPI
    \end{itemize}
\end{itemize}
Dit wordt verder uitgewerkt in Hoofdstuk~\ref{ch:web_scraping}.\\
Om te weten of een publicatie echt nieuw is, moet ze ondubbelzijdig geïdentificeerd kunnen worden. Titel, auteurs, tijdschrift, enz. maken een publicatie echter niet uniek. Er kunnen namelijk variante benamingen voorkomen van titels, auteurs, enz. Voor literatuur is het de DOI die een publicatie uniek maakt. Er wordt op zoek gegaan naar de DOI aan de hand van een stapsgewijze procedure:
\begin{itemize}
    \item DOI opzoeken in de link naar de webpagina van de publicatie
    \item DOI opzoeken in Crossref op basis van titel, auteurs, tijdschrift en jaartal
    \item DOI opzoeken in de webpagina van de publicatie
    \begin{itemize}
        \item dit kan een HTML pagina zijn
        \item dit kan een PDF document zijn
        \item dit kan een HTML pagina zijn met een embedded PDF document
    \end{itemize}
\end{itemize}
Dit wordt verder uitgewerkt in Hoofdstuk~\ref{ch:linked_data}.\\
Dan resteert de vraag of de publicatie echt nieuw is, of dat ze reeds aan IMIS toegevoegd werd? Het antwoord daarop is afhankelijk van de beschikbare informatie die tijdens de voorgaande stappen gevonden werd.
\begin{itemize}
    \item De DOI is gevonden: er kan met 100\% zekerheid opgezocht worden of de publicatie reeds in IMIS zit of niet.
    \item De DOI is niet gevonden: omwille van de variante benamingen kan niet met volledige zekerheid opgezocht worden of een publicatie reeds in IMIS zit of niet. Wel kan door middel van semantic search de waarschijnlijkheid berekend worden dat de publicatie reeds in IMIS zit op basis van de titel, auteurs, tijdschrift en jaartal. Daarvoor moeten de embeddings van deze gegevens van de publicatie vergeleken worden met de embeddings van alle publicatie in IMIS. De mate van overeenkomst tussen de embeddings geeft een probabiliteitswaarde voor de aanwezigheid van de publicatie in IMIS.
\end{itemize}
Dit wordt verder uitgewerkt in Hoofdstuk~\ref{ch:semantic_search}.




