%%=============================================================================
%% Methodologie
%%=============================================================================

\chapter{\IfLanguageName{dutch}{Methodologie}{Methodology}}%
\label{ch:methodologie}

%% TODO: In dit hoofstuk geef je een korte toelichting over hoe je te werk bent
%% gegaan. Verdeel je onderzoek in grote fasen, en licht in elke fase toe wat
%% de doelstelling was, welke deliverables daar uit gekomen zijn, en welke
%% onderzoeksmethoden je daarbij toegepast hebt. Verantwoord waarom je
%% op deze manier te werk gegaan bent.
%% 
%% Voorbeelden van zulke fasen zijn: literatuurstudie, opstellen van een
%% requirements-analyse, opstellen long-list (bij vergelijkende studie),
%% selectie van geschikte tools (bij vergelijkende studie, "short-list"),
%% opzetten testopstelling/PoC, uitvoeren testen en verzamelen
%% van resultaten, analyse van resultaten, ...
%%
%% !!!!! LET OP !!!!!
%%
%% Het is uitdrukkelijk NIET de bedoeling dat je het grootste deel van de corpus
%% van je bachelorproef in dit hoofstuk verwerkt! Dit hoofdstuk is eerder een
%% kort overzicht van je plan van aanpak.
%%
%% Maak voor elke fase (behalve het literatuuronderzoek) een NIEUW HOOFDSTUK aan
%% en geef het een gepaste titel.
\section{Google Scholar zoekopdracht}
GS biedt een aantal filtermogelijkheden om de meest relevante resultaten te bekomen voor de zoekopdracht. De filter moet manueel geconfigureerd worden voor elke zoekopdracht en vervolgens kan er een e-mail alert ingesteld worden per zoekopdracht.
\section{Parsen van Google Scholar zoekresultaat}
Er moet een geautomatiseerd systeem komen dat 
\begin{itemize}
  \item de binnenkomende e-mail alerts in een mailbox leest
  \item vervolgens de zoekresultaten in de body parset
  \item tenslotte voor elk zoekresultaat de overeenkomstige link bezoekt teneinde de Digital Object Indentifier (DOI) op te zoeken.
\end{itemize}
Daarvoor moet er custom code geschreven worden en bijhorend moet de vraag gesteld worden welke technology stack er gebruikt zal worden? Daarbij zijn 2 aspecten belangrijk:
\begin{itemize}
    \item Wat is de bestaande technology stack van de klant en kan de te ontwikkelen software daarin ondergebracht worden?
    \item Welke technologie is het meest geschikt om het gestelde probleem op te lossen?
\end{itemize}
Voor het eerste punt zijn er bij de klant 2 pijlers:
\begin{itemize}
    \item Alle websites en datasystemen zijn geschreven in Php, al dan niet gebruik makend van het Symfony framework. Voor hun data steunen ze voornamelijk op SqlServer en op PostgreSQL. Voor hun user interface steunen ze hoofdzakelijk op Twig en op NextJS.
    \item Veel data processing scripts draaien in hun eigen docker container. Ze gebruiken vooral R en Python als programmeertaal.
\end{itemize}
Onze opdracht past het best binnen de tweede pijler. Het is niet opportuun om een webserver te belasten met het systematisch herhalen van een opdracht die los staat van de websites die hij host. De GS alerts zijn op zich ook data die in hun eigen container verwerkt zullen worden.\\
Voor het tweede punt is de technologie zeer duidelijk. Python (maar ook R) beschikken over hele grote bibliotheken die toelaten om uiteenlopende soorten data te verwerken.\\
Het programmeerwerk zal uitgevoerd worden in Python. Op het moment van schrijven is Python versie 3.13.1 de meest recente stabiele versie.
\begin{figure}
    \centering
    \includegraphics[width=0.8\textwidth]{./2_parse_zoekresultaat/1_python_release_cycle.jpg}
    \caption[Python release cycle.]{\label{fig:Python release cycle}Python release cycle.\autocite{pythonreleasecycle2025}}
\end{figure}
\begin{figure}
    \centering
    \includegraphics[width=0.8\textwidth]{./2_parse_zoekresultaat/2_end_of_life_date_python.JPG}
    \caption[End of life date Python.]{\label{fig:End of life date Python}End of life date Python.\autocite{endoflifedatepython2025}}
\end{figure}
\FloatBarrier
De geparsete e-mail alerts worden opgeslagen in een MongoDB database, dit is een NoSQL document store databank. Een RabbitMQ broker zorgt ervoor dat voor elke e-mail iedere stap in de flow doorlopen wordt. Tenslotte wordt alles ondergebracht in zijn eigen Docker container die binnen het project met elkaar communiceren.
\section{Crossref}
\lipsum[1-1]
\section{Natural language processing}
\lipsum[1-1]
\section{IMIS-input}
\lipsum[1-1]


