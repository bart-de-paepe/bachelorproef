%%=============================================================================
%% Methodologie
%%=============================================================================

\chapter{\IfLanguageName{dutch}{Google Scholar alert}{Google Scholar alerts}}%
\label{ch:googlescholaralert}

%% TODO: In dit hoofstuk geef je een korte toelichting over hoe je te werk bent
%% gegaan. Verdeel je onderzoek in grote fasen, en licht in elke fase toe wat
%% de doelstelling was, welke deliverables daar uit gekomen zijn, en welke
%% onderzoeksmethoden je daarbij toegepast hebt. Verantwoord waarom je
%% op deze manier te werk gegaan bent.
%% 
%% Voorbeelden van zulke fasen zijn: literatuurstudie, opstellen van een
%% requirements-analyse, opstellen long-list (bij vergelijkende studie),
%% selectie van geschikte tools (bij vergelijkende studie, "short-list"),
%% opzetten testopstelling/PoC, uitvoeren testen en verzamelen
%% van resultaten, analyse van resultaten, ...
%%
%% !!!!! LET OP !!!!!
%%
%% Het is uitdrukkelijk NIET de bedoeling dat je het grootste deel van de corpus
%% van je bachelorproef in dit hoofstuk verwerkt! Dit hoofdstuk is eerder een
%% kort overzicht van je plan van aanpak.
%%
%% Maak voor elke fase (behalve het literatuuronderzoek) een NIEUW HOOFDSTUK aan
%% en geef het een gepaste titel.

\section{Instellen van de zoekopdract \& aanmaken van het alert}
Om op de hoogte te blijven van nieuwe publicaties kan een index van academische literatuur gebruikt worden. Zoals eerder beschreven is Google Scholar (GS) daarvoor een interessante bron.\\
Het is de bedoeling om alleen nieuwe publicaties te ontvangen die nog niet eerder opgezocht werden. Bijgevolg wordt er beter geen gebruik gemaakt van de GS zoekpagina, omdat die dezelfde publicaties kan tonen bij opeenvolgende opzoekingen. Daarentegen zijn er GS alerts. Dat zijn meldingen onder de vorm van automatische e-mails waarin de nieuwe publicaties opgelijst staan. Aangezien de meldingen steeds gekoppeld zijn aan een account, houdt het systeem rekening met de zoekgeschiedenis en zijn de zoekresultaten incrementeel.\\
Bijlage ``Google Scholar zoekopdracht''\ref{ch:googlescholarzoekopdracht} geeft een uitvoerige uiteenzetting over het opmaken van een zoekopdracht en het instellen van een melding.\\
In het kader van deze opdracht werd een nieuw account \textbf{google-scholar@marineinfo.org} aangemaakt, als een gedeeld account waar meerdere gebruikers toegang toe hebben. Daarna werd een Google account aangemaakt met hetzelfde e-mailadres. Vervolgens werden meerdere zoekopdrachten opgemaakt telkens voor een bepaald project, en werden deze geactiveerd als meldingen.\\
Bijvoorbeeld om relevante publicaties over het VLIZ te vinden, worden volgende zoektermen gebruikt:
\begin{itemize}
    \item Vlaams Instituut voor de Zee
    \item Vlaams Instituut van de Zee
    \item Flanders Marine Institute
    \item VLIZ
    \item Simon Stevin
    \item R/V Simon Stevin
    \item RV Simon Stevin
    \item Marine Station Ostend
    \item Mariene Station Oostende
\end{itemize}

Vanaf dat moment zal Google Scholar voor elke zoekopdracht dagelijks e-mails met nieuwe zoekresultaten sturen naar het e-mailadres van het account, zolang de melding geactiveerd is.