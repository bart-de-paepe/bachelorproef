%%=============================================================================
%% Methodologie
%%=============================================================================

\chapter{\IfLanguageName{dutch}{Parsen van Google Scholar zoekresultaten}{Parse Google Scholar search results}}%
\label{ch:parsezoekresultaat}

%% TODO: In dit hoofstuk geef je een korte toelichting over hoe je te werk bent
%% gegaan. Verdeel je onderzoek in grote fasen, en licht in elke fase toe wat
%% de doelstelling was, welke deliverables daar uit gekomen zijn, en welke
%% onderzoeksmethoden je daarbij toegepast hebt. Verantwoord waarom je
%% op deze manier te werk gegaan bent.
%% 
%% Voorbeelden van zulke fasen zijn: literatuurstudie, opstellen van een
%% requirements-analyse, opstellen long-list (bij vergelijkende studie),
%% selectie van geschikte tools (bij vergelijkende studie, "short-list"),
%% opzetten testopstelling/PoC, uitvoeren testen en verzamelen
%% van resultaten, analyse van resultaten, ...
%%
%% !!!!! LET OP !!!!!
%%
%% Het is uitdrukkelijk NIET de bedoeling dat je het grootste deel van de corpus
%% van je bachelorproef in dit hoofstuk verwerkt! Dit hoofdstuk is eerder een
%% kort overzicht van je plan van aanpak.
%%
%% Maak voor elke fase (behalve het literatuuronderzoek) een NIEUW HOOFDSTUK aan
%% en geef het een gepaste titel.
\section{Doelstelling}
De Google Scholar alert geeft aanleiding tot e-mails met de meest recente zoekresultaten. Voor elke e-mail en voor elk zoekresultaat wordt de Digital Object Identifier (DOI) opgezocht. Daarvoor worden 3 stappen doorlopen:
\begin{itemize}
    \item E-mail lezen
    \item Body van de e-mail parsen
    \item DOI opzoeken in de link die aanwezig is in het geparsete resultaat
\end{itemize}
\subsection{E-mail lezen}
Op gezette tijdstippen wordt de inbox van het account google-scholar@marineinfo.org geopend op zoek naar ongelezen e-mails. Elke e-mail wordt gelezen en de informatie van de e-mail wordt opgeslaan. De e-mail wordt verplaatst van de inbox naar een werkmap overeenkomstig het onderwerp van de e-mail. Het onderwerp komt overeen met de zoekopdracht.
\begin{figure}
    \centering
    \includegraphics[width=0.8\textwidth]{./2_parse_zoekresultaat/email-lezen.jpg}
    \caption[E-mail lezen.]{\label{fig:E-mail lezen}Flow e-mail lezen.}
\end{figure}
\FloatBarrier
\subsection{Body van de e-mail parsen}
\subsection{DOI opzoeken in de link van het geparsete resultaat}
\section{Analyse}
\subsection{Use Cases}
\begin{figure}
    \centering
    \includegraphics[width=0.8\textwidth]{./2_parse_zoekresultaat/3_use_case_diagram.PNG}
    \caption[Use case diagram.]{\label{fig:Use case diagram}Use case diagram.}
\end{figure}
\FloatBarrier
\begin{table}[H]
    \centering
    \begin{tabular}{ | m{5cm} | m{10cm}| } 
        \hline
        \rowcolor{lightgray}
        beschrijving & Google Scholar stuurt e-mails naar google-scholar@vliz.be. Deze e-mails moeten automatisch verwerkt worden. \\ 
        \hline
        primary actor & - \\ 
        \hline
        stakeholders & - \\ 
        \hline
        pre-condities & onverwerkte e-mail \\ 
        \hline
        post-condities & verwerkte e-mails \\ 
        \hline
        normaal verloop & 
        \begin{enumerate}
            \item systeem opent e-mail
            \item systeem leest afzender
            \item systeem evalueert afzender (DR-email)
            \item systeem leest datum / tijdstip
            \item systeem leest onderwerp
            \item systeem leest body
            \item systeem slaat deze gegevens op
            \item systeem verplaatst e-mail van inbox naar mailbox folder op basis van onderwerp
        \end{enumerate} \\ 
        \hline
        alternatief verloop & 
        \begin{description}
            \item 3A. de afzender is verkeerd (DR-email)
            \item 3A1. systeem verplaatst e-mail van inbox naar SPAM folder
        \end{description} \\ 
        \hline
        domeinregels & DR-email: afzender moet zijn: scholaralerts-noreply@google.com \\ 
        \hline
    \end{tabular}
    \caption{Table to test captions and labels.}
\end{table}

\subsection{Domeinmodel}
De 3 bovenstaande stappen geven aanleiding tot 4 objecten die de kern van het domeinmodel uitmaken en de nodige properties bevatten om de tussenresultaten bij te houden alsook de uiteindelijke DOI.
\begin{figure}
    \centering
    \includegraphics[width=0.8\textwidth]{./2_parse_zoekresultaat/4_domeinmodel.PNG}
    \caption[Domeinmodel.]{\label{fig:Domeinmodel}Domeinmodel.}
\end{figure}
\FloatBarrier
\section{Implementatie}
\subsection{E-mail lezen}
Meerdere specifieke Python libraries ondersteunen alle mogelijke bewerkingen met e-mail:
\begin{itemize}
    \item \textbf{imaplib - IMAP4 protocol client \autocite{Imaplib2025}}: Legt een verbinding met een IMAP server en implementeert de functionaliteiten van het IMAP \footnote{Internet Message Access Protocol} protocol om e-mails op te halen.\autocite{Imap2025}
    \item \textbf{email - An email and MIME handling package \autocite{Email2025}}: Laat toe om emails te verwerken adhv. 3 modules.
    \begin{itemize}
        \item \textbf{message}: Alle bewerkingen met een e-mail.
        \item \textbf{parser}: Om een e-mail om te zetten in tekst.
        \item \textbf{generator}: Om tekst om te zetten in een e-mail. 
    \end{itemize}
\end{itemize}
De gelezen e-mail wordt opgeslaan in een document store database \autocite{Documentstore2025}. Dit type databank is geschikt voor het bijhouden van dynamische data die als geheel (als één document) opgeslaan wordt. MongoDB \autocite{Mongodb2025} is de meest gekende technologie van dit type databank. Pymongo \autocite{Pymongo2025} is de specifieke python library die de verbinding met de Mongodb server mogelijk maakt.\\
Het custom script dat de e-mail leest wordt opgeroepen door de Windows task scheduler. Click\_ \autocite{Click2025} laat toe om meerdere commando's te definiëren in hetzelfde script.\\
Tenslotte wordt er gebruik gemaakt van de Dependency Injector \autocite{Dependencyinjector2025} om al deze services door middel van dependency injection \autocite{Di2025} ter beschikking te stellen aan het script.\\
Aan het einde van de process dat de e-mail leest, wordt de body van de e-mail in een wachtrij geplaatst zodat deze opgepikt wordt door de volgende stappen.
\section{Resultaat}
\subsection{E-mail lezen}
Het resultaat van deze stap is eerder een tussenresultaat waarbij verschillende onderdelen van de e-mail zoals datum, onderwerp, en andere opgeslagen worden in de databank. Deze opgeslagen informatie vormt de bron voor de verdere verwerking in stap 2.
\begin{figure}
    \centering
    \includegraphics[width=0.8\textwidth]{./2_parse_zoekresultaat/json_view.JPG}
    \caption[Databank JSON view.]{\label{fig:Databank JSON view}Databank JSON view.}
\end{figure}
\begin{figure}
    \centering
    \includegraphics[width=0.8\textwidth]{./2_parse_zoekresultaat/table_view.JPG}
    \caption[Databank table view.]{\label{fig:Databank table view}Databank table view.}
\end{figure}
\FloatBarrier



