%%=============================================================================
%% Conclusie
%%=============================================================================

\chapter{Conclusie}%
\label{ch:conclusie}

% TODO: Trek een duidelijke conclusie, in de vorm van een antwoord op de
% onderzoeksvra(a)g(en). Wat was jouw bijdrage aan het onderzoeksdomein en
% hoe biedt dit meerwaarde aan het vakgebied/doelgroep? 
% Reflecteer kritisch over het resultaat. In Engelse teksten wordt deze sectie
% ``Discussion'' genoemd. Had je deze uitkomst verwacht? Zijn er zaken die nog
% niet duidelijk zijn?
% Heeft het onderzoek geleid tot nieuwe vragen die uitnodigen tot verder 
%onderzoek?
De IMIS collecties van het VLIZ kunnen voortaan semi-automatisch uitgebreid worden door het onderzoek van deze bachelorproef.\\
Google Scholar alerts zijn een automatische en incrementele bron van gegevens die zullen blijven stromen zolang de melding bestaat.\\ 
Web scraping technieken met LLMs verdienen bijzondere aandacht. Online modellen laten toe om een SERP te parsen en bovendien zijn ze bestand tegen interne veranderingen van de SERP. Na dit onderzoek moet er verder getest worden met verschillende lokale modellen op verschillende systemen. Het succes van de online modellen doet toch verwachten dat er ook lokaal een slaagkans moet zijn. Tot dan wordt er gewerkt met Beautiful Soup. Aangezien de structuur van de SERP vast ligt is dit zeker geen slechte oplossing. De code van Beautiful Soup en de HTML structuur gaan hand in hand zodat wijzigingen van die laatste naar verwachting vrij eenvoudig op te vangen zijn in de code. Belangrijk daarbij is dat er een notificatiesysteem voorzien wordt dat afgaat wanneer er zo een wijziging zou optreden.\\
Het berekenen van een relevantiescore op basis van de frequentie van de zoekopdracht in het zoekresultaat geeft zeker aanleiding tot een nieuwe indicator, maar de mate waarin die effectief ook voorspelt of een publicatie interessant is voor IMIS is zeker vatbaar voor discussie. Verder onderzoek moet dus ook blijven inzetten op de ``Natural Language Understanding'' om de waarde van een publicatie voor IMIS beter te berekenen.\\
Het opsporen van de DOI van elke publicatie volgens een stapsgewijze procedure is efficiënt doch niet feilloos. Er blijft een minderheid van publicaties waarvoor de DOI niet bestaat of niet gevonden kan worden. Verder onderzoek moet vooral inzetten op het vinden van de juiste DOI op de webpagina van de publicatie wanneer daar meerdere DOIs aanwezig zijn. Dat is een eenvoudige taak voor een mens, maar dat is het niet voor een computer.\\
Tenslotte helpt het opzoeken van de gelijkenis tussen de titel van een publicatie en alle titels in IMIS aan de hand van ``Semantic search'' wanneer een manuele beslissing moet genomen worden om een publicatie toe te voegen aan IMIS. Bij een volledige match zal er geen twijfel zijn, maar bij een gedeeltelijke match, bijvoorbeeld in het geval van variaties in de titel, moet er beter onderzocht worden wat de threshold is om te beslissen om een publicatie aan IMIS toe te voegen.

