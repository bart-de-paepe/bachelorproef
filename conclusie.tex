%%=============================================================================
%% Conclusie
%%=============================================================================

\chapter{Conclusie}%
\label{ch:conclusie}

% TODO: Trek een duidelijke conclusie, in de vorm van een antwoord op de
% onderzoeksvra(a)g(en). Wat was jouw bijdrage aan het onderzoeksdomein en
% hoe biedt dit meerwaarde aan het vakgebied/doelgroep? 
% Reflecteer kritisch over het resultaat. In Engelse teksten wordt deze sectie
% ``Discussion'' genoemd. Had je deze uitkomst verwacht? Zijn er zaken die nog
% niet duidelijk zijn?
% Heeft het onderzoek geleid tot nieuwe vragen die uitnodigen tot verder 
%onderzoek?
De IMIS collecties van het VLIZ kunnen voortaan semi-automatisch uitgebreid worden door het onderzoek van deze bachelorproef. Elke nieuwe publicatie afkomstig van een Google Scholar zoekopdracht krijgt een score die aangeeft of de publicatie reeds in IMIS zit. Voor alle publicaties die uniek identificeerbaar zijn, krijgen deze een score van 100\% of 0\% en kunnen zo in het eerste geval automatisch aan IMIS toegevoegd worden. Voor alle publicaties die niet uniek identificeerbaar zijn zal de score tussen 0\% en 100\% liggen. In die gevallen is manuele tussenkomst nodig om te beslissen of de publicatie al dan niet aan IMIS toegevoegd moet worden.  
\lipsum[76-80]

