%===============================================================================
% LaTeX sjabloon voor de bachelorproef toegepaste informatica aan HOGENT
% Meer info op https://github.com/HoGentTIN/latex-hogent-report
%===============================================================================

\documentclass[dutch,dit,thesis]{hogentreport}

% TODO:
% - If necessary, replace the option `dit`' with your own department!
%   Valid entries are dbo, dbt, dgz, dit, dlo, dog, dsa, soa
% - If you write your thesis in English (remark: only possible after getting
%   explicit approval!), remove the option "dutch," or replace with "english".

\usepackage{lipsum} % For blind text, can be removed after adding actual content

%% Pictures to include in the text can be put in the graphics/ folder
\graphicspath{{../graphics/}}

%% For source code highlighting, requires pygments to be installed
%% Compile with the -shell-escape flag!
%% \usepackage[chapter]{minted}
%% If you compile with the make_thesis.{bat,sh} script, use the following
%% import instead:
\usepackage[chapter,outputdir=../output]{minted}
\usemintedstyle{solarized-light}

%% Formatting for minted environments.
\setminted{%
    autogobble,
    frame=lines,
    breaklines,
    linenos,
    tabsize=4
}

%% Ensure the list of listings is in the table of contents
\renewcommand\listoflistingscaption{%
    \IfLanguageName{dutch}{Lijst van codefragmenten}{List of listings}
}
\renewcommand\listingscaption{%
    \IfLanguageName{dutch}{Codefragment}{Listing}
}
\renewcommand*\listoflistings{%
    \cleardoublepage\phantomsection\addcontentsline{toc}{chapter}{\listoflistingscaption}%
    \listof{listing}{\listoflistingscaption}%
}

% Other packages not already included can be imported here

%%---------- Document metadata -------------------------------------------------
% TODO: Replace this with your own information
\author{Ernst Aarden}
\supervisor{Dhr. F. Van Houte}
\cosupervisor{Mevr. S. Beeckman}
\title[Optionele ondertitel]%
    {Titel van de bachelorproef}
\academicyear{\advance\year by -1 \the\year--\advance\year by 1 \the\year}
\examperiod{1}
\degreesought{\IfLanguageName{dutch}{Professionele bachelor in de toegepaste informatica}{Bachelor of applied computer science}}
\partialthesis{false} %% To display 'in partial fulfilment'
%\institution{Internshipcompany BVBA.}

%% Add global exceptions to the hyphenation here
\hyphenation{back-slash}

%% The bibliography (style and settings are  found in hogentthesis.cls)
\addbibresource{bachproef.bib}            %% Bibliography file
\addbibresource{../voorstel/voorstel.bib} %% Bibliography research proposal
\defbibheading{bibempty}{}

%% Prevent empty pages for right-handed chapter starts in twoside mode
\renewcommand{\cleardoublepage}{\clearpage}

\renewcommand{\arraystretch}{1.2}

%% Content starts here.
\begin{document}

%---------- Front matter -------------------------------------------------------

\frontmatter

\hypersetup{pageanchor=false} %% Disable page numbering references
%% Render a Dutch outer title page if the main language is English
\IfLanguageName{english}{%
    %% If necessary, information can be changed here
    \degreesought{Professionele Bachelor toegepaste informatica}%
    \begin{otherlanguage}{dutch}%
       \maketitle%
    \end{otherlanguage}%
}{}

%% Generates title page content
\maketitle
\hypersetup{pageanchor=true}

%%=============================================================================
%% Voorwoord
%%=============================================================================

\chapter*{\IfLanguageName{dutch}{Woord vooraf}{Preface}}%
\label{ch:voorwoord}

%% TODO:
%% Het voorwoord is het enige deel van de bachelorproef waar je vanuit je
%% eigen standpunt (``ik-vorm'') mag schrijven. Je kan hier bv. motiveren
%% waarom jij het onderwerp wil bespreken.
%% Vergeet ook niet te bedanken wie je geholpen/gesteund/... heeft

Aan de basis van deze bachelorproef ligt een JIRA ticket dat al een tijdje oud is. Dat ticket was aangemaakt door het team van IMIS met de bedoeling om een bestaande procedure te verbeteren, doch zonder daar enige prioriteit aan te koppelen. Bij mijn zoektocht naar een onderwerp kwam ik al snel bij mijn werkgever, het VLIZ, terecht. JIRA werd erbij gehaald en het ticket waarvan sprake stak met kop en schouders boven andere onderwerpen uit omwille van de toepasbaarheid, de mate van uitdaging en de haalbaarheid.\\
Wat volgde was een boeiende verdieping in de wereld van de academische literatuur, in het gigantische net van de web scraping, en niet in het minst van het onmetelijke universum van de LLMs.\\\\
Ik wil eerst en vooral mijn manager Bart Vanhoorne bedanken die een omgeving creëert waar het mogelijk is om aan een bachelorproef te werken.\\\\
Bijzondere dank ook voor mijn collega Milan Lamote voor zijn aanstekelijke positiviteit en omdat hij zonder aarzelen het co-promotorschap van deze bachelorproef aanvaardde.\\\\
Bijzondere dank ook voor mijn promotor Jan Claes voor het opzetten van een transparant kader voor deze bachelorproef en voor zijn kritische feedback.\\\\
Oprechte dank voor mijn collega Fons Verheyde voor zijn enthousiasme bij het bespreken van deze bachelorproef.\\\\
Tenslotte aan het einde van deze bachelorproef, maar vooral aan het einde van deze opleiding, eeuwige dank aan mijn echtgenote Lies Knockaert. Veel van de punten op mijn curriculum zijn onrechtstreeks ook haar verdienste.
\\\\
Bart De Paepe,\\
Sint-Baafs-Vijve, 1 mei 2025
%%=============================================================================
%% Samenvatting
%%=============================================================================

% TODO: De "abstract" of samenvatting is een kernachtige (~ 1 blz. voor een
% thesis) synthese van het document.
%
% Een goede abstract biedt een kernachtig antwoord op volgende vragen:
%
% 1. Waarover gaat de bachelorproef?
% 2. Waarom heb je er over geschreven?
% 3. Hoe heb je het onderzoek uitgevoerd?
% 4. Wat waren de resultaten? Wat blijkt uit je onderzoek?
% 5. Wat betekenen je resultaten? Wat is de relevantie voor het werkveld?
%
% Daarom bestaat een abstract uit volgende componenten:
%
% - inleiding + kaderen thema
% - probleemstelling
% - (centrale) onderzoeksvraag
% - onderzoeksdoelstelling
% - methodologie
% - resultaten (beperk tot de belangrijkste, relevant voor de onderzoeksvraag)
% - conclusies, aanbevelingen, beperkingen
%
% LET OP! Een samenvatting is GEEN voorwoord!

%%---------- Nederlandse samenvatting -----------------------------------------
%
% TODO: Als je je bachelorproef in het Engels schrijft, moet je eerst een
% Nederlandse samenvatting invoegen. Haal daarvoor onderstaande code uit
% commentaar.
% Wie zijn bachelorproef in het Nederlands schrijft, kan dit negeren, de inhoud
% wordt niet in het document ingevoegd.

\IfLanguageName{english}{%
\selectlanguage{dutch}
\chapter*{Samenvatting}
\lipsum[1-4]
\selectlanguage{english}
}{}

%%---------- Samenvatting -----------------------------------------------------
% De samenvatting in de hoofdtaal van het document

\chapter*{\IfLanguageName{dutch}{Samenvatting}{Abstract}}

Zoekresultaten afkomstig van Google Scholar moeten aan het Integrated Marine Information System (IMIS) toegevoegd worden. Dit onderzoek bekijkt hoe dit proces volledig of grotendeels geautomatiseerd kan worden. Daarbij komen verschillende aspecten kijken die allemaal stuk voor stuk afzonderlijk behandeld worden.\\
Zo worden Google Scholar alerts gebruikt om continu nieuwe zoekresultaten te ontvangen. Er wordt uitvoerig toegelicht hoe een zoekopdracht aangemaakt moet worden en hoe daarvoor een melding ingesteld kan worden.\\
Google Scholar zoekresultaten komen in de vorm van een HTML pagina. HTML aan IMIS toevoegen is niet interessant. Er wordt dieper ingegaan hoe de lijst met zoekresultaten van de HTML pagina gescraped kan worden. Daarbij worden verschillende technieken uitgeprobeerd waaronder Large Language Models (LLMs). Uiteindelijk wordt een klassieke benadering gekozen die de HTML parset aan de hand van Beautiful Soup.\\
Na het scrapen zou de gevonden informatie in IMIS opgeslagen kunnen worden. Er zijn echter nog bijkomende stappen nodig om ervoor te zorgen dat enkel kwalitatieve gegevens naar IMIS vloeien.
Behalve het feit dat Google Scholar de zoekresultaten selecteert voor de zoekopdracht en rangschikt volgens een bepaalde volgorde, zijn er verder geen criteria die aangeven hoe relevant een zoekresultaat is voor IMIS. Natural Language Processing (NLP) biedt daar oplossingen voor. Er wordt een relevantiescore berekend van de mate waarin de zoekopdracht aanwezig is in het zoekresultaat. Algemeen wordt daarvoor aangenomen dat de frequentie van de zoekopdracht in het zoekresultaat recht evenredig is met de relevantie van het resultaat voor IMIS.\\
Er mogen ook geen duplicaten in IMIS opgeslagen worden. Duplicaten kunnen gedetecteerd worden aan de hand van de Digital Object Identifier (DOI) van de publicatie. Die kan stapsgewijs opgezocht worden in de link van de publicatie, in Crossref, of op de webpagina van de publicatie. Maar er is geen garantie dat de DOI gevonden zal worden. Voor die gevallen moet er minstens een score gegeven worden of een publicatie al dan niet een duplicaat is. Dat kan gedaan worden aan de hand van ``Semantic search''. Daarvoor worden embeddings berekend voor alle titels in IMIS. Vervolgens wordt de embedding van de titel van het zoekresultaat daarmee vergeleken. De mate van gelijkenis is een score voor duplicaten van de publicatie.\\
Uiteindelijk worden al deze stappen geïntegreerd in een pijplijn voor het semi-automatisch toevoegen van publicaties aan IMIS. In het geval de relevantiescore voldoende is en indien de DOI gevonden wordt, kan de publicatie zonder tussenkomst aan IMIS toegevoegd worden. Wanneer er geen DOI gevonden wordt, is er nog steeds een manuele stap nodig om te beslissen op basis van de duplicatenscore of de publicatie toegevoegd mag worden.


%---------- Inhoud, lijst figuren, ... -----------------------------------------

\tableofcontents

% In a list of figures, the complete caption will be included. To prevent this,
% ALWAYS add a short description in the caption!
%
%  \caption[short description]{elaborate description}
%
% If you do, only the short description will be used in the list of figures

\listoffigures

% If you included tables and/or source code listings, uncomment the appropriate
% lines.
\listoftables

\listoflistings

% Als je een lijst van afkortingen of termen wil toevoegen, dan hoort die
% hier thuis. Gebruik bijvoorbeeld de ``glossaries'' package.
% https://www.overleaf.com/learn/latex/Glossaries

%---------- Kern ---------------------------------------------------------------

\mainmatter{}

% De eerste hoofdstukken van een bachelorproef zijn meestal een inleiding op
% het onderwerp, literatuurstudie en verantwoording methodologie.
% Aarzel niet om een meer beschrijvende titel aan deze hoofdstukken te geven of
% om bijvoorbeeld de inleiding en/of stand van zaken over meerdere hoofdstukken
% te verspreiden!

%%=============================================================================
%% Inleiding
%%=============================================================================

\chapter{\IfLanguageName{dutch}{Inleiding}{Introduction}}%
\label{ch:inleiding}

Het Vlaams Instituut voor de Zee (VLIZ) \autocite{Vliz2024} is een pionier in zeekennis. Dit wetenschappelijk instituut gelegen in Oostende heeft onder andere een mandaat om een complete en geactualiseerde catalogus bij te houden van alle wetenschappelijke publicaties in de mariene sector. Al meer dan 20 jaar bouwt het Integrated Marine Information System (IMIS) aan deze catalogus die intussen beschikt over meerdere collecties van marien wetenschappelijk referentiemateriaal.\\
Naast wetenschappelijke literatuur zitten er ook collecties van mariene wetenschappelijke projecten in het systeem. Het is belangrijk voor het VLIZ om te weten door hoeveel publicaties er naar een project (vb. het World Register of Marine Species (WoRMS) \autocite{Worms2024}) verwezen wordt. Deze maatstaf geeft een indicatie van het draagvlak van elk project binnen de wetenschappelijke gemeenschap en is één van de belangrijkste criteria tijdens projectevaluaties. Daarom is het cruciaal om continu nieuwe publicaties waarin verwezen wordt naar die projecten op te zoeken. Daarvoor wordt op heden Google Scholar gebruikt die bekend staat als de meest uitgebreide en geactualiseerde index. Op die manier blijft IMIS up-to-date en zijn de projectreferenties steeds geactualiseerd.

\section{\IfLanguageName{dutch}{Probleemstelling}{Problem Statement}}%
\label{sec:probleemstelling}

Momenteel verloopt dit proces binnen het VLIZ grotendeels handmatig. De zoekresultaten, volgens een bepaalde zoekfilter per project, afkomstig van Google Scholar worden manueel gefilterd en de Digital Object Identifier (DOI) van de geselecteerde artikels wordt opgezocht.
Vervolgens worden de basisgegevens van elk artikel zoals titel, auteurs, datum en uitgever opgevraagd op basis van de DOI in Crossref \autocite{Crossref2024}. Met deze informatie wordt manueel beslist om het artikel al dan niet toe te voegen aan een collectie binnen IMIS.\\
Dit is een tijdrovend proces. Daarom is er vraag naar automatisatie die de zoekresultaten verwerkt en gestructureerd opslaat. De beslissing om een artikel toe te voegen aan IMIS blijft nog altijd een manuele stap, maar naar verwachting moet dit sneller, efficiënter en accurater verlopen. Dat moet mogelijk zijn doordat de heterogene zoekresultaten omgezet worden in gestructureerde informatie. Bovendien zal een score voor elke publicatie een indicatie geven of ze reeds aan IMIS toegevoegd werd. 

\section{\IfLanguageName{dutch}{Onderzoeksvraag}{Research question}}%
\label{sec:onderzoeksvraag}

De centrale vraag die onderzocht moet worden luidt: ``Hoe kunnen de zoekresultaten van Google Scholar automatisch omgezet worden in gestructureerde data?''
Dit omvat voornamelijk 2 problemen:
\begin{itemize}
    \item Hoe kunnen de uiteenlopende zoekresultaten van Google Scholar automatisch verwerkt worden?
    \item Zijn alle zoekresultaten uniek identificeerbaar?
\end{itemize}
De uitgewerkte oplossing zal ook met volgende aspecten rekening moeten houden:
\begin{itemize}
    \item Hoe kunnen ook steeds nieuwe zoekresultaten van de zoekopdracht systematisch opgezocht worden?
    \item Hoe kan de Search Engine Results Page (SERP) omgezet worden in gestructureerde data?
    \item Hoe kan een publicatie uniek geïdentificeerd worden?
    \item Kan ook zonder unieke identifier bepaald worden wat de toegevoegde waarde is van een publicatie voor IMIS? 
\end{itemize}

\section{\IfLanguageName{dutch}{Onderzoeksdoelstelling}{Research objective}}%
\label{sec:onderzoeksdoelstelling}

Het beoogde resultaat van het onderzoek is om het toevoegen van Google Scholar zoekresultaten aan IMIS zoveel mogelijk te automatiseren:
\begin{itemize}
    \item Een proof-of-concept van de meest geschikte methode om de zoekresultaten om te zetten in gestructureerde data.
    \item Een proof-of-concept van het proces dat aan elk resultaat een score geeft of de publicatie reeds in IMIS zit.
\end{itemize}

\section{\IfLanguageName{dutch}{Opzet van deze bachelorproef}{Structure of this bachelor thesis}}%
\label{sec:opzet-bachelorproef}

% Het is gebruikelijk aan het einde van de inleiding een overzicht te
% geven van de opbouw van de rest van de tekst. Deze sectie bevat al een aanzet
% die je kan aanvullen/aanpassen in functie van je eigen tekst.

De rest van deze bachelorproef is als volgt opgebouwd:

In Hoofdstuk~\ref{ch:stand-van-zaken} wordt een overzicht gegeven van de stand van zaken binnen het onderzoeksdomein, op basis van een literatuurstudie.

In Hoofdstuk~\ref{ch:methodologie} wordt de methodologie toegelicht en worden de gebruikte onderzoekstechnieken besproken om een antwoord te kunnen formuleren op de onderzoeksvragen.

Dit wordt verder uitgewerkt in Hoofdstuk~\ref{ch:web_scraping} voor het scrapen van de zoekresultaten, Hoofdstuk~\ref{ch:linked_data} voor het opzoeken van de unieke identificatie van elke publicatie, en Hoofdstuk~\ref{ch:semantic_search} voor het bepalen van de meerwaarde met betrekking tot IMIS van elke publicatie.

% TODO: Vul hier aan voor je eigen hoofstukken, één of twee zinnen per hoofdstuk

In Hoofdstuk~\ref{ch:conclusie}, tenslotte, wordt de conclusie gegeven en een antwoord geformuleerd op de onderzoeksvragen. Daarbij wordt ook een aanzet gegeven voor toekomstig onderzoek binnen dit domein.
\chapter{\IfLanguageName{dutch}{Stand van zaken}{State of the art}}%
\label{ch:stand-van-zaken}

% Tip: Begin elk hoofdstuk met een paragraaf inleiding die beschrijft hoe
% dit hoofdstuk past binnen het geheel van de bachelorproef. Geef in het
% bijzonder aan wat de link is met het vorige en volgende hoofdstuk.

% Pas na deze inleidende paragraaf komt de eerste sectiehoofding.
Het huidige digitale tijdperk dat hoofdzakelijk gekenmerkt wordt door de toenemende belangstelling in AI, levert eveneens een gigantische hoeveelheid aan online data. Met deze grote berg van informatie die beschikbaar is op de pagina's van het world wide web, is het bijzonder relevant voor businesses om te begrijpen hoe ze deze data kunnnen ontginnen teneinde er bruikbare informatie uit te filteren \autocite{Lotfi2021}. Een belangrijk begrip dat daarbij prominent op de voorgrond treedt, is 'web scraping' of 'web crawling'. <<Web scraping is een geautomatiseerd data extractie proces van websites met gebruik van gespecialiseerde software>> \textcite{Bhatt2023}. Maar web scraping is lang niet de enige techniek die gebruikt wordt om data van een webpagina te halen \autocite{Gray2012}. Andere technieken zijn het gebruik van API's, screen readers en het ontleden van online PDF documenten. Het voordeel van web scraping is dat de website niet over een API met ruwe data hoeft te beschikken. De HTML code waarmee de website opgebouwd is, wordt gebruikt door de scraper om de eigenlijke inhoud eruit te filteren.\\
Web scrapers kunnen gebaseerd zijn op verschillende technologieën zoals 'spidering' en 'pattern matching'. Daarnaast biedt de programmeertaal waarin ze geïmplenteerd zijn ook telkens andere mogelijkheden \autocite{Bhatt2023}.
\textcite{Lotfi2021} onderscheidt web scrapers zowel op basis van hun toepassing (vb. medisch, social media, financieel, marketing, onderzoek) als op basis van hun  methodiek (vb. copy \& paste, HTML parsing, DOM parsing, HTML DOM, reguliere expressies, XPath, vetical aggregation platform, semantic annotation recognizing, computer vision web page analyzer). Hoewel de methodologie kan wijzigen, blijft het uitgangspunt van integriteit, correctheid en betrouwbaarheid van de data steeds van primordiaal belang \autocite{Lotfi2021}.
Verder duidt \textcite{Lotfi2021} ook nog op het iteratieve karakter van een web scaper. Het programma wordt gevoed met een url, en zal dan op zijn beurt nieuwe urls zoeken op de pagina zodat die ook bezocht kunnen worden.\\
\textcite{Singrodia2019} vult bovenstaande kenmerken van web scraping nog verder aan met de systematische omzetting van ongestructureerde gegevens op webpagina's naar gestructureerde data in een databank. De gegenereerde data heeft aanleiding tot filtering of statistieken. Het biedt vele voordelen aangezien de data opgeschoond is en daardoor als vrij van fouten kan beschouwd worden. Andere voordelen zijn tijdswinst en centrale opslag wat verdere verwerking ten goede komt.\\
Daarenboven vergroot web scraping de snelheid en het volume van de data die verwerkt kan worden aanzienlijk, en vermindert het ook het aantal fouten die zouden optreden door menselijke verwerking \autocite{Bhatt2023}. Uiteindelijk zet web scraping de online informatie om in business intelligence afhankelijk van het uitgangspunt van de eindgebruiker.\\
Na zo uitvoerig web scrapers te bespreken, is het niet onbelangrijk even stil te staan bij de vraag of web scrapen wel legaal is? Volgens \textcite{EPSI2015} is daar geen straightforward antwoord op, de situatie is afhankelijk van het betrokken land. Maar over het algemeen zijn de meest reguleringen wel in het voordeel van web scraping. Men moet vooral opletten met het repsecteren van het eigendomsrecht wanneer inhoud verworven wordt door scaping en vervolgens verder gebruikt wordt.\\
Tot slot van de stand van zake omtrent web scraping nog een pleidooi voor Python. Al is het perfect mogelijk om web scraping te ontwikkelen in Php, Java, en andere, toch biedt Python de meest gebruiksvriendelijke tools aan \textcite{Kumar2023}. <<Beautiful Soup is een van de eenvoudigste bibliotheken om data te scrapen van websites. Een simpele find\_all() in Beautiful Soup, is krachtig genoeg om de data uit het gehele document te doorzoeken. Daarna is de taak om de data te structureren. Dat kan in Python aan de hand van Pandas die in staat is om de data in een geordend formaat te presenteren.>>\\
Om verder gebruik te maken van deze data moet deze eerst gestuctureerd opgeslaan worden. \textcite{Mitchell2015} reikt een aantal methodes aan waarop dit gedaan kan worden, steeds afhankelijk van het beoogde resultaat. De skills om al die data te beheren en ermee te interageren is zo mogelijk nog belangrijk dan het scrapen op zich. De data waarvan sprake is niet gekenmerkt door strikte relaties, maar stelt eerder een collectie semi gestructureerde data voor. Een document store database lijkt in dit geval het meest aangewezen. \textcite{Lourenco2015} vergelijkt de hangbare NoSQL databases en stelt MongoDB voor als een consistente \footnote{alle clients zien steeds dezelfde data} document store database.\\
De website in de spotlight voor dit onderzoek is Google Scholar (GS), de grootste bron van wetenschappelijke publicaties op heden. De beta versie verscheen in 2004 en sindsdien wordt het systeem voornamelijk door academici gebruikt om een persoonlijke bibliotheek aan te leggen tezamen met statistieken omtrent citaties en h-indexen \footnote{De h-index van een wetenschappelijk onderzoeker komt overeen met de grootste h van het aantal publicaties die minstens h keer geciteerd zijn in ander werk.}. Volgens \textcite{Oh2019} is er een symbiose tussen GS and de academische wereld waarbij academici gratis hun werk kunnen aabieden op GS. GS op zijn beurt zorgt voor goede visibiliteit van dat werk door onder andere citaten, referenties en gelijkaardige onderwerpen. In tegenstelling tot de baanbrekende vernieuwing die GS introduceerde, gaat het raadplegen van deze gegevens wel gepaard met een aantal hindernissen, in het bijzonder op grote schaal.\\
Web scraping is 1 van de technieken die aangewend worden om dit probleem te omzeilen. Meerdere studies (\autocite{Pratiba2018},\autocite{Rafsanjani2022},\autocite{Amin2024},\autocite{Sulistya2024}) gebruiken bovenstaande concepten van web scrapers en gestructureerde opslag om automatisch data te ontleden van GS aan de hand van gebruiksvriendelijke interfaces. Er zijn onderling steeds wel verschillen tussen de gebruikte tools en het gekozen formaat die beide afhankelijk zijn van het beoogde resultaat. Maar het omliggend kader is steeds hetzelfde met gebruik van web scraping en gestructureerde opslag.\\
Het is evident dat HTML pagina's verwerkt kunnen worden door web scrapers, maar daarnaast zijn ongeveer 70\% van de feiten die op het internet gepresenteerd worden, verkregen uit PDF-documenten \autocite{Singrodia2019}. Dit verklaart de noodzaak om zowel HTML pagina's als PDF documenten te scrapen.\\
\textcite{Yang2017} beschrijft welke HTML elementen belangrijk zijn bij het ontleden van een GS pagina en \textcite{Rahmatulloh2020} toont hoe deze kunnen gemapt worden naar de custom code van de scraper.\\

Dit hoofdstuk bevat je literatuurstudie. De inhoud gaat verder op de inleiding, maar zal het onderwerp van de bachelorproef *diepgaand* uitspitten. De bedoeling is dat de lezer na lezing van dit hoofdstuk helemaal op de hoogte is van de huidige stand van zaken (state-of-the-art) in het onderzoeksdomein. Iemand die niet vertrouwd is met het onderwerp, weet nu voldoende om de rest van het verhaal te kunnen volgen, zonder dat die er nog andere informatie moet over opzoeken \autocite{Pollefliet2011}.

Je verwijst bij elke bewering die je doet, vakterm die je introduceert, enz.\ naar je bronnen. In \LaTeX{} kan dat met het commando \texttt{$\backslash${textcite\{\}}} of \texttt{$\backslash${autocite\{\}}}. Als argument van het commando geef je de ``sleutel'' van een ``record'' in een bibliografische databank in het Bib\LaTeX{}-formaat (een tekstbestand). Als je expliciet naar de auteur verwijst in de zin (narratieve referentie), gebruik je \texttt{$\backslash${}textcite\{\}}. Soms is de auteursnaam niet expliciet een onderdeel van de zin, dan gebruik je \texttt{$\backslash${}autocite\{\}} (referentie tussen haakjes). Dit gebruik je bv.~bij een citaat, of om in het bijschrift van een overgenomen afbeelding, broncode, tabel, enz. te verwijzen naar de bron. In de volgende paragraaf een voorbeeld van elk.

\textcite{Knuth1998} schreef een van de standaardwerken over sorteer- en zoekalgoritmen. Experten zijn het erover eens dat cloud computing een interessante opportuniteit vormen, zowel voor gebruikers als voor dienstverleners op vlak van informatietechnologie~\autocite{Creeger2009}.

Let er ook op: het \texttt{cite}-commando voor de punt, dus binnen de zin. Je verwijst meteen naar een bron in de eerste zin die erop gebaseerd is, dus niet pas op het einde van een paragraaf.

\begin{figure}
  \centering
  \includegraphics[width=0.8\textwidth]{grail.jpg}
  \caption[Voorbeeld figuur.]{\label{fig:grail}Voorbeeld van invoegen van een figuur. Zorg altijd voor een uitgebreid bijschrift dat de figuur volledig beschrijft zonder in de tekst te moeten gaan zoeken. Vergeet ook je bronvermelding niet!}
\end{figure}

\begin{listing}
  \begin{minted}{python}
    import pandas as pd
    import seaborn as sns

    penguins = sns.load_dataset('penguins')
    sns.relplot(data=penguins, x="flipper_length_mm", y="bill_length_mm", hue="species")
  \end{minted}
  \caption[Voorbeeld codefragment]{Voorbeeld van het invoegen van een codefragment.}
\end{listing}

\begin{table}
  \centering
  \begin{tabular}{lcr}
    \toprule
    \textbf{Kolom 1} & \textbf{Kolom 2} & \textbf{Kolom 3} \\
    $\alpha$         & $\beta$          & $\gamma$         \\
    \midrule
    A                & 10.230           & a                \\
    B                & 45.678           & b                \\
    C                & 99.987           & c                \\
    \bottomrule
  \end{tabular}
  \caption[Voorbeeld tabel]{\label{tab:example}Voorbeeld van een tabel.}
\end{table}


%%=============================================================================
%% Methodologie
%%=============================================================================

\chapter{\IfLanguageName{dutch}{Methodologie}{Methodology}}%
\label{ch:methodologie}

%% TODO: In dit hoofstuk geef je een korte toelichting over hoe je te werk bent
%% gegaan. Verdeel je onderzoek in grote fasen, en licht in elke fase toe wat
%% de doelstelling was, welke deliverables daar uit gekomen zijn, en welke
%% onderzoeksmethoden je daarbij toegepast hebt. Verantwoord waarom je
%% op deze manier te werk gegaan bent.
%% 
%% Voorbeelden van zulke fasen zijn: literatuurstudie, opstellen van een
%% requirements-analyse, opstellen long-list (bij vergelijkende studie),
%% selectie van geschikte tools (bij vergelijkende studie, "short-list"),
%% opzetten testopstelling/PoC, uitvoeren testen en verzamelen
%% van resultaten, analyse van resultaten, ...
%%
%% !!!!! LET OP !!!!!
%%
%% Het is uitdrukkelijk NIET de bedoeling dat je het grootste deel van de corpus
%% van je bachelorproef in dit hoofstuk verwerkt! Dit hoofdstuk is eerder een
%% kort overzicht van je plan van aanpak.
%%
%% Maak voor elke fase (behalve het literatuuronderzoek) een NIEUW HOOFDSTUK aan
%% en geef het een gepaste titel.
Om nieuwe publicaties die gerelateerd zijn aan wetenschappelijke projecten toe te voegen aan IMIS, moet er natuurlijk een signaal zijn wanneer dergelijke publicaties beschikbaar zijn. Indexen van academische literatuur zijn daarvoor een geschikte bron en zoals eerder beschreven is Google Scholar (GS) een interessant alternatief. Het is de bedoeling om enkel nieuwe resultaten te ontvangen en bijgevolg vervalt dus de optie om publicaties op te zoeken aan de hand van de GS zoekpagina. De zoekpagina houdt geen gegevens bij van vorige queries en daarom is het zeer aannemelijk dat dezelfde resultaten zullen voorkomen bij opeenvolgende zoekopdrachten. GS laat echter toe om meldingen aan te maken voor een bepaalde zoekopdracht. Die stuurt automatisch nieuwe resultaten door per e-mail onder de vorm van de GS SERP \footnote{Search Engine Result Page}. Het opstellen van een zoekopdracht en het aanmaken van een alert worden verder uitgewerkt in hoofdstuk~\ref{ch:googlescholaralert}.\\
GS meldingen zijn e-mails in HTML formaat. Ze bevatten inhoud die overeenkomt met de GS SERP overeenkomstig de zoekopdracht. De SERP bevat een vaste structuur. Het is een lijst met zoekresultaten die telkens dezelfde elementen bevatten:
\begin{itemize}
    \item titel
    \item link naar de webpagina van de publicatie
    \item auteurs
    \item tijdschrift
    \item jaartal
    \item abstract van de publicatie of een fragment ervan
\end{itemize}
Die HTML moet omgezet worden in gestructureerde data door middel van HTML scraping technieken zoals te zien zijn in tabel \ref{table:HTML scraping technieken}.
\begin{table}[h!]
    \begin{tabularx}{\textwidth}{|>{\hsize=1.0\hsize\linewidth=\hsize}X
            |>{\hsize=1.0\hsize\linewidth=\hsize}X|}
        \hline
        HTML scraping door een LLM &
        \begin{itemize}
            \item online model
            \begin{itemize}
                \item OpenAI
                \item Generieke procedure onafhankelijk van het model
            \end{itemize}
            \item lokaal model
        \end{itemize}
         \\
        \hline
         HTML scraping door het parsen van de DOM &
        \begin{itemize}
            \item Beautiful Soup
            \item SerpAPI
        \end{itemize}\\
        \hline
    \end{tabularx}
    \caption{HTML scraping technieken.}
    \label{table:HTML scraping technieken}
\end{table}

Web scraping wordt verder uitgewerkt in hoofdstuk~\ref{ch:web_scraping}.\\
Elk zoekresultaat is op basis van het algoritme van GS gematched met de zoekopdracht. Maar wil dat daarom ook zeggen dat de publicatie interessant is voor IMIS? Aan de hand van Natural Language Processing (NLP) wordt een score berekend van de relevantie van het zoekresultaat voor de overeenkomstige collectie in IMIS. NLP wordt verder uitgewerkt in Hoofdstuk~\ref{ch:natural_language_processing}.\\
Om te weten of een publicatie echt nieuw is, moet ze ondubbelzinnig geïdentificeerd kunnen worden. Titel, auteurs, tijdschrift, enz. maken een publicatie echter niet uniek. Er kunnen namelijk variante benamingen voorkomen van titels, auteurs, enz. Voor literatuur is het de DOI die een publicatie uniek maakt. Voor elk zoekresultaat wordt er op zoek gegaan naar de DOI aan de hand van een stapsgewijze procedure:
\begin{enumerate}
    \item DOI opzoeken in de link naar de webpagina van de publicatie
    \item DOI opzoeken in Crossref op basis van de titel
    \item DOI opzoeken op de webpagina van de publicatie
    \begin{itemize}
        \item dit kan een HTML pagina zijn
        \item dit kan een PDF document zijn
        \item dit kan een HTML pagina zijn met een embedded PDF document
    \end{itemize}
\end{enumerate}
Het opzoeken van de DOI wordt verder uitgewerkt in hoofdstuk~\ref{ch:linked_data}.\\
Dan resteert de vraag of de publicatie echt nieuw is, of dat ze reeds aan IMIS toegevoegd werd? Het antwoord daarop is afhankelijk van de beschikbare informatie die tijdens de voorgaande stappen gevonden werd.
\begin{itemize}
    \item De DOI is gevonden: er kan met 100\% zekerheid opgezocht worden of de publicatie reeds in IMIS zit of niet.
    \item De DOI is niet gevonden: omwille van de variante benamingen kan niet met volledige zekerheid opgezocht worden of een publicatie reeds in IMIS zit of niet. Wel kan door middel van semantic search op basis van de titel de waarschijnlijkheid berekend worden dat de publicatie reeds in IMIS zit.
\end{itemize}
Semantic search wordt verder uitgewerkt in hoofdstuk~\ref{ch:semantic_search}.\\
Tenslotte worden alle methodes toegepast op de volledige dataset van alle GS alerts met betrekking tot het VLIZ voor de periode van februari tot en met mei 2025. Deze resultaten worden besproken in hoofdstuk~\ref{ch:resultaten}.






% Voeg hier je eigen hoofdstukken toe die de ``corpus'' van je bachelorproef
% vormen. De structuur en titels hangen af van je eigen onderzoek. Je kan bv.
% elke fase in je onderzoek in een apart hoofdstuk bespreken.

%\input{...}
%\input{...}
%...

%%=============================================================================
%% Conclusie
%%=============================================================================

\chapter{Conclusie}%
\label{ch:conclusie}

% TODO: Trek een duidelijke conclusie, in de vorm van een antwoord op de
% onderzoeksvra(a)g(en). Wat was jouw bijdrage aan het onderzoeksdomein en
% hoe biedt dit meerwaarde aan het vakgebied/doelgroep? 
% Reflecteer kritisch over het resultaat. In Engelse teksten wordt deze sectie
% ``Discussion'' genoemd. Had je deze uitkomst verwacht? Zijn er zaken die nog
% niet duidelijk zijn?
% Heeft het onderzoek geleid tot nieuwe vragen die uitnodigen tot verder 
%onderzoek?
De IMIS collecties van het VLIZ kunnen voortaan semi-automatisch uitgebreid worden door het onderzoek van deze bachelorproef.\\
Google Scholar alerts zijn een automatische en incrementele bron van gegevens die zullen blijven stromen zolang de melding bestaat.\\ 
Web scraping technieken met LLMs verdienen bijzondere aandacht. Online modellen laten toe om een SERP te parsen en bovendien zijn ze bestand tegen interne veranderingen van de SERP. Na dit onderzoek moet er verder getest worden met verschillende lokale modellen op verschillende systemen. Het succes van de online modellen doet toch verwachten dat er ook lokaal een slaagkans moet zijn. Tot dan wordt er gewerkt met Beautiful Soup. Aangezien de structuur van de SERP vast ligt, is dit zeker geen slechte oplossing. De code van Beautiful Soup en de HTML structuur gaan hand in hand zodat wijzigingen van die laatste naar verwachting vrij eenvoudig op te vangen zijn in de code. Belangrijk daarbij is dat er een notificatiesysteem voorzien wordt dat afgaat wanneer er zo een wijziging zou optreden.\\
Het berekenen van een relevantiescore op basis van de frequentie van de zoekopdracht in het zoekresultaat geeft zeker aanleiding tot een nieuwe indicator, maar de mate waarin die effectief ook voorspelt of een publicatie interessant is voor IMIS is zeker vatbaar voor discussie. Verder onderzoek moet dus ook blijven inzetten op de ``Natural Language Understanding'' om de waarde van een publicatie voor IMIS beter te berekenen.\\
Het opsporen van de DOI van elke publicatie volgens een stapsgewijze procedure is efficiënt doch niet feilloos. Er blijft een minderheid van publicaties waarvoor de DOI niet bestaat of niet gevonden kan worden. Verder onderzoek moet vooral inzetten op het vinden van de juiste DOI op de webpagina van de publicatie wanneer daar meerdere DOIs aanwezig zijn. Dat is een eenvoudige taak voor een mens, maar dat is het niet voor een computer.\\
Tenslotte helpt het opzoeken van de gelijkenis tussen de titel van een publicatie en alle titels in IMIS aan de hand van ``Semantic search'' wanneer een manuele beslissing moet genomen worden om een publicatie toe te voegen aan IMIS. Bij een volledige match zal er geen twijfel zijn, maar bij een gedeeltelijke match, bijvoorbeeld in het geval van variaties in de titel, moet er beter onderzocht worden wat de threshold is om te beslissen om een publicatie aan IMIS toe te voegen.



%---------- Bijlagen -----------------------------------------------------------

\appendix

\chapter{Onderzoeksvoorstel}

Het onderwerp van deze bachelorproef is gebaseerd op een onderzoeksvoorstel dat vooraf werd beoordeeld door de promotor. Dat voorstel is opgenomen in deze bijlage.

%% TODO: 
%\section*{Samenvatting}

% Kopieer en plak hier de samenvatting (abstract) van je onderzoeksvoorstel.

% Verwijzing naar het bestand met de inhoud van het onderzoeksvoorstel
\input{../voorstel/voorstel-inhoud}

%%---------- Andere bijlagen --------------------------------------------------
% TODO: Voeg hier eventuele andere bijlagen toe. Bv. als je deze BP voor de
% tweede keer indient, een overzicht van de verbeteringen t.o.v. het origineel.
%\input{...}

%%---------- Backmatter, referentielijst ---------------------------------------

\backmatter{}

\setlength\bibitemsep{2pt} %% Add Some space between the bibliograpy entries
\printbibliography[heading=bibintoc]

\end{document}
