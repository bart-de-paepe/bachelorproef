%===============================================================================
% LaTeX sjabloon voor de bachelorproef toegepaste informatica aan HOGENT
% Meer info op https://github.com/HoGentTIN/latex-hogent-report
%===============================================================================

\documentclass[dutch,dit,thesis]{hogentreport}

% TODO:
% - If necessary, replace the option `dit`' with your own department!
%   Valid entries are dbo, dbt, dgz, dit, dlo, dog, dsa, soa
% - If you write your thesis in English (remark: only possible after getting
%   explicit approval!), remove the option "dutch," or replace with "english".

\usepackage{lipsum} % For blind text, can be removed after adding actual content
\usepackage{array} % Tables with a fixed width
\usepackage[table]{xcolor}
\usepackage{float}
\usepackage{placeins}
\usepackage{tabularx}
\usepackage{geometry}

%% Pictures to include in the text can be put in the graphics/ folder
\graphicspath{{../graphics/}}

%% For source code highlighting, requires pygments to be installed
%% Compile with the -shell-escape flag!
%% \usepackage[chapter]{minted}
%% If you compile with the make_thesis.{bat,sh} script, use the following
%% import instead:
\usepackage[chapter,outputdir=../output]{minted}
\usemintedstyle{solarized-light}

%% Formatting for minted environments.
\setminted{%
    autogobble,
    frame=lines,
    breaklines,
    linenos,
    tabsize=4
}

\usepackage{listings}

%% Ensure the list of listings is in the table of contents
\renewcommand\listoflistingscaption{%
    \IfLanguageName{dutch}{Lijst van codefragmenten}{List of listings}
}
\renewcommand\listingscaption{%
    \IfLanguageName{dutch}{Codefragment}{Listing}
}
\renewcommand*\listoflistings{%
    \cleardoublepage\phantomsection\addcontentsline{toc}{chapter}{\listoflistingscaption}%
    \listof{listing}{\listoflistingscaption}%
}

% Other packages not already included can be imported here

%%---------- Document metadata -------------------------------------------------
% TODO: Replace this with your own information
\author{Bart De Paepe}
\supervisor{Dhr. Jan Claes}
\cosupervisor{Dhr. Milan Lamote}
\title[]%
    {Google Scholar zoekresultaten voor wetenschappelijke projecten: linked data \& natural language processing}
\academicyear{\advance\year by -1 \the\year--\advance\year by 1 \the\year}
\examperiod{1}
\degreesought{\IfLanguageName{dutch}{Professionele bachelor in de toegepaste informatica}{Bachelor of applied computer science}}
\partialthesis{false} %% To display 'in partial fulfilment'
%\institution{Internshipcompany BVBA.}

%% Add global exceptions to the hyphenation here
\hyphenation{back-slash}

%% The bibliography (style and settings are  found in hogentthesis.cls)
\addbibresource{bachproef.bib}            %% Bibliography file
\addbibresource{../voorstel/voorstel.bib} %% Bibliography research proposal
\defbibheading{bibempty}{}

%% Prevent empty pages for right-handed chapter starts in twoside mode
\renewcommand{\cleardoublepage}{\clearpage}

\renewcommand{\arraystretch}{1.2}

%% Content starts here.
\begin{document}

%---------- Front matter -------------------------------------------------------

\frontmatter

\hypersetup{pageanchor=false} %% Disable page numbering references
%% Render a Dutch outer title page if the main language is English
\IfLanguageName{english}{%
    %% If necessary, information can be changed here
    \degreesought{Professionele Bachelor toegepaste informatica}%
    \begin{otherlanguage}{dutch}%
       \maketitle%
    \end{otherlanguage}%
}{}

%% Generates title page content
\maketitle
\hypersetup{pageanchor=true}

%%=============================================================================
%% Voorwoord
%%=============================================================================

\chapter*{\IfLanguageName{dutch}{Woord vooraf}{Preface}}%
\label{ch:voorwoord}

%% TODO:
%% Het voorwoord is het enige deel van de bachelorproef waar je vanuit je
%% eigen standpunt (``ik-vorm'') mag schrijven. Je kan hier bv. motiveren
%% waarom jij het onderwerp wil bespreken.
%% Vergeet ook niet te bedanken wie je geholpen/gesteund/... heeft

Aan de basis van deze bachelorproef ligt een JIRA ticket dat al een tijdje oud is. Dat ticket was aangemaakt door het team van IMIS met de bedoeling om een bestaande procedure te verbeteren, doch zonder daar enige prioriteit aan te koppelen. Bij mijn zoektocht naar een onderwerp kwam ik al snel bij mijn werkgever, het VLIZ, terecht. JIRA werd erbij gehaald en het ticket waarvan sprake stak met kop en schouders boven andere onderwerpen uit omwille van de toepasbaarheid, de mate van uitdaging en de haalbaarheid.\\
Wat volgde was een boeiende verdieping in de wereld van de academische literatuur, in het gigantische net van de web scraping, en niet in het minst van het onmetelijke universum van de LLMs.\\\\
Ik wil eerst en vooral mijn manager Bart Vanhoorne bedanken die een omgeving creëert waar het mogelijk is om aan een bachelorproef te werken.\\\\
Bijzondere dank ook voor mijn collega Milan Lamote voor zijn aanstekelijke positiviteit en omdat hij zonder aarzelen het co-promotorschap van deze bachelorproef aanvaardde.\\\\
Bijzondere dank ook voor mijn promotor Jan Claes voor het opzetten van een transparant kader voor deze bachelorproef en voor zijn kritische feedback.\\\\
Oprechte dank voor mijn collega Fons Verheyde voor zijn enthousiasme bij het bespreken van deze bachelorproef.\\\\
Tenslotte aan het einde van deze bachelorproef, maar vooral aan het einde van deze opleiding, eeuwige dank aan mijn echtgenote Lies Knockaert. Veel van de punten op mijn curriculum zijn onrechtstreeks ook haar verdienste.
\\\\
Bart De Paepe,\\
Sint-Baafs-Vijve, 1 mei 2025
%%=============================================================================
%% Samenvatting
%%=============================================================================

% TODO: De "abstract" of samenvatting is een kernachtige (~ 1 blz. voor een
% thesis) synthese van het document.
%
% Een goede abstract biedt een kernachtig antwoord op volgende vragen:
%
% 1. Waarover gaat de bachelorproef?
% 2. Waarom heb je er over geschreven?
% 3. Hoe heb je het onderzoek uitgevoerd?
% 4. Wat waren de resultaten? Wat blijkt uit je onderzoek?
% 5. Wat betekenen je resultaten? Wat is de relevantie voor het werkveld?
%
% Daarom bestaat een abstract uit volgende componenten:
%
% - inleiding + kaderen thema
% - probleemstelling
% - (centrale) onderzoeksvraag
% - onderzoeksdoelstelling
% - methodologie
% - resultaten (beperk tot de belangrijkste, relevant voor de onderzoeksvraag)
% - conclusies, aanbevelingen, beperkingen
%
% LET OP! Een samenvatting is GEEN voorwoord!

%%---------- Nederlandse samenvatting -----------------------------------------
%
% TODO: Als je je bachelorproef in het Engels schrijft, moet je eerst een
% Nederlandse samenvatting invoegen. Haal daarvoor onderstaande code uit
% commentaar.
% Wie zijn bachelorproef in het Nederlands schrijft, kan dit negeren, de inhoud
% wordt niet in het document ingevoegd.

\IfLanguageName{english}{%
\selectlanguage{dutch}
\chapter*{Samenvatting}
\lipsum[1-4]
\selectlanguage{english}
}{}

%%---------- Samenvatting -----------------------------------------------------
% De samenvatting in de hoofdtaal van het document

\chapter*{\IfLanguageName{dutch}{Samenvatting}{Abstract}}

Zoekresultaten afkomstig van Google Scholar moeten aan het Integrated Marine Information System (IMIS) toegevoegd worden. Dit onderzoek bekijkt hoe dit proces volledig of grotendeels geautomatiseerd kan worden. Daarbij komen verschillende aspecten kijken die allemaal stuk voor stuk afzonderlijk behandeld worden.\\
Zo worden Google Scholar alerts gebruikt om continu nieuwe zoekresultaten te ontvangen. Er wordt uitvoerig toegelicht hoe een zoekopdracht aangemaakt moet worden en hoe daarvoor een melding ingesteld kan worden.\\
Google Scholar zoekresultaten komen in de vorm van een HTML pagina. HTML aan IMIS toevoegen is niet interessant. Er wordt dieper ingegaan hoe de lijst met zoekresultaten van de HTML pagina gescraped kan worden. Daarbij worden verschillende technieken uitgeprobeerd waaronder Large Language Models (LLMs). Uiteindelijk wordt een klassieke benadering gekozen die de HTML parset aan de hand van Beautiful Soup.\\
Na het scrapen zou de gevonden informatie in IMIS opgeslagen kunnen worden. Er zijn echter nog bijkomende stappen nodig om ervoor te zorgen dat enkel kwalitatieve gegevens naar IMIS vloeien.
Behalve het feit dat Google Scholar de zoekresultaten selecteert voor de zoekopdracht en rangschikt volgens een bepaalde volgorde, zijn er verder geen criteria die aangeven hoe relevant een zoekresultaat is voor IMIS. Natural Language Processing (NLP) biedt daar oplossingen voor. Er wordt een relevantiescore berekend van de mate waarin de zoekopdracht aanwezig is in het zoekresultaat. Algemeen wordt daarvoor aangenomen dat de frequentie van de zoekopdracht in het zoekresultaat recht evenredig is met de relevantie van het resultaat voor IMIS.\\
Er mogen ook geen duplicaten in IMIS opgeslagen worden. Duplicaten kunnen gedetecteerd worden aan de hand van de Digital Object Identifier (DOI) van de publicatie. Die kan stapsgewijs opgezocht worden in de link van de publicatie, in Crossref, of op de webpagina van de publicatie. Maar er is geen garantie dat de DOI gevonden zal worden. Voor die gevallen moet er minstens een score gegeven worden of een publicatie al dan niet een duplicaat is. Dat kan gedaan worden aan de hand van ``Semantic search''. Daarvoor worden embeddings berekend voor alle titels in IMIS. Vervolgens wordt de embedding van de titel van het zoekresultaat daarmee vergeleken. De mate van gelijkenis is een score voor duplicaten van de publicatie.\\
Uiteindelijk worden al deze stappen geïntegreerd in een pijplijn voor het semi-automatisch toevoegen van publicaties aan IMIS. In het geval de relevantiescore voldoende is en indien de DOI gevonden wordt, kan de publicatie zonder tussenkomst aan IMIS toegevoegd worden. Wanneer er geen DOI gevonden wordt, is er nog steeds een manuele stap nodig om te beslissen op basis van de duplicatenscore of de publicatie toegevoegd mag worden.


%---------- Inhoud, lijst figuren, ... -----------------------------------------

\tableofcontents

% In a list of figures, the complete caption will be included. To prevent this,
% ALWAYS add a short description in the caption!
%
%  \caption[short description]{elaborate description}
%
% If you do, only the short description will be used in the list of figures

\listoffigures

% If you included tables and/or source code listings, uncomment the appropriate
% lines.
\listoftables

\listoflistings

% Als je een lijst van afkortingen of termen wil toevoegen, dan hoort die
% hier thuis. Gebruik bijvoorbeeld de ``glossaries'' package.
% https://www.overleaf.com/learn/latex/Glossaries

%---------- Kern ---------------------------------------------------------------

\mainmatter{}

% De eerste hoofdstukken van een bachelorproef zijn meestal een inleiding op
% het onderwerp, literatuurstudie en verantwoording methodologie.
% Aarzel niet om een meer beschrijvende titel aan deze hoofdstukken te geven of
% om bijvoorbeeld de inleiding en/of stand van zaken over meerdere hoofdstukken
% te verspreiden!

%%=============================================================================
%% Inleiding
%%=============================================================================

\chapter{\IfLanguageName{dutch}{Inleiding}{Introduction}}%
\label{ch:inleiding}

Het Vlaams Instituut voor de Zee (VLIZ) \autocite{Vliz2024} is een pionier in zeekennis. Dit wetenschappelijk instituut gelegen in Oostende heeft onder andere een mandaat om een complete en geactualiseerde catalogus bij te houden van alle wetenschappelijke publicaties in de mariene sector. Al meer dan 20 jaar bouwt het Integrated Marine Information System (IMIS) aan deze catalogus die intussen beschikt over meerdere collecties van marien wetenschappelijk referentiemateriaal.\\
Naast wetenschappelijke literatuur zitten er ook collecties van mariene wetenschappelijke projecten in het systeem. Het is belangrijk voor het VLIZ om te weten door hoeveel publicaties er naar een project (vb. het World Register of Marine Species (WoRMS) \autocite{Worms2024}) verwezen wordt. Deze maatstaf geeft een indicatie van het draagvlak van elk project binnen de wetenschappelijke gemeenschap en is één van de belangrijkste criteria tijdens projectevaluaties. Daarom is het cruciaal om continu nieuwe publicaties waarin verwezen wordt naar die projecten op te zoeken. Daarvoor wordt op heden Google Scholar gebruikt die bekend staat als de meest uitgebreide en geactualiseerde index. Op die manier blijft IMIS up-to-date en zijn de projectreferenties steeds geactualiseerd.

\section{\IfLanguageName{dutch}{Probleemstelling}{Problem Statement}}%
\label{sec:probleemstelling}

Momenteel verloopt dit proces binnen het VLIZ grotendeels handmatig. De zoekresultaten, volgens een bepaalde zoekfilter per project, afkomstig van Google Scholar worden manueel gefilterd en de Digital Object Identifier (DOI) van de geselecteerde artikels wordt opgezocht.
Vervolgens worden de basisgegevens van elk artikel zoals titel, auteurs, datum en uitgever opgevraagd op basis van de DOI in Crossref \autocite{Crossref2024}. Met deze informatie wordt manueel beslist om het artikel al dan niet toe te voegen aan een collectie binnen IMIS.\\
Dit is een tijdrovend proces. Daarom is er vraag naar automatisatie die de zoekresultaten verwerkt en gestructureerd opslaat. De beslissing om een artikel toe te voegen aan IMIS blijft nog altijd een manuele stap, maar naar verwachting moet dit sneller, efficiënter en accurater verlopen. Dat moet mogelijk zijn doordat de heterogene zoekresultaten omgezet worden in gestructureerde informatie. Bovendien zal een score voor elke publicatie een indicatie geven of ze reeds aan IMIS toegevoegd werd. 

\section{\IfLanguageName{dutch}{Onderzoeksvraag}{Research question}}%
\label{sec:onderzoeksvraag}

De centrale vraag die onderzocht moet worden luidt: ``Hoe kunnen de zoekresultaten van Google Scholar automatisch omgezet worden in gestructureerde data?''
Dit omvat voornamelijk 2 problemen:
\begin{itemize}
    \item Hoe kunnen de uiteenlopende zoekresultaten van Google Scholar automatisch verwerkt worden?
    \item Zijn alle zoekresultaten uniek identificeerbaar?
\end{itemize}
De uitgewerkte oplossing zal ook met volgende aspecten rekening moeten houden:
\begin{itemize}
    \item Hoe kunnen ook steeds nieuwe zoekresultaten van de zoekopdracht systematisch opgezocht worden?
    \item Hoe kan de Search Engine Results Page (SERP) omgezet worden in gestructureerde data?
    \item Hoe kan een publicatie uniek geïdentificeerd worden?
    \item Kan ook zonder unieke identifier bepaald worden wat de toegevoegde waarde is van een publicatie voor IMIS? 
\end{itemize}

\section{\IfLanguageName{dutch}{Onderzoeksdoelstelling}{Research objective}}%
\label{sec:onderzoeksdoelstelling}

Het beoogde resultaat van het onderzoek is om het toevoegen van Google Scholar zoekresultaten aan IMIS zoveel mogelijk te automatiseren:
\begin{itemize}
    \item Een proof-of-concept van de meest geschikte methode om de zoekresultaten om te zetten in gestructureerde data.
    \item Een proof-of-concept van het proces dat aan elk resultaat een score geeft of de publicatie reeds in IMIS zit.
\end{itemize}

\section{\IfLanguageName{dutch}{Opzet van deze bachelorproef}{Structure of this bachelor thesis}}%
\label{sec:opzet-bachelorproef}

% Het is gebruikelijk aan het einde van de inleiding een overzicht te
% geven van de opbouw van de rest van de tekst. Deze sectie bevat al een aanzet
% die je kan aanvullen/aanpassen in functie van je eigen tekst.

De rest van deze bachelorproef is als volgt opgebouwd:

In Hoofdstuk~\ref{ch:stand-van-zaken} wordt een overzicht gegeven van de stand van zaken binnen het onderzoeksdomein, op basis van een literatuurstudie.

In Hoofdstuk~\ref{ch:methodologie} wordt de methodologie toegelicht en worden de gebruikte onderzoekstechnieken besproken om een antwoord te kunnen formuleren op de onderzoeksvragen.

Dit wordt verder uitgewerkt in Hoofdstuk~\ref{ch:web_scraping} voor het scrapen van de zoekresultaten, Hoofdstuk~\ref{ch:linked_data} voor het opzoeken van de unieke identificatie van elke publicatie, en Hoofdstuk~\ref{ch:semantic_search} voor het bepalen van de meerwaarde met betrekking tot IMIS van elke publicatie.

% TODO: Vul hier aan voor je eigen hoofstukken, één of twee zinnen per hoofdstuk

In Hoofdstuk~\ref{ch:conclusie}, tenslotte, wordt de conclusie gegeven en een antwoord geformuleerd op de onderzoeksvragen. Daarbij wordt ook een aanzet gegeven voor toekomstig onderzoek binnen dit domein.
\chapter{\IfLanguageName{dutch}{Stand van zaken}{State of the art}}%
\label{ch:stand-van-zaken}

% Tip: Begin elk hoofdstuk met een paragraaf inleiding die beschrijft hoe
% dit hoofdstuk past binnen het geheel van de bachelorproef. Geef in het
% bijzonder aan wat de link is met het vorige en volgende hoofdstuk.

% Pas na deze inleidende paragraaf komt de eerste sectiehoofding.
Het huidige digitale tijdperk dat hoofdzakelijk gekenmerkt wordt door de toenemende belangstelling in AI, levert eveneens een gigantische hoeveelheid aan online data. Met deze grote berg van informatie die beschikbaar is op de pagina's van het world wide web, is het bijzonder relevant voor businesses om te begrijpen hoe ze deze data kunnnen ontginnen teneinde er bruikbare informatie uit te filteren \autocite{Lotfi2021}. Een belangrijk begrip dat daarbij prominent op de voorgrond treedt, is 'web scraping' of 'web crawling'. <<Web scraping is een geautomatiseerd data extractie proces van websites met gebruik van gespecialiseerde software>> \textcite{Bhatt2023}. Maar web scraping is lang niet de enige techniek die gebruikt wordt om data van een webpagina te halen \autocite{Gray2012}. Andere technieken zijn het gebruik van API's, screen readers en het ontleden van online PDF documenten. Het voordeel van web scraping is dat de website niet over een API met ruwe data hoeft te beschikken. De HTML code waarmee de website opgebouwd is, wordt gebruikt door de scraper om de eigenlijke inhoud eruit te filteren.\\
Web scrapers kunnen gebaseerd zijn op verschillende technologieën zoals 'spidering' en 'pattern matching'. Daarnaast biedt de programmeertaal waarin ze geïmplenteerd zijn ook telkens andere mogelijkheden \autocite{Bhatt2023}.
\textcite{Lotfi2021} onderscheidt web scrapers zowel op basis van hun toepassing (vb. medisch, social media, financieel, marketing, onderzoek) als op basis van hun  methodiek (vb. copy \& paste, HTML parsing, DOM parsing, HTML DOM, reguliere expressies, XPath, vetical aggregation platform, semantic annotation recognizing, computer vision web page analyzer). Hoewel de methodologie kan wijzigen, blijft het uitgangspunt van integriteit, correctheid en betrouwbaarheid van de data steeds van primordiaal belang \autocite{Lotfi2021}.
Verder duidt \textcite{Lotfi2021} ook nog op het iteratieve karakter van een web scaper. Het programma wordt gevoed met een url, en zal dan op zijn beurt nieuwe urls zoeken op de pagina zodat die ook bezocht kunnen worden.\\
\textcite{Singrodia2019} vult bovenstaande kenmerken van web scraping nog verder aan met de systematische omzetting van ongestructureerde gegevens op webpagina's naar gestructureerde data in een databank. De gegenereerde data heeft aanleiding tot filtering of statistieken. Het biedt vele voordelen aangezien de data opgeschoond is en daardoor als vrij van fouten kan beschouwd worden. Andere voordelen zijn tijdswinst en centrale opslag wat verdere verwerking ten goede komt.\\
Daarenboven vergroot web scraping de snelheid en het volume van de data die verwerkt kan worden aanzienlijk, en vermindert het ook het aantal fouten die zouden optreden door menselijke verwerking \autocite{Bhatt2023}. Uiteindelijk zet web scraping de online informatie om in business intelligence afhankelijk van het uitgangspunt van de eindgebruiker.\\
Na zo uitvoerig web scrapers te bespreken, is het niet onbelangrijk even stil te staan bij de vraag of web scrapen wel legaal is? Volgens \textcite{EPSI2015} is daar geen straightforward antwoord op, de situatie is afhankelijk van het betrokken land. Maar over het algemeen zijn de meest reguleringen wel in het voordeel van web scraping. Men moet vooral opletten met het repsecteren van het eigendomsrecht wanneer inhoud verworven wordt door scaping en vervolgens verder gebruikt wordt.\\
Tot slot van de stand van zake omtrent web scraping nog een pleidooi voor Python. Al is het perfect mogelijk om web scraping te ontwikkelen in Php, Java, en andere, toch biedt Python de meest gebruiksvriendelijke tools aan \textcite{Kumar2023}. <<Beautiful Soup is een van de eenvoudigste bibliotheken om data te scrapen van websites. Een simpele find\_all() in Beautiful Soup, is krachtig genoeg om de data uit het gehele document te doorzoeken. Daarna is de taak om de data te structureren. Dat kan in Python aan de hand van Pandas die in staat is om de data in een geordend formaat te presenteren.>>\\
Om verder gebruik te maken van deze data moet deze eerst gestuctureerd opgeslaan worden. \textcite{Mitchell2015} reikt een aantal methodes aan waarop dit gedaan kan worden, steeds afhankelijk van het beoogde resultaat. De skills om al die data te beheren en ermee te interageren is zo mogelijk nog belangrijk dan het scrapen op zich. De data waarvan sprake is niet gekenmerkt door strikte relaties, maar stelt eerder een collectie semi gestructureerde data voor. Een document store database lijkt in dit geval het meest aangewezen. \textcite{Lourenco2015} vergelijkt de hangbare NoSQL databases en stelt MongoDB voor als een consistente \footnote{alle clients zien steeds dezelfde data} document store database.\\
De website in de spotlight voor dit onderzoek is Google Scholar (GS), de grootste bron van wetenschappelijke publicaties op heden. De beta versie verscheen in 2004 en sindsdien wordt het systeem voornamelijk door academici gebruikt om een persoonlijke bibliotheek aan te leggen tezamen met statistieken omtrent citaties en h-indexen \footnote{De h-index van een wetenschappelijk onderzoeker komt overeen met de grootste h van het aantal publicaties die minstens h keer geciteerd zijn in ander werk.}. Volgens \textcite{Oh2019} is er een symbiose tussen GS and de academische wereld waarbij academici gratis hun werk kunnen aabieden op GS. GS op zijn beurt zorgt voor goede visibiliteit van dat werk door onder andere citaten, referenties en gelijkaardige onderwerpen. In tegenstelling tot de baanbrekende vernieuwing die GS introduceerde, gaat het raadplegen van deze gegevens wel gepaard met een aantal hindernissen, in het bijzonder op grote schaal.\\
Web scraping is 1 van de technieken die aangewend worden om dit probleem te omzeilen. Meerdere studies (\autocite{Pratiba2018},\autocite{Rafsanjani2022},\autocite{Amin2024},\autocite{Sulistya2024}) gebruiken bovenstaande concepten van web scrapers en gestructureerde opslag om automatisch data te ontleden van GS aan de hand van gebruiksvriendelijke interfaces. Er zijn onderling steeds wel verschillen tussen de gebruikte tools en het gekozen formaat die beide afhankelijk zijn van het beoogde resultaat. Maar het omliggend kader is steeds hetzelfde met gebruik van web scraping en gestructureerde opslag.\\
Het is evident dat HTML pagina's verwerkt kunnen worden door web scrapers, maar daarnaast zijn ongeveer 70\% van de feiten die op het internet gepresenteerd worden, verkregen uit PDF-documenten \autocite{Singrodia2019}. Dit verklaart de noodzaak om zowel HTML pagina's als PDF documenten te scrapen.\\
\textcite{Yang2017} beschrijft welke HTML elementen belangrijk zijn bij het ontleden van een GS pagina en \textcite{Rahmatulloh2020} toont hoe deze kunnen gemapt worden naar de custom code van de scraper.\\

Dit hoofdstuk bevat je literatuurstudie. De inhoud gaat verder op de inleiding, maar zal het onderwerp van de bachelorproef *diepgaand* uitspitten. De bedoeling is dat de lezer na lezing van dit hoofdstuk helemaal op de hoogte is van de huidige stand van zaken (state-of-the-art) in het onderzoeksdomein. Iemand die niet vertrouwd is met het onderwerp, weet nu voldoende om de rest van het verhaal te kunnen volgen, zonder dat die er nog andere informatie moet over opzoeken \autocite{Pollefliet2011}.

Je verwijst bij elke bewering die je doet, vakterm die je introduceert, enz.\ naar je bronnen. In \LaTeX{} kan dat met het commando \texttt{$\backslash${textcite\{\}}} of \texttt{$\backslash${autocite\{\}}}. Als argument van het commando geef je de ``sleutel'' van een ``record'' in een bibliografische databank in het Bib\LaTeX{}-formaat (een tekstbestand). Als je expliciet naar de auteur verwijst in de zin (narratieve referentie), gebruik je \texttt{$\backslash${}textcite\{\}}. Soms is de auteursnaam niet expliciet een onderdeel van de zin, dan gebruik je \texttt{$\backslash${}autocite\{\}} (referentie tussen haakjes). Dit gebruik je bv.~bij een citaat, of om in het bijschrift van een overgenomen afbeelding, broncode, tabel, enz. te verwijzen naar de bron. In de volgende paragraaf een voorbeeld van elk.

\textcite{Knuth1998} schreef een van de standaardwerken over sorteer- en zoekalgoritmen. Experten zijn het erover eens dat cloud computing een interessante opportuniteit vormen, zowel voor gebruikers als voor dienstverleners op vlak van informatietechnologie~\autocite{Creeger2009}.

Let er ook op: het \texttt{cite}-commando voor de punt, dus binnen de zin. Je verwijst meteen naar een bron in de eerste zin die erop gebaseerd is, dus niet pas op het einde van een paragraaf.

\begin{figure}
  \centering
  \includegraphics[width=0.8\textwidth]{grail.jpg}
  \caption[Voorbeeld figuur.]{\label{fig:grail}Voorbeeld van invoegen van een figuur. Zorg altijd voor een uitgebreid bijschrift dat de figuur volledig beschrijft zonder in de tekst te moeten gaan zoeken. Vergeet ook je bronvermelding niet!}
\end{figure}

\begin{listing}
  \begin{minted}{python}
    import pandas as pd
    import seaborn as sns

    penguins = sns.load_dataset('penguins')
    sns.relplot(data=penguins, x="flipper_length_mm", y="bill_length_mm", hue="species")
  \end{minted}
  \caption[Voorbeeld codefragment]{Voorbeeld van het invoegen van een codefragment.}
\end{listing}

\begin{table}
  \centering
  \begin{tabular}{lcr}
    \toprule
    \textbf{Kolom 1} & \textbf{Kolom 2} & \textbf{Kolom 3} \\
    $\alpha$         & $\beta$          & $\gamma$         \\
    \midrule
    A                & 10.230           & a                \\
    B                & 45.678           & b                \\
    C                & 99.987           & c                \\
    \bottomrule
  \end{tabular}
  \caption[Voorbeeld tabel]{\label{tab:example}Voorbeeld van een tabel.}
\end{table}


%%=============================================================================
%% Methodologie
%%=============================================================================

\chapter{\IfLanguageName{dutch}{Methodologie}{Methodology}}%
\label{ch:methodologie}

%% TODO: In dit hoofstuk geef je een korte toelichting over hoe je te werk bent
%% gegaan. Verdeel je onderzoek in grote fasen, en licht in elke fase toe wat
%% de doelstelling was, welke deliverables daar uit gekomen zijn, en welke
%% onderzoeksmethoden je daarbij toegepast hebt. Verantwoord waarom je
%% op deze manier te werk gegaan bent.
%% 
%% Voorbeelden van zulke fasen zijn: literatuurstudie, opstellen van een
%% requirements-analyse, opstellen long-list (bij vergelijkende studie),
%% selectie van geschikte tools (bij vergelijkende studie, "short-list"),
%% opzetten testopstelling/PoC, uitvoeren testen en verzamelen
%% van resultaten, analyse van resultaten, ...
%%
%% !!!!! LET OP !!!!!
%%
%% Het is uitdrukkelijk NIET de bedoeling dat je het grootste deel van de corpus
%% van je bachelorproef in dit hoofstuk verwerkt! Dit hoofdstuk is eerder een
%% kort overzicht van je plan van aanpak.
%%
%% Maak voor elke fase (behalve het literatuuronderzoek) een NIEUW HOOFDSTUK aan
%% en geef het een gepaste titel.
Om nieuwe publicaties die gerelateerd zijn aan wetenschappelijke projecten toe te voegen aan IMIS, moet er natuurlijk een signaal zijn wanneer dergelijke publicaties beschikbaar zijn. Indexen van academische literatuur zijn daarvoor een geschikte bron en zoals eerder beschreven is Google Scholar (GS) een interessant alternatief. Het is de bedoeling om enkel nieuwe resultaten te ontvangen en bijgevolg vervalt dus de optie om publicaties op te zoeken aan de hand van de GS zoekpagina. De zoekpagina houdt geen gegevens bij van vorige queries en daarom is het zeer aannemelijk dat dezelfde resultaten zullen voorkomen bij opeenvolgende zoekopdrachten. GS laat echter toe om meldingen aan te maken voor een bepaalde zoekopdracht. Die stuurt automatisch nieuwe resultaten door per e-mail onder de vorm van de GS SERP \footnote{Search Engine Result Page}. Het opstellen van een zoekopdracht en het aanmaken van een alert worden verder uitgewerkt in hoofdstuk~\ref{ch:googlescholaralert}.\\
GS meldingen zijn e-mails in HTML formaat. Ze bevatten inhoud die overeenkomt met de GS SERP overeenkomstig de zoekopdracht. De SERP bevat een vaste structuur. Het is een lijst met zoekresultaten die telkens dezelfde elementen bevatten:
\begin{itemize}
    \item titel
    \item link naar de webpagina van de publicatie
    \item auteurs
    \item tijdschrift
    \item jaartal
    \item abstract van de publicatie of een fragment ervan
\end{itemize}
Die HTML moet omgezet worden in gestructureerde data door middel van HTML scraping technieken zoals te zien zijn in tabel \ref{table:HTML scraping technieken}.
\begin{table}[h!]
    \begin{tabularx}{\textwidth}{|>{\hsize=1.0\hsize\linewidth=\hsize}X
            |>{\hsize=1.0\hsize\linewidth=\hsize}X|}
        \hline
        HTML scraping door een LLM &
        \begin{itemize}
            \item online model
            \begin{itemize}
                \item OpenAI
                \item Generieke procedure onafhankelijk van het model
            \end{itemize}
            \item lokaal model
        \end{itemize}
         \\
        \hline
         HTML scraping door het parsen van de DOM &
        \begin{itemize}
            \item Beautiful Soup
            \item SerpAPI
        \end{itemize}\\
        \hline
    \end{tabularx}
    \caption{HTML scraping technieken.}
    \label{table:HTML scraping technieken}
\end{table}

Web scraping wordt verder uitgewerkt in hoofdstuk~\ref{ch:web_scraping}.\\
Elk zoekresultaat is op basis van het algoritme van GS gematched met de zoekopdracht. Maar wil dat daarom ook zeggen dat de publicatie interessant is voor IMIS? Aan de hand van Natural Language Processing (NLP) wordt een score berekend van de relevantie van het zoekresultaat voor de overeenkomstige collectie in IMIS. NLP wordt verder uitgewerkt in Hoofdstuk~\ref{ch:natural_language_processing}.\\
Om te weten of een publicatie echt nieuw is, moet ze ondubbelzinnig geïdentificeerd kunnen worden. Titel, auteurs, tijdschrift, enz. maken een publicatie echter niet uniek. Er kunnen namelijk variante benamingen voorkomen van titels, auteurs, enz. Voor literatuur is het de DOI die een publicatie uniek maakt. Voor elk zoekresultaat wordt er op zoek gegaan naar de DOI aan de hand van een stapsgewijze procedure:
\begin{enumerate}
    \item DOI opzoeken in de link naar de webpagina van de publicatie
    \item DOI opzoeken in Crossref op basis van de titel
    \item DOI opzoeken op de webpagina van de publicatie
    \begin{itemize}
        \item dit kan een HTML pagina zijn
        \item dit kan een PDF document zijn
        \item dit kan een HTML pagina zijn met een embedded PDF document
    \end{itemize}
\end{enumerate}
Het opzoeken van de DOI wordt verder uitgewerkt in hoofdstuk~\ref{ch:linked_data}.\\
Dan resteert de vraag of de publicatie echt nieuw is, of dat ze reeds aan IMIS toegevoegd werd? Het antwoord daarop is afhankelijk van de beschikbare informatie die tijdens de voorgaande stappen gevonden werd.
\begin{itemize}
    \item De DOI is gevonden: er kan met 100\% zekerheid opgezocht worden of de publicatie reeds in IMIS zit of niet.
    \item De DOI is niet gevonden: omwille van de variante benamingen kan niet met volledige zekerheid opgezocht worden of een publicatie reeds in IMIS zit of niet. Wel kan door middel van semantic search op basis van de titel de waarschijnlijkheid berekend worden dat de publicatie reeds in IMIS zit.
\end{itemize}
Semantic search wordt verder uitgewerkt in hoofdstuk~\ref{ch:semantic_search}.\\
Tenslotte worden alle methodes toegepast op de volledige dataset van alle GS alerts met betrekking tot het VLIZ voor de periode van februari tot en met mei 2025. Deze resultaten worden besproken in hoofdstuk~\ref{ch:resultaten}.





%%=============================================================================
%% Methodologie
%%=============================================================================

\chapter{\IfLanguageName{dutch}{Google Scholar alert}{Google Scholar alert}}%
\label{ch:googlescholaralert}

%% TODO: In dit hoofstuk geef je een korte toelichting over hoe je te werk bent
%% gegaan. Verdeel je onderzoek in grote fasen, en licht in elke fase toe wat
%% de doelstelling was, welke deliverables daar uit gekomen zijn, en welke
%% onderzoeksmethoden je daarbij toegepast hebt. Verantwoord waarom je
%% op deze manier te werk gegaan bent.
%% 
%% Voorbeelden van zulke fasen zijn: literatuurstudie, opstellen van een
%% requirements-analyse, opstellen long-list (bij vergelijkende studie),
%% selectie van geschikte tools (bij vergelijkende studie, "short-list"),
%% opzetten testopstelling/PoC, uitvoeren testen en verzamelen
%% van resultaten, analyse van resultaten, ...
%%
%% !!!!! LET OP !!!!!
%%
%% Het is uitdrukkelijk NIET de bedoeling dat je het grootste deel van de corpus
%% van je bachelorproef in dit hoofstuk verwerkt! Dit hoofdstuk is eerder een
%% kort overzicht van je plan van aanpak.
%%
%% Maak voor elke fase (behalve het literatuuronderzoek) een NIEUW HOOFDSTUK aan
%% en geef het een gepaste titel.

\section{Google Scholar alerts}
Om op de hoogte te blijven van nieuwe publicaties kan een index van academische literatuur gebruikt worden. Zoals eerder beschreven is Google Scholar daarvoor een waardevolle bron.\\
Het is de bedoeling om alleen nieuwe publicaties te ontvangen die nog niet eerder opgezocht werden. Bijgevolg wordt er beter geen gebruik gemaakt van de Google Scholar zoekpagina, omdat die dezelfde publicaties kan tonen bij opeenvolgende opzoekingen. Daarentegen zijn er Google Scholar alerts. Dat zijn meldingen onder de vorm van automatische e-mails waarin de nieuwe publicaties opgelijst staan. Aangezien de meldingen steeds gekoppeld zijn aan een account, houdt het systeem rekening met de zoekgeschiedenis en zijn de zoekresultaten incrementeel.\\
Bijlage Google Scholar zoekopdracht heeft een uitvoerige uiteenzetting over het opmaken van een zoekopdracht en het instellen van een melding.\\
In het kader van deze opdracht werd een nieuw account \textbf{google-scholar@marineinfo.org} aangemaakt, als een gedeeld account waar meerdere gebruikers toegang tot hebben. Vervolgens werd een Google account aangemaakt met hetzelfde e-mailadres. Vervolgens werden meerdere zoekopdrachten opgemaakt telkens voor een bepaald project, en geäctiveerd als meldingen.
Bijvoorbeeld om relevante publicaties over het VLIZ te vinden, worden volgende zoektermen gebruikt:
\begin{itemize}
    \item Vlaams Instituut voor de Zee
    \item Vlaams Instituut van de Zee
    \item Flanders Marine Institute
    \item VLIZ
    \item Simon Stevin
    \item R/V Simon Stevin
    \item RV Simon Stevin
    \item Marine Station Ostend
    \item Mariene Station Oostende
\end{itemize}

Vanaf dat moment zal Google Scholar voor elke zoekopdracht e-mails met zoekresultaten sturen naar het e-mailadres van het account, zolang dat de melding geäctiveerd is.
%%=============================================================================
%% Methodologie
%%=============================================================================

\chapter{\IfLanguageName{dutch}{Web scraping}{Web scraping}}%
\label{ch:web_scraping}

%% TODO: In dit hoofstuk geef je een korte toelichting over hoe je te werk bent
%% gegaan. Verdeel je onderzoek in grote fasen, en licht in elke fase toe wat
%% de doelstelling was, welke deliverables daar uit gekomen zijn, en welke
%% onderzoeksmethoden je daarbij toegepast hebt. Verantwoord waarom je
%% op deze manier te werk gegaan bent.
%% 
%% Voorbeelden van zulke fasen zijn: literatuurstudie, opstellen van een
%% requirements-analyse, opstellen long-list (bij vergelijkende studie),
%% selectie van geschikte tools (bij vergelijkende studie, "short-list"),
%% opzetten testopstelling/PoC, uitvoeren testen en verzamelen
%% van resultaten, analyse van resultaten, ...
%%
%% !!!!! LET OP !!!!!
%%
%% Het is uitdrukkelijk NIET de bedoeling dat je het grootste deel van de corpus
%% van je bachelorproef in dit hoofstuk verwerkt! Dit hoofdstuk is eerder een
%% kort overzicht van je plan van aanpak.
%%
%% Maak voor elke fase (behalve het literatuuronderzoek) een NIEUW HOOFDSTUK aan
%% en geef het een gepaste titel.
\section{Inleiding}
De e-mails afkomstig van Google Scholar bevatten een lijst met zoekresultaten. Het formaat van de e-mail is HTML en de opmaak van de lijst is gelijkaardig aan die van de Google Scholar resultaten pagina. Dit wordt algemeen benoemd als een SERP (Search Engine Result Page) en is voor iedereen die vertrouwd is met het internet herkenbaar als de lijst met zoekresultaten van Google (zie \ref{fig:Google ScholarSERP}).
\begin{figure}
    \centering
    \includegraphics[width=0.8\textwidth]{./4_NLP/SERP.jpg}
    \caption[Google Scholar SERP.]{\label{fig:Google ScholarSERP}Google Scholar SERP.}
\end{figure}\\
De Google Scholar SERP heeft een vaste structuur, namelijk een lijst met zoekresultaten bestaande uit:
\begin{itemize}
    \item titel
    \item link naar de webpagina van de publicatie
    \item auteurs
    \item naam van het tijdschrift
    \item jaartal
    \item abstract of fragment van het abstract
\end{itemize} 
Bovenstaande gegevens moeten uit de SERP gefilterd worden zodat ze opgeslaan kunnen worden voor verder gebruik in de volgende stappen.\\
Informatie uit HTML pagina's halen is algemeen gekend onder de naam ``web scraping''. Deze techniek geniet veel aandacht omdat hij de gebruiker in staat stelt om veel data te vergaren, en data is het nieuwe goud. Tot zover dit gefilosofeer over web scraping. Het punt is dat in die context meerdere technieken bestaan om dezelfde job toe doen:
\begin{itemize}
    \item web scraping met gebruik van een LLM \footnote{Large Language Model}
    \begin{itemize}
        \item online model
        \item lokaal model
    \end{itemize}
    \item web scraping met verwerking van de DOM
    \begin{itemize}
        \item Beautifulsoup
        \item Serpapi
    \end{itemize}
\end{itemize} 

\section{Web scraping met gebruik van een LLM }
De meest gekende vorm van web scraping gebruikt de HTML structuur om er de inhoud uit te filteren. Daar komt dus niets van AI bij kijken. Het probleem met die aanpak is dat de custom code die de HTML structuur verwerkt, sterk afhankelijk is van de HTML zelf. Bijvoorbeeld wanneer de titel van een publicatie tussen <h3> tags staat die gekenmerkt worden door een class="gse\_alrt\_title" (zie \ref{code:HTMLcodefragment}), dan zal de custom code specifiek filteren op die DOM elementen.
\begin{listing}
<h3 style=\"font-weight:normal;margin:0;font-size:17px;line-height:20px;\"><span style=\"font-size:11px;font-weight:bold;color:#1a0dab;vertical-align:2px\">[HTML]</span> <a href="https://scholar.google.be/scholar\_url?url=https://www.sciencedirect.com/science/article/pii/S0191886925001308\&amp;hl=nl\&amp;sa=X\&amp;d=9286058128011819102\&amp;ei=9W3gZ7v-GcCSieoPqLGJyAM\&amp;scisig=AFWwaebeqxEetR78Fz7JpxPtT7ui\&amp;oi=scholaralrt\&amp;hist=P\_QG1LwAAAAJ:2727769339669043622:AFWwaeYuVCLO-kY6yEQWAvLJNk68\&amp;html=\&amp;pos=0\&amp;folt=kw-top\" class=\"gse\_alrt\_title\" style=\"font-size:17px;color:#1a0dab;line-height:22px\">The role of narcissistic personality <b>traits </b>in bullying behavior in adolescence–A systematic review and meta-analysis</a></h3>
\caption[Prompt htmlfragment]{HTML fragment van de titel van een publicatie.}
\label{code:HTMLcodefragment}
\end{listing}
Maar wanneer de structuur van de HTML wijzigt om welke reden dan ook, dan zal de web scraper niet langer werken zolang de custom code niet werd aangepast.\\
Daarom begint dit hoofdstuk met meer recente technieken die gebaseerd zijn op AI en die niet strikt afhankelijk zijn van de HTML structuur.
\subsection{Web scraping met OpenAI}
LLMs zijn bijzonder goed in het beantwoorden van vragen. OpenAI is voor het brede publiek beter gekend door zijn chatbot ChatGPT. Maar hoe goed is OpenAI in het parsen van een webpagina? OpenAI biedt ook een API aan waarmee gebruikers opdrachten kunnen sturen naar een model. De opdracht/vraag in kwestie is: ``Geef de titels, orginele links, auteurs, naam van de tijdschriften, jaartal van de publicaties en tekstfragmenten van hetvolgende Google Scholar zoekresultaat.''\\
\textcite{Serpapiai2025} beschrijft stap voor stap hoe de gevraagde gegevens verkregen kunnen worden met het ``gpt-4-1106-preview'' model van OpenAI.
\textcite{Depaepeopenai2025} maakt hiervan een implementatie aangepast voor de Google Scholar alerts.
Het is vereist om een account te registreren bij OpenAI en om die account te crediteren. Vervolgens kan er met de API gewerkt worden en wordt er betaald naargelang het verbruik.
Een test met 1 Google Scholar alert met 10 zoekresultaten heeft een prijs van 0,17\$ zoals te zien is in figuur \ref{fig:OpenAI dashboard}. Het model gebruikte daarvoor 14743 tokens. 
\begin{figure}
    \centering
    \includegraphics[width=0.8\textwidth]{./4_NLP/openai_billing.png}
    \caption[OpenAI dashboard.]{\label{fig:OpenAI dashboard}OpenAI dashboard.}
\end{figure}
De prompt voor het model is te zien in codefragment \ref{code:Promptcodefragment}.
\begin{listing}
    \begin{minted}{python}
        messages=[
        {"role": "system",
            "content": "You are a master at scraping Google Scholar results data. Scrape top 10 organic results data from Google Scholar search result page."},
        {"role": "user", "content": body_text}
        ],
    \end{minted}
    \caption[Prompt codefragment]{Codefragment voor het opstellen van een prompt.}
    \label{code:Promptcodefragment}
\end{listing}
De verwachte parameters die het model moet zoeken zijn te zien in codefragment \ref{code:Parse opties codefragment}.
\begin{listing}
    \begin{minted}{python}
        "function": {
            "name": "parse_data",
            "description": "Parse organic results from Google Scholar SERP raw HTML data nicely",
            "parameters": {
                'type': 'object',
                'properties': {
                    'data': {
                        'type': 'array',
                        'items': {
                            'type': 'object',
                            'properties': {
                                'title': {'type': 'string'},
                                'original_url': {'type': 'string'},
                                'authors': {'type': 'string'},
                                'year_of_publication': {'type': 'integer'},
                                'journal_name': {'type': 'string'},
                                'snippet': {'type': 'string'}
                            }
                        }
                    }
                }
            }
        }
    \end{minted}
    \caption[Parse opties codefragment]{Codefragment voor het opstellen van parse opties.}
    \label{code:Parse opties codefragment}
\end{listing}
De test was succesvol. Het model vond de gevraagde parameters voor elk van de 10 zoekresultaten. Toch houdt het onderzoek naar web scraping hier niet op. Het resultaat tot zover is immers betalend en dat is niet noodzakelijk de beste oplossing voor het probleem.

\subsection{Web scraping met Mirascope en Anthropic}
De vorige test was zeer specifiek geschreven voor OpenAI. De code zou niet werken voor een model van een andere provider. Daarom wordt er verder gezocht naar een meer generieke methode die niet afhankelijk is van de provider van het model.\\
\textcite{Mirascope2025} is een gespecialiseerde bibliotheek om op uniforme wijze met verschillende LLMs te werken.
\textcite{Anthropic2025} beschrijft stap voor stap hoe de gevraagde gegevens verkregen kunnen worden aan de hand van Mirascope en het Anthropic Claude model \autocite{Anthropicmodel2025}. Het free tier van Anthropic is uitgebreider dan dat van OpenAI zodat het account voor dit experiment niet gecrediteerd moest worden.
\textcite{Depaepeanthropic2025} maakt hiervan een implementatie aangepast voor de Google Scholar alerts.
Ook deze test was succesvol. Het model vond de gevraagde parameters voor elk van de 10 zoekresultaten. Toch houdt het onderzoek naar web scraping ook hier niet op. De oplossing tot zover heeft nog steeds een beperkte free tier en meerdere opdrachten zouden ook aanleiding geven tot een betalende service.
Daarnaast zijn beide oplossingen ook afhankelijk van een online service wat in bepaalde gevallen niet wenselijk kan zijn (bijvoorbeeld afhankelijk van de klant).
\subsection{Web scraping met een lokaal model}
Om niet afhankelijk te zijn van een online service moet het model lokaal gehost worden. \textcite{Ollama2025} is een platform om LLMs te hosten op je eigen systeem. Eenmaal geïnstalleerd (bijvoorbeeld via Docker) kan een model naar keuze gedownload en uitgevoerd worden op de lokale machine.
Het is daarbij aan de gebruiker om te kiezen welk model het best geschikt is voor het probleem, maar ook om de vereisten van het model en de beschikbare hardware op elkaar af te stemmen. De commando's \ref{Ollama} illustreren hoe het llama3:8b model (4.7GB) gedownload en uitgevoerd wordt op een lokaal systeem via Docker.
\begin{listing}
    docker exec -it ollama ollama pull llama3:8b
    docker exec -it ollama ollama run llama3
    \caption[Ollama]Ollama}
    \label{Ollama}
\end{listing}
\textcite{Scrapegraphaillama2025} beschrijft stap voor stap hoe webpagina's gescraped kunnen worden aan de hand van ScrapegraphAI en het llama3 model.
\textcite{ScrapeGraphAI2025} is een library die LLMs en grafen combineert om webpagina's te scrapen. \textcite{Llama32025} is ideaal voor breed gebruik en is kosten-efficiënt. Het munt uit in het begrijpen en genereren van menselijke teksten.
\textcite{Depaepeollama2025} maakt hiervan een implementatie aangepast voor de Google Scholar alerts.
De test was niet succesvol. Het model vond geen enkele van de gevraagde parameters voor geen enkele van de 10 zoekresultaten.
In eerste instantie werd gedacht dat het geselecteerde model te zwak is voor deze taak. Daarom werd een andere test uitgevoerd op een systeem met 40GB RAM met het gemma3:27b model (17GB) \autocite{Gemma32025}. Maar ook deze test was niet succesvol: de opdracht bleef gedurende 1 uur lopen zonder resultaat, waarna het commando manueel onderbroken werd. 
\section{Web scraping via het parsen van de DOM}
Tot zover is er nog geen bruikbaar resultaat. Het gebruik van een online model is betalend en kan niet gewenst zijn door de klant. Anderzijds leidt een lokaal model voorlopig niet tot een goed resultaat. Daarom wordt er teruggegrepen naar de meer klassieke benadering van web scraping waarbij de HTML structuur ontleed wordt aan de hand van custom code.
De Google Scholar SERP is een gestructureerde HTML pagina. Voor het parsen wordt er gebruik gemaakt van deze structuur.
Elk afzonderlijk resultaat heeft een H3-tag met als klasse 'gse\_alrt\_title' waarin de titel staat. De korte tekst van het zoekresultaat staat in een DIV-tag met als klasse 'gse\_alrt\_sni'. Tussen de titel en het snippet staan de auteurs, de uitgever en het jaartal gegroepeerd in een DIV-tag zoals te zien is in figuur \ref{fig:serp-html}.
\begin{figure}
    \centering
    \includegraphics[width=0.8\textwidth]{./2_parse_zoekresultaat/serp-html.JPG}
    \caption[HTML structuur van de GS alert.]{\label{fig:serp-html}HTML structuur van de GS alert.}
\end{figure}
\FloatBarrier
Meerdere specifieke libraries laten toe om HTML en XML te parsen, maar de meest gekende bibliotheek is Beautiful Soup \autocite{Soup2025}. Door de naam van de HTML-tag en de gebruikte klasse (gse\_alrt\_title, en gse\_alrt\_sni) mee te geven aan Beautiful Soup, worden de overeenkomstige elementen in het HTML-document weergegeven. Beautiful Soup laat ook toe om te navigeren in het document zodat het element met de auteurs, de uitgever en het jaartal gemakkelijk gevonden kan worden.
\textcite{Depaepeollama2025} maakt hiervan een implementatie aangepast voor de Google Scholar alerts. De volledige uitwerking is te zien in bijlage Parse Google Scholar zoekopdracht.
De test met een SERP met 10 zoekresultaten was geslaagd. De gevraagde parameters worden gevonden voor elk van de 10 resultaten.\\
Tenslotte moet ook nog SerpAPI \autocite{Serpapi2025} vermeld worden. Tijdens het onderzoek naar web scraping kwam dit platform meermaals naar voren, zelfs voor de toepassingen die gebruik maken van AI. SerpAPI biedt specifieke oplossingen voor elke mogelijke search engine, waaronder ook Google Scholar. Alleen focust hun product telkens op de online search engine en niet op alerts waardoor de zoekresultaten niet incrementeel zijn.\\
Tot besluit van dit hoofdstuk wordt er dus verder gewerkt met het parsen van de zoekresultaten aan de hand van Beautifulsoup. Hoewel deze methode volledig afhankelijk is van de structuur van de HTML, is ze toch zeer begrijpbaar en efficiënt. Mits implementatie van een notificatie kunnen veranderingen van de HTML structuur gemeld worden zodat de code vervolgens aangepast kan worden. 

%%=============================================================================
%% Methodologie
%%=============================================================================

\chapter{\IfLanguageName{dutch}{Natural Language Processing}{Natural Language Processing}}%
\label{ch:natural_language_processing}

%% TODO: In dit hoofstuk geef je een korte toelichting over hoe je te werk bent
%% gegaan. Verdeel je onderzoek in grote fasen, en licht in elke fase toe wat
%% de doelstelling was, welke deliverables daar uit gekomen zijn, en welke
%% onderzoeksmethoden je daarbij toegepast hebt. Verantwoord waarom je
%% op deze manier te werk gegaan bent.
%% 
%% Voorbeelden van zulke fasen zijn: literatuurstudie, opstellen van een
%% requirements-analyse, opstellen long-list (bij vergelijkende studie),
%% selectie van geschikte tools (bij vergelijkende studie, "short-list"),
%% opzetten testopstelling/PoC, uitvoeren testen en verzamelen
%% van resultaten, analyse van resultaten, ...
%%
%% !!!!! LET OP !!!!!
%%
%% Het is uitdrukkelijk NIET de bedoeling dat je het grootste deel van de corpus
%% van je bachelorproef in dit hoofstuk verwerkt! Dit hoofdstuk is eerder een
%% kort overzicht van je plan van aanpak.
%%
%% Maak voor elke fase (behalve het literatuuronderzoek) een NIEUW HOOFDSTUK aan
%% en geef het een gepaste titel.
\section{Inleiding}
Na de web scraping zijn de gegevens van elk zoekresultaat gestructureerd opgeslagen. Per definitie zijn al deze resultaten relevant voor IMIS want Google Scholar (GS) heeft ze opgezocht in functie van de zoekopdracht. De sortering van de zoekresultaten gebeurt op basis van de omvang van de integrale tekst, het aanzien van het tijdschrift en van de auteurs en het aantal citaties (inclusief de gedateerdheid ervan). Op die manier volgt GS de manier van werken binnen de academische wereld.\\
Maar daarnaast is het nog steeds mogelijk om voor elk zoekresultaat een score af te leiden die aangeeft hoe relevant de publicatie is voor de zoekopdracht aan de hand van de frequentie van elke zoekterm in de tekst. Algemeen wordt aangenomen dat hoe meer een zoekterm voorkomt in de tekst, des te relevanter die tekst zal zijn.\\
Daar bestaan Natural Language Processing (NLP) technieken voor. NLP is een verzamelnaam van een hele groep toepassingen en algoritmes die tekst omzetten in informatie. Deze zijn zeer divers en daarom ook onderverdeeld in verschillende deelgebieden. Het bepalen van de relevantie van een tekst in functie van een trefwoord valt eerder onder het deelgebied van de ``Natural Language Understanding'' die alle technieken bundelt die toelaten om de betekenis van een tekst beter te begrijpen.\\

\section{Relevantiescore}
Het uitgangspunt is om een score te berekenen door te tellen hoe vaak een trefwoord voorkomt in de tekst. De frequentie is dan recht evenredig met de relevantie van die tekst. Een eenvoudige techniek voor het bepalen van de frequentie is aan de hand van een ``Bag of Words'' (BoW). Dat is een tabel met een rij voor elke tekst en een kolom voor elk uniek woord. De cellen tonen het aantal keer dat het woord voorkomt in de tekst.  Dit is te zien in figuur \ref{fig:bow}.
\begin{figure}[h!]
    \centering
    \includegraphics[width=0.8\textwidth]{./4_NLP/bow.jpg}
    \caption[Bag of Words.]{\label{fig:bow}Bag of Words.}
\end{figure}
Toch is de frequentie op zich nog niet de beste waardemeter voor een tekst. Er kunnen namelijk woorden heel vaak voorkomen die op zich weinig vertellen over het onderwerp. Om daaraan tegemoet te komen bestaat er een variant van de BoW, die in plaats van gewoon te tellen de ``term frequency-inverse document frequency'' (Tf-Idf) geeft. De score is dan ook omgekeerd evenredig met de frequentie van de zoekterm in alle documenten. De formules om de Tf-Idf te berekenen, zijn te zien in codefragment \ref{code:tf-idf} en het resultaat in figuur \ref{fig:tf-idf}.
\begin{listing}[h!]
    \[
        score\ =\ tf\ \ast\ idf
    \]  
    where
    \[
        tf=term\ frequency\ \left(see\ above\right)
    \] 
    
    \[
        idf_t=log\left(\frac{N}{df_t}\right)
    \]  
    
    \[ 
        N=total\ number\ ofdocuments
    \]  
    
    \[ 
        df_t=the\ number\ of\ documents\ in\ which\ term\ t\ occurs
    \] 
     
    \caption[term frequency-inverse document frequency]{Berekening van de term frequency-inverse document frequency.}
    \label{code:tf-idf}
\end{listing}
\begin{figure}[h!]
    \centering
    \includegraphics[width=0.8\textwidth]{./4_NLP/tf-idf.jpg}
    \caption[term frequency-inverse document frequency.]{\label{fig:tf-idf}Term frequency-inverse document frequency.}
\end{figure}
De Tf-Idf methode is de de facto berekening van de relevantie van een term voor een bepaalde tekst. De ``TFidVectorizer()'' \footnote{https://scikit-learn.org/stable/modules/generated/sklearn.feature\_extraction.text.TfidfVectorizer.html} van Scikit-Learn is een praktische tool om de Tf-Idf score van elk afzonderlijk woord in een tekst te berekenen. De code op github \textcite{Depaepenlp2025} maakt hiervan een implementatie aangepast voor de GS alerts.
\section{Tekstverwerking}
De BoW geeft wel aanleiding tot tabellen met zeer grote dimensies. Hoe meer teksten er zijn, des te uitgebreider zal de woordenschat worden. Daarom wordt de BoW altijd voorafgegaan door tekstverwerking die stopwoorden \footnote{Stopwoorden (of, a, the, in ,you, ...) komen vaak voor maar hebben geen toegevoegde waarde over het onderwerp.} verwijdert en lemmatisering \footnote{Lemmatisering zet woorden om naar hun basisvorm, maar houdt daarbij rekening met de context. Voorbeeld ``caring'' wordt omgezet in ``care'' en niet in ``car''.}.\\
De voornaamwoorden in de tekst worden vervangen door het trefwoord waarnaar ze verwijzen (vb. tabel \ref{table:nlp}).
\begin{table}[h!]
    \centering
    \begin{tabular}{|c|c|} 
        \hline
        voor&``Het VLIZ is pionier in zeekennis.\\&Het heeft de Simon Stevin als onderzoeksvaartuig.\\&Daarmee voert het marien onderzoek uit.''\\
        \hline
        na&``Het VLIZ is pionier in zeekennis.\\&Het VLIZ heeft de Simon Stevin als onderzoeksvaartuig.\\&Simon Stevin als onderzoeksvaartuig voert het VLIZ marien onderzoek uit.''\\
        \hline
    \end{tabular}
    \caption{Voor en na coreference resolution.}
    \label{table:nlp}
\end{table}
Het opzoeken van de paren (voornaamwoord - zelfstandig naamwoord) aan de hand van de Spacy Coreference resolver \autocite{Spacy2025} is te zien in codefragment \ref{code:coreference} en is eveneens geïmplementeerd in \textcite{Depaepenlp2025}.\\
\begin{listing}
    \begin{minted}{python}
        import spacy
        ...
        def __init__(self, db_service: DBService, logging_service: LoggingService):
            self.db_service = db_service
            self.logging_service = logging_service
            self.nlp = spacy.load("en_coreference_web_trf")
        ...
        def coreference_resolution(self, text):
            doc = self.nlp(text)
            spans = doc.spans
            span_array = []
            for spangroup in spans.values():
                span_tuple = []
                for span in spangroup:
                    span_tuple.append(text[span.start_char:span.end_char])
                    span_array.append(span_tuple)
            return span_array
    \end{minted}
    \caption[Coreference resolver]{Coreference resolver}
    \label{code:coreference}
\end{listing}
Na NLP is de tweede onderzoeksdoelstelling bereikt: Een proof-of-concept van het proces dat een score voor elk zoekresultaat berekent.
%%=============================================================================
%% Methodologie
%%=============================================================================

\chapter{\IfLanguageName{dutch}{Linked data}{Linked data}}%
\label{ch:linked_data}

%% TODO: In dit hoofstuk geef je een korte toelichting over hoe je te werk bent
%% gegaan. Verdeel je onderzoek in grote fasen, en licht in elke fase toe wat
%% de doelstelling was, welke deliverables daar uit gekomen zijn, en welke
%% onderzoeksmethoden je daarbij toegepast hebt. Verantwoord waarom je
%% op deze manier te werk gegaan bent.
%% 
%% Voorbeelden van zulke fasen zijn: literatuurstudie, opstellen van een
%% requirements-analyse, opstellen long-list (bij vergelijkende studie),
%% selectie van geschikte tools (bij vergelijkende studie, "short-list"),
%% opzetten testopstelling/PoC, uitvoeren testen en verzamelen
%% van resultaten, analyse van resultaten, ...
%%
%% !!!!! LET OP !!!!!
%%
%% Het is uitdrukkelijk NIET de bedoeling dat je het grootste deel van de corpus
%% van je bachelorproef in dit hoofstuk verwerkt! Dit hoofdstuk is eerder een
%% kort overzicht van je plan van aanpak.
%%
%% Maak voor elke fase (behalve het literatuuronderzoek) een NIEUW HOOFDSTUK aan
%% en geef het een gepaste titel.
\section{Inleiding}
De web scraping leverde gegevens zoals titel, auteurs, naam van tijdschrift en jaartal van de publicatie op. Deze informatie heeft echter geen garantie van uniciteit. Dat is te wijten aan mogelijke varianten (zoals bijvoorbeeld afkortingen) in de titel, auteurs en naam van tijdschriften.\\
Nochtans moet een publicatie uniek geïdentificeerd kunnen worden om uit te sluiten dat er reeds duplicaten in IMIS zitten.
De DOI vormt de unieke indentifier van elke publicatie. Er zal getracht worden om deze voor elke publicatie op te zoeken. Hiervoor worden 3 afzonderlijke bronnen geraadpleegd in chronologische volgorde en zolang de DOI niet gevonden werd:
\begin{enumerate}
    \item In de URL van de link naar de originele publicatie (zie figuur \ref{fig:DOIurl})
    \item Aan de hand van een opzoeking in Crossref op basis van de titel
    \item Op de originele webpagina van de publicatie (zie figuur \ref{fig:DOIwebpagina})
\end{enumerate}

\begin{figure}[h!]
    \centering
    \includegraphics[width=0.8\textwidth]{./3_crossref/DOI_Link.jpg}
    \caption[Illustratie van online publicatie waar de DOI deel uitmaakt van de URL.]{\label{fig:DOIurl}Illustratie van online publicatie waar de DOI deel uitmaakt van de URL.}
\end{figure}
\begin{figure}[h!]
    \centering
    \includegraphics[width=0.8\textwidth]{./3_crossref/DOI_webpagina.jpg}
    \caption[Illustratie van online publicatie waar de DOI geen deel uitmaakt van de URL, maar wel op de webpagina staat.]{\label{fig:DOIwebpagina}Illustratie van online publicatie waar de DOI geen deel uitmaakt van de URL, maar wel op de webpagina staat.}
\end{figure}

\section{De DOI opzoeken in de URL}
\label{Doiurl}
De web scraping vond ook de URL naar de webpagina van de oorspronkelijke publicatie. Empirisch valt het op dat veel van die URLs opgebouwd zijn met de DOI van de publicatie (vb. in codefragment \ref{code:oriurl}).
\begin{listing}
    https://pubs.acs.org/doi/full/10.1021/acsami.4c21991
    \caption[URL met DOI]{Originele URL van de publicatie}
    \label{code:oriurl}
\end{listing}\\
Alle DOIs hebben dezelfde structuur: ze beginnen met het cijfer 10, gevolgd door een punt en 4 tot 9 cijfers, daarna volgt een slash. Verder kan elke willekeurige opeenvolging van letters, cijfers, speciale tekens en slashes voorkomen.
De lijst met reguliere expressies die volgens \textcite{CrossrefRegex2025} gebruikt wordt om DOIs te matchen is te zien in tabel \ref{table:regex}.
\begin{table}[h!]
    \begin{tabularx}{\linewidth}{|X|}
        \hline
        \begin{itemize}
            \item \texttt{/\textasciicircum10.\textbackslash d{4,9}/[-.\_;()/:A-Z0-9]+\$/i}
            \item \texttt{/\textasciicircum10.1002/[\textasciicircum\textbackslash s]+\$/i}
            \item \texttt{/\textasciicircum10.\textbackslash d{4}/\textbackslash d+-\textbackslash d+X?(\textbackslash d+)\textbackslash d+<[\textbackslash d\textbackslash w]+:[\textbackslash d\textbackslash w]*>\textbackslash d+.\textbackslash d+.\textbackslash w+;\textbackslash d\$/i}
            \item \texttt{/\textasciicircum10.1021/\textbackslash w\textbackslash w\textbackslash d++\$/i}
            \item \texttt{/\textasciicircum10.1207/[\textbackslash w\textbackslash d]+\textbackslash \&\textbackslash d+\_\textbackslash d+\$/i}
        \end{itemize}
        \hline
    \end{tabularx}
    \caption{Reguliere expressies om een DOI te matchen.}
    \label{table:regex}
\end{table}
Indien één van deze reguliere expressies matcht met de URL, dan is de DOI gevonden.
De omzetting in code is te zien in codefragment \ref{code:Urlregex}.
\begin{listing}
    \begin{minted}{python}
        def search_in_text(text, link):
            # find using regex
            patterns = get_patterns()
            doi_result = None
            while (doi_result is None) and (len(patterns) > 0):
                doi_result = re.search(patterns.pop(), text, re.IGNORECASE)
        
            if doi_result is not None:
                # update DOI
                doi = doi_result.group(0)
                link.doi = doi
                link.is_doi_success = True
                link.log_message = "DOI successfully retrieved"
        
        ...
        
        def get_patterns():
            patterns = [r"10.1207/[\w\d]+\&\d+_\d+", r"10.1021/\w\w\d++",
            r"10.\d{4}/\d+-\d+X?(\d+)\d+<[\d\w]+:[\d\w]*>\d+.\d+.\w+;\d", r"10.1002/[^\s]+",
            r"10.\d{4,9}/[-._;()/:A-Z0-9]+"]
            return patterns
    \end{minted}
    \caption[Opzoeken van DOI in de URL van de publicatie]{Opzoeken van DOI in de URL van de publicatie.}
    \label{code:Urlregex}
\end{listing}
\section{De DOI opzoeken in Crossref}
Indien de voorgaande stap geen DOI opleverde, dan wordt hier op basis van de titel een opzoeking van de DOI gedaan in Crossref. Daarvoor beschikt Crossref over een API \autocite{Crossrefhowtoapi2025}. De documentatie leert dat ondoordachte requests kunnen leiden tot zeer langdurige queries en/of ongewenste resultaten. Concreet wordt er afgeraden om meer dan 2 velden op te nemen in de query van een sample, of om meer dan 2 resultaten te vragen in het geval van een matching. Verder wordt voor matching aangeraden om te zoeken aan de hand van de bibliografische gegevens zoals te zien in codefragment \ref{code:Bibliographic}.
\begin{listing}
    http://api.crossref.org/works?query.bibliographic="Toward a Unified Theory of High-Energy Metaphysics, Josiah Carberry 2008-08-13"&rows=2
    \caption[Crossref query op basis van de bibliografie]{Query op basis van de bibliografie}
    \label{code:Bibliographic}
\end{listing}
Maar Google Scholar alerts geven geen bibliografie, daarom wordt enkel de titel verder gebruikt.
Het is niet nodig om de API rechtstreeks te bevragen aangezien meerdere onafhankelijke Python libraries daar een wrapper voor geschreven hebben:
\begin{itemize}
    \item Crossref Commons for Python \autocite{Crossrefcommons2025}
    \item Habanero \autocite{Habanero2025}
    \item Crossrefapi \autocite{Crossrefapi2025}
\end{itemize}
Al deze bibliotheken bieden dezelfde tools en presteren gelijkaardig. Zonder bijzondere reden, behalve dat Crossref Commons ontwikkeld wordt door Crossref zelf, wordt er met Crossref Commons for Python gewerkt. De code om een sample op te vragen op basis van de titel is bijzonder compact zoals te zien in \ref{code:Crossref_commons}.
\begin{listing}
    \begin{minted}{python}
        try:
            filter = {}
            queries = {'query.title': title}
            response = crossref_commons.sampling.get_sample(size=2, filter=filter, queries=queries)
    \end{minted}
    \caption[Crossref commons]{Opvragen van een sample op basis van de titel aan Crossref.}
    \label{code:Crossref_commons}
\end{listing}
Indien de Crossref API een resultaat geeft, dan is de DOI gevonden.
\section{De DOI opzoeken op de webpagina van de publicatie}
Indien de voorgaande stap geen DOI opleverde, dan wordt de DOI gezocht op de webpagina van de publicatie. In de meeste gevallen is dat op de website van de uitgever van het tijdschrift. Op die pagina staan altijd de titel, auteurs, naam van het tijdschrift, jaartal en abstract van de publicatie.
Soms staat ook de DOI op die pagina.
In sommige gevallen is de integrale tekst van de publicatie hier beschikbaar als HTML of als PDF document.\\
Het formaat van de pagina kan verschillend zijn:
\begin{itemize}
    \item een gewone HTML pagina
    \item een PDF document
    \item een webpagina met een embedded PDF document
\end{itemize}
Voor elk van de 3 gevallen is er een andere verwerking nodig.
\begin{itemize}
    \item In het geval van een HTML pagina wordt de inhoud geparsed met Beautiful Soup net zoals dat eerder gebeurde voor de SERP. Vervolgens wordt een DOI opgezocht in de inhoud door middel van de reguliere expressies uit hoofdstuk \ref{Doiurl}.
    \item In het geval van een PDF document is er een extra tussenstap nodig. De inhoud moet gelezen worden met gebruik van PyMuPDF \autocite{Pymupdf2025}. Daarna wordt er ook gezocht aan de hand van de reguliere expressies. Voorbeeld in codefragment \ref{code:DOIpymupdf}.
    \begin{listing}
        \begin{minted}{python}
            def search_in_pdf(pdf, link):
                doc = pymupdf.Document(stream=pdf)
                # Extract all Document Text
                text = chr(12).join([page.get_text() for page in doc])
                patterns = get_patterns()
                doi_result = None
                while (doi_result is None) and (len(patterns) > 0):
                    pattern = re.compile(patterns.pop(), re.IGNORECASE)
                    doi_result = pattern.search(text)
            
                if doi_result is not None:
                    #update DOI
                    doi = doi_result.group(0)
                    link.doi = doi
                    link.is_doi_success = True
                    link.log_message = "DOI successfully retrieved"
        \end{minted}
        \caption[Pymupdf]{Openen van een online pdf.}
        \label{code:DOIpymupdf}
    \end{listing}
    \item In het geval van een embedded PDF is de inhoud niet onmiddellijk beschikbaar. De gebruiker moet als het ware nog een handeling verrichten (vb. op een knop klikken) alvorens toegang te krijgen tot de inhoud. Dat gaat niet voor een geautomatiseerd script, maar door middel van Selenium \autocite{Selenium2025} kan de gebruikersinteractie nagebootst worden zodat de embedded content toch automatisch gedownload wordt. Voorbeeld in codefragment \ref{code:DOIselenium}.
    \begin{listing}
        \begin{minted}{python}
            ...
            url = link.location_replace_url
            options = webdriver.ChromeOptions()
            download_folder = os.path.join(str(Path(__file__).parent.parent.parent.parent), "online_pdf")
            profile = {
                "plugins.plugins_list": [{"enabled": False, "name": "Chrome PDF Viewer"}],
                "download.default_directory": download_folder,
                "download.extensions_to_open": "",
                "download.prompt_for_download": False,
                "download.directory_upgrade": True,
                "plugins.always_open_pdf_externally": True
            }
            options.add_experimental_option("prefs", profile)
            options.add_argument("start-maximized") # open Browser in maximized mode
            options.add_argument("disable-infobars") # disabling infobars
            options.add_argument("--disable-extensions") # disabling extensions
            options.add_argument("--disable-gpu") # applicable to windows os only
            options.add_argument("--disable-dev-shm-usage") # overcome limited resource problems
            options.add_argument("--no-sandbox") # Bypass OS security  model
            options.add_argument("--headless")
            driver = webdriver.Chrome(options=options)
            driver.get(url)
            driver.close()
            ...
        \end{minted}
        \caption[Selenium]{Nabootsen van gebruikersinteractie met Selenium.}
        \label{code:DOIselenium}
    \end{listing} Eenmaal gedownload kan het bestand gewoon geopend worden en doorzocht worden naar een DOI op dezelfde manier als voor PDF documenten.
\end{itemize}
Er is geen garantie dat de DOI op de webpagina van de publicatie gevonden wordt. Anderzijds is het ook mogelijk dat er meerdere verschillende DOIs (vb. van referenties) gevonden worden. Voor een mens is het vaak evident om te weten welke DOI dan juist is, maar voor geautomatiseerd script is daar geen context voor.\\
Er kan dus besloten worden dat de DOI met grote zekerheid achterhaald kan worden op basis van de URL of aan de hand van Crossref. In het geval van een opzoeking op de webpagina van de publicatie is de vindkans een pak kleiner.


%%=============================================================================
%% Methodologie
%%=============================================================================

\chapter{\IfLanguageName{dutch}{Semantic search}{Semantic search}}%
\label{ch:semantic_search}

%% TODO: In dit hoofstuk geef je een korte toelichting over hoe je te werk bent
%% gegaan. Verdeel je onderzoek in grote fasen, en licht in elke fase toe wat
%% de doelstelling was, welke deliverables daar uit gekomen zijn, en welke
%% onderzoeksmethoden je daarbij toegepast hebt. Verantwoord waarom je
%% op deze manier te werk gegaan bent.
%% 
%% Voorbeelden van zulke fasen zijn: literatuurstudie, opstellen van een
%% requirements-analyse, opstellen long-list (bij vergelijkende studie),
%% selectie van geschikte tools (bij vergelijkende studie, "short-list"),
%% opzetten testopstelling/PoC, uitvoeren testen en verzamelen
%% van resultaten, analyse van resultaten, ...
%%
%% !!!!! LET OP !!!!!
%%
%% Het is uitdrukkelijk NIET de bedoeling dat je het grootste deel van de corpus
%% van je bachelorproef in dit hoofstuk verwerkt! Dit hoofdstuk is eerder een
%% kort overzicht van je plan van aanpak.
%%
%% Maak voor elke fase (behalve het literatuuronderzoek) een NIEUW HOOFDSTUK aan
%% en geef het een gepaste titel.
\section{Inleiding}
Alle publicaties in IMIS moeten uniek zijn. Op basis van voorgaande stappen is dat met 100\% zekerheid te bepalen in het geval de DOI van de publicatie gevonden werd.
Als de DOI niet gevonden werd, kan er niet met zekerheid gesproken worden, maar er kan wel een score berekend worden dat de publicatie reeds in IMIS zit. Dat is mogelijk aan de hand van semantic search.\\
Op basis van de titel van het artikel wordt een embedding berekend. Vervolgens wordt die vergeleken met alle embeddings van de titels in IMIS om zo tot een score te komen van de gelijkenis tussen beide. Dit wordt voorgesteld in figuur \ref{fig:Semanticsearch}
\begin{figure}
    \centering
    \includegraphics[width=0.8\textwidth]{./5_semantic_search/embeddings.jpg}
    \caption[Semantic search.]{\label{fig:Semanticsearch}Semantic search.}
\end{figure}
\\
Indien de score hoog is dan is er veel waarschijnlijkheid dat de publicatie reeds in IMIS zit. In het geval van een lage score is de waarschijnlijkheid eerder klein.\\
\section{Chroma}
Embeddings worden opgeslaan in een vector databank. Er zijn talloze producten beschikbaar:
\begin{itemize}
    \item Pinecone
    \item Chroma
    \item Weaviate
    \item Qdrant
    \item Milvus
    \item Vespa
    \item SingleStore
    \item Redis
    \item Elastic Stack
    \item Mongo
    \item …
    \item (Every database that can store an n-array of numbers)
\end{itemize}
Chroma \autocite{Chroma2025} is een gebruiksvriendelijke database die toelaat om semantic search lokaal te testen zonder extra kosten, zonder account, en zonder installatie van andere software.\\
Het online artikel \textcite{Usechroma2025} beschrijft stap voor stap hoe een semantic search met Chroma uitgevoerd kan worden. 
De code op github \textcite{Depaepechroma2025} maakt hiervan een implementatie aangepast voor de Google Scholar alerts.
De connectie met de Chroma database is te zien in codefragment \ref{code:Chromadb}.
\begin{listing}
    \begin{minted}{python}
        ...
        chroma_client = chromadb.Client()
        # https://docs.trychroma.com/docs/collections/configure
        self.collection = chroma_client.create_collection(
            name="my_collection",
            metadata={
                "hnsw:space": "cosine"
            }
        )
        ...
    \end{minted}
    \caption[Chroma codefragment]{Codefragment voor het connecteren met Chroma.}
    \label{code:Chromadb}
\end{listing}
Vervolgens worden alle titels uit IMIS opgevraagd en worden de overeenkomstige embeddings berekend aan de hand van de code in codefragment \ref{code:Chromaembeddings}
\begin{listing}
    \begin{minted}{python}
        def initialize_embeddings(self):
            result = requests.get(``https://www.vliz.be/nl/imis?count=2000&module=ref&searchspcollist=39&show=json'')
            publications = result.json()
            documents = []
            ids = []
            count = 1
            for publication in publications:
            #self.logging_service.logger.debug(publication)
                documents.append(publication['StandardTitle'])
                ids.append(f"id{count}")
                count+=1
        
            self.collection.add(
                documents=documents,
                ids=ids
            )
    \end{minted}
    \caption[Embeddings codefragment]{Codefragment voor het berekenen van de embeddings.}
    \label{code:Chromaembeddings}
\end{listing}
Dan kan een titel opgezocht worden zoals getoond wordt in codefragment \ref{code:Chromaquery}.
\begin{listing}
    \begin{minted}{python}
        def do_semantic_search(self, title):
            results = self.collection.query(
                query_texts=[title],  # Chroma will embed this for you
                n_results=2  # how many results to return
            )
            return results['distances'][0][0]
    \end{minted}
    \caption[Query codefragment]{Codefragment voor het opzoeken van een titel.}
    \label{code:Chromaquery}
\end{listing}
\\
De test was succesvol. Bij wijze van proef werd de titel van 1 van de zoekresultaten toegevoegd aan de lijst van titels uit IMIS. Het resultaat is te zien in figuur \ref{fig:Chroma}.
\begin{figure}
    \centering
    \includegraphics[width=1.0\textwidth]{./5_semantic_search/chroma-cosine-score.jpg}
    \caption[Chroma resultaat.]{\label{fig:Chroma}Chroma resultaat.}
\end{figure}
Deze oplossing voegt Chroma toe als database naast MongoDB die nodig is voor de Web scraping. Het zou beter zijn indien de embeddings ook in MongoDB opgeslaan kunnen worden. Dat wordt verder onderzocht.
\section{MongoDB}
Atlas Vector Search \autocite{Mongodbvectorsearch2025} laat toe om de embeddings tezamen op te slaan met de originele data waar ze van afgeleid zijn.\\
De online documentatie \textcite{Usemongodbvectorsearch2025} beschrijft stap voor stap hoe een semantic search met Atlas Vector Search lokaal uitgevoerd kan worden. 
De code op github \textcite{Depaepemongodb2025} maakt hiervan een implementatie aangepast voor de Google Scholar alerts.
De selectie van het model voor de embeddings is te zien in codefragment \ref{code:Mongodbmodel}.
\begin{listing}
    \begin{minted}{python}
        ...
        self.model = SentenceTransformer("nomic-ai/nomic-embed-text-v1", trust_remote_code=True)
        ...
    \end{minted}
    \caption[MongoDB model codefragment]{Codefragment voor het configureren van het model met MongoDB.}
    \label{code:Mongodbmodel}
\end{listing}
Vervolgens worden alle titels uit IMIS opgevraagd en worden de overeenkomstige embeddings berekend aan de hand van de code in codefragment \ref{code:Mongodbembeddings}
\begin{listing}
    \begin{minted}{python}
        def initialize_embeddings(self):
            docs = []
            data = []
            embeddings = []
            result = requests.get(IMIS)
            publications = result.json()
            for publication in publications:
                data.append(publication['StandardTitle'])
                embeddings.append(self.get_embedding(publication['StandardTitle']))
        
            for i, (embedding, title) in enumerate(zip(embeddings, data)):
            doc = {
                "_id": i,
                "title": title,
                "embedding": embedding,
            }
            docs.append(doc)
            self.db_service.set_collection("embeddings")
            self.db_service.insert_many(docs)
    \end{minted}
    \caption[Embeddings codefragment]{Codefragment voor het berekenen van de embeddings.}
    \label{code:Mongodbembeddings}
\end{listing}
Vervolgens wordt de lijst met embedding geïndexeerd zoals getoond wordt in codefragment \ref{code:Mongodbindex}.
\begin{listing}
    \begin{minted}{python}
        def create_search_index(self):
            search_index_model = SearchIndexModel(
            definition={
                "fields": [
                {
                    "type": "vector",
                    "path": "embedding",
                    "similarity": "dotProduct",
                    "numDimensions": 768
                }
                ]
            },
            name="vector_index",
            type="vectorSearch"
            )
            self.db_service.set_collection("embeddings")
            self.db_service.create_search_index(model=search_index_model)
    \end{minted}
    \caption[Index codefragment]{Codefragment voor het indexeren van de embeddings.}
    \label{code:Mongodbindex}
\end{listing}
Dan kan een titel opgezocht worden zoals getoond wordt in codefragment \ref{code:Mongodbquery}.
\begin{listing}
    \begin{minted}{python}
        def do_semantic_search(self, title):
            query_embedding = self.get_embedding(title)
            pipeline = [
            {
                "$vectorSearch": {
                    "index": "vector_index",
                    "queryVector": query_embedding,
                    "path": "embedding",
                    "exact": True,
                    "limit": 5
                }
            },
            {
                "$project": {
                    "_id": 0,
                    "title": 1,
                    "score": {
                        "$meta": "vectorSearchScore"
                    }
                }
            }
            ]
            self.db_service.set_collection("embeddings")
            result = self.db_service.aggregate(pipeline)
    \end{minted}
    \caption[Query codefragment]{Codefragment voor het opzoeken van een titel.}
    \label{code:Mongodbquery}
\end{listing}
\\
De test was succesvol. Bij wijze van proef werd de titel van 1 van de zoekresultaten toegevoegd aan de lijst van titels uit IMIS. Het resultaat is te zien in figuur \ref{fig:Mongodb}.
\begin{figure}
    \centering
    \includegraphics[width=1.0\textwidth]{./5_semantic_search/mongodb-cosine-score.jpg}
    \caption[Mongodb resultaat.]{\label{fig:Mongodb}Mongodb resultaat.}
\end{figure}
Deze oplossing breidt de bestaande MongoDB databank uit met embeddings. Voor artikels zonder indentifcatie kan een score berekend worden die duidt op duplicaten in IMIS. 

%%=============================================================================
%% Methodologie
%%=============================================================================

\chapter{\IfLanguageName{dutch}{Resultaten en discussie}{Resultaten en discussie}}%
\label{ch:resultaten}

%% TODO: In dit hoofstuk geef je een korte toelichting over hoe je te werk bent
%% gegaan. Verdeel je onderzoek in grote fasen, en licht in elke fase toe wat
%% de doelstelling was, welke deliverables daar uit gekomen zijn, en welke
%% onderzoeksmethoden je daarbij toegepast hebt. Verantwoord waarom je
%% op deze manier te werk gegaan bent.
%% 
%% Voorbeelden van zulke fasen zijn: literatuurstudie, opstellen van een
%% requirements-analyse, opstellen long-list (bij vergelijkende studie),
%% selectie van geschikte tools (bij vergelijkende studie, "short-list"),
%% opzetten testopstelling/PoC, uitvoeren testen en verzamelen
%% van resultaten, analyse van resultaten, ...
%%
%% !!!!! LET OP !!!!!
%%
%% Het is uitdrukkelijk NIET de bedoeling dat je het grootste deel van de corpus
%% van je bachelorproef in dit hoofstuk verwerkt! Dit hoofdstuk is eerder een
%% kort overzicht van je plan van aanpak.
%%
%% Maak voor elke fase (behalve het literatuuronderzoek) een NIEUW HOOFDSTUK aan
%% en geef het een gepaste titel.
\section{Inleiding}
De Google Scholar alerts leverden 23 notificatie e-mails met zoekresultaten op voor de VLIZ zoekopdracht (zie hoofdstuk \ref{ch:googlescholaralert}) sinds 1 februari 2025 tot 16 mei 2025.\\
Figuur \ref{fig:GSAlertsTimeline} toont de tijdslijn van de Google Scholar alerts ontvangen tijdens deze tijdsspanne.
\begin{figure}[h!]
    \centering
    \includegraphics[width=0.8\textwidth]{./9_resultaten/GS_alerts_timeline.png}
    \caption[Tijdslijn Google Scholar alerts.]{\label{fig:GSAlertsTimeline}Tijdslijn Google Scholar alerts.}
\end{figure}
Een volledig overzicht van alle resultaten na web scraping, natural language processing, Crossref en semantic search is weergegeven in bijlage \ref{ch:overzicht_resultaten}.

\section{Web scraping}
Titel, link en tekst worden systematisch goed gevonden door de scraping. Voor auteurs, tijdschrift en jaartal is dat beduidend minder het geval. De reden daarvoor is dat deze 3 laatste parameters binnen dezelfde tag staan en gescheiden worden door specifieke karakters. Dat maakt het een pak moeilijker voor de scraper om die informatie te  ontleden, vooral in het geval wanneer 1 of 2 van die gegevens ook ontbreken in het zoekresultaat zoals te zien is in figuur \ref{fig:GSauteurtijdschriftjaartal}.
\begin{figure}[h!]
    \centering
    \includegraphics[width=0.8\textwidth]{./9_resultaten/GS_alerts_auteurtijdschriftjaartal.jpg}
    \caption[Google Scholar alert zonder auteur, tijdschrift of jaartal.]{\label{fig:GSauteurtijdschriftjaartal}Google Scholar alert zonder auteur, tijdschrift of jaartal.}
\end{figure}

\section{Relevantiescore}
De relevantiescore werd berekend op basis van het voorkomen van de zoektermen in de tekst. Op die manier vallen de zoekresultaten waarin de zoektermen niet voorkomen onmiddellijk op door hun waarde 0 zoals te zien is in figuur  \ref{fig:GSrelevantie}.
\begin{figure}[h!]
    \centering
    \includegraphics[width=0.8\textwidth]{./9_resultaten/GS_alerts_relevantiescore.png}
    \caption[Google Scholar alert relevantiescore.]{\label{fig:GSrelevantie}Google Scholar alert relevantiescore.}
\end{figure}

\section{DOI}
Voor het opzoeken van de DOI boekt de tool het meeste winst ten opzichte van de huidige situtatie. Dat blijkt ook uit de gebruikersfeedback. Dit kan geverifieerd worden door data uit IMIS te verwerken met de tool en vervolgens de aanwezigheid van de DOI te vergelijken tussen IMIS en de resultaten.\\
Daarvoor wordt er een steekproef uit IMIS genomen van alle data sinds begin 2025 tot op heden.
De overeenkomstige query is:
\[https://www.vliz.be/nl/imis?module=ref&show=json\]
\[&spcol=39,141,396,536,733,746,747,748,749,750,751,\]
\[752,753,754,755,756,757,758,759,760,761,762,763,764,\]
\[765,766,768,769,770,771,772,773,774,775,776,777,778,\]
\[779,780,781,782,783,784,785,786,787,788,789,790,791,\]
\[802,803,900,911,981,982,992,997,1002,1051,1064\]
\[&optionYear=after&searchYear=2024&sort=date\]
Dat levert op heden 917 records op waarvan vervolgens de DOI opgezocht wordt aan de hand van de tool.
\begin{table}[h!]
    \caption{Statistieken Linked data}
    \centering
    \begin{tabularx}{\textwidth}{|X|p{4cm}|} 
        \hline
        Percentage aantal keer dat de DOI gevonden is.&48\%\\
        \hline
        \begin{itemize}
            \item Percentage aantal keer dat de DOI in de URL staat.
            \item Percentage aantal keer dat de DOI via Crossref gevonden wordt.
            \item Percentage aantal keer dat de DOI op de webpagina staat.
        \end{itemize}
        &
        \begin{itemize}
            \item 7\%
            \item 20\%
            \item 73\%
        \end{itemize}
        \\
        \hline
    \end{tabularx}
    \label{table:statistieken_linked_data}
\end{table}
\section{Semantic search}
Voor de resterende 52\% van de zoekresultaten waarvan geen DOI gevonden kon worden, werd semantic search van de titel uitgevoerd in de IMIS databank. Dat vormde een hele uitdaging aangezien er 260.000 records zijn waarin de titel opgezocht moet worden.
De query om de titels op te zoeken is:
\[https://www.vliz.be/nl/imis?module=ref&show=json\]
\[&spcol=39,141,396,536,733,746,747,748,749,750,751,\]
\[752,753,754,755,756,757,758,759,760,761,762,763,764,\]
\[765,766,768,769,770,771,772,773,774,775,776,777,778,\]
\[779,780,781,782,783,784,785,786,787,788,789,790,791,\]
\[802,803,900,911,981,982,992,997,1002,1051,1064\]
Daarbij duiden de ``spcol'' waarden op alle mogelijke collecties binnen IMIS die te maken hebben met het VLIZ (de zoekterm van de Google Scholar alert).
Omdat IMIS niet toelaat om 260.000 records ineens op te vragen, wordt er gewerkt met 26 batches van 10.000 records. De overeenkomstige query is:
\[https://www.vliz.be/nl/imis?module=ref&show=json\]
\[&spcol=39,141,396,536,733,746,747,748,749,750,751,\]
\[752,753,754,755,756,757,758,759,760,761,762,763,764,\]
\[765,766,768,769,770,771,772,773,774,775,776,777,778,\]
\[779,780,781,782,783,784,785,786,787,788,789,790,791,\]
\[802,803,900,911,981,982,992,997,1002,1051,1064\]
\[&count=10000&start=0\]
Daarbij verhoogt de parameter ``start'' telkens met 10.000.\\
Deze manier van werken heeft mogelijks een invloed op het resultaat: de positie van een titel opzoeken in 10.000 resultaten is niet hetzelfde als in 260.000 resultaten.
In figuur \ref{fig:GSsss} is te zien dat de Semantic search score toeneemt in functie van het aantal titels waarin gezocht wordt. Maar de mate van toename neemt sterk af bij grote aantallen. Daarom wordt er aangenomen dat de Semantic search score bij 10.000 titels representatief is voor de score bij 260.000 titels.\\
\begin{figure}[h!]
    \centering
    \includegraphics[width=0.8\textwidth]{./9_resultaten/sss.png}
    \caption[Semantic search score vs. number of titles.]{\label{fig:GSsss}Semantic search score vs. number of titles.}
\end{figure}
Na verwerking is de Semantic search score de hoogste waarde van alle batches.\\\\
Alle gevonden scores liggen onder de 0.9. Dat is de threshold die empirisch bepaald werd om te beslissen of een titel reeds in IMIS zit (zie tabel \ref{table:empirisch}).\\
\begin{table}[h!]
    \caption{Empirische test kritische waarde semantic search}
    \centering
    \begin{tabularx}{\textwidth}{|X|p{4cm}|} 
        \hline
        \textbf{Titel}&\textbf{Semantic search score}\\
        \hline
        Genome-wide association study revealed candidate genes associated with egg-laying time traits in layer chicken&\textcolor{orange}{1.0}\\
        \hline
        Genome wide association study revealed candidate genes associated with egg laying time traits in layer chicken&\textcolor{orange}{0.9958072900772095}\\
        \hline
        Gen. wide assoc. study revealed candidate genes associated with egg time traits in layer chicken&\textcolor{orange}{0.9708424806594849}\\
        \hline
        Genome-wide association study revealed candidate genes associated with egg-laying time traits&\textcolor{orange}{0.9543328285217285}\\
        \hline
        Genome wide association study revealed candidate genes associated with egg laying time traits&\textcolor{orange}{0.9487157464027405}\\
        \hline
        Gen. wide assoc. study revealed candidate genes associated with egg time traits&\textcolor{orange}{0.9242777228355408}\\
        \hline
        with time study revealed associated Genome-wide in chicken genes traits candidate association egg-laying&\textcolor{orange}{0.9624322652816772}\\
        \hline
        Gen.-wide assoc. study reveal cand. gene associate with egg-lay time trait in layer chicken&\textcolor{orange}{0.9395400881767273}\\
        \hline
        Genome-wide association study revealed some new candidate genes associated with flowering and maturity time of soybean in Central and West Siberian regions of Russia&0.8146297931671143\\
        \hline
        Genome- and Exome-Wide Association Studies Revealed Candidate Genes Associated with DaTscan Imaging Features&0.7932324409484863\\
        \hline
        Genome-wide association study revealed candidate genes associated with leaf size in alfalfa&0.8393213748931885\\
        \hline
        Genome-Wide Association Studies Revealed the Genetic Loci and Candidate Genes of Pod-Related Traits in Peanut&0.8185563087463379\\
        \hline
    \end{tabularx}
    \label{table:empirisch}
\end{table}
Bijgevolg is er geen enkel duplicaat aanwezig tussen de zoekresultaten.
\section{Gebruikersfeedback}
In deze vroege fase van ingebruiksname van de tool is er nog niet veel bewustzijn bij de gebruikers.
Er werd aan 5 kerngebruikers van de tool gevraagd om 2 vragen te beantwoorden:
\begin{itemize}
    \item Is de tool een verbetering ten opzichte van de manuele verwerking van de Google Scholar alerts?
    \item Gaan de publicaties uit de Google Scholar alerts nu sneller verwerkt worden dankzij de tool?
\end{itemize}
3 gebruikers deelden hun feedback zoals te zien is in tabel \ref{table:gebruikersfeedback}.
\begin{table}[h!]
    \tiny
    \caption{gebruikersfeedback}
    \centering
    \begin{tabularx}{\textwidth}{|p{2cm}|X|} 
        \hline
        \rowcolor{lightgray}
        \multicolumn{2}{|X|}{\textbf{Is de tool een verbetering ten opzichte van de manuele verwerking van de Google Scholar alerts?}} \\
        \hline
        gebruiker 1 &\enquote{Ja, collega's hadden elk eigen Scholar alerts ingesteld die naar hun persoonlijke mailbox werden gestuurd afhankelijk van het project waarop zij werken. Deze mails werden vaak rechtstreeks doorgestuurd naar de bib die dan afhankelijk van de verstuurder de collectie bepalen en op zoek gingen of de publicatie interessant is voor IMIS en deze nog niet in IMIS beschikbaar is.
        Het verzamelen van deze meldingen in 1 mailbox die automatisch wat pre-processing uitvoert en een handige lijst samenstelt zal de last voor de bib sterk verminderen.}\\
        gebruiker 2&\enquote{De tool probeert een eerste antwoord te bieden aan een van de grotere actuele problemen binnen wetenschappelijke instellingen: het geautomatiseerd binnentrekken van een gefilterde set zoekresultaten van publicaties. Binnen onze instelling VLIZ bestaat deze filter voornamelijk uit vooraf ingestelde Google Scholar alerts, die vooralsnog manueel verwerkt dienden te worden. De verdienste bestaat erin een werkbare tool te maken voor de meest omvangrijke subset: wetenschappelijke papers met een DOI.}\\
        gebruiker 3&\enquote{Laat het me houden op een hoopvolle, positieve ‘ja’.}\\
        \hline
        \rowcolor{lightgray}
        \multicolumn{2}{|X|}{\textbf{Gaan de publicaties uit de Google Scholar alerts nu sneller verwerkt worden dankzij de tool?}} \\
        \hline
        gebruiker 1&\enquote{Minder handmatig werk zorgt enerzijds voor een snellere inplanning en minder tegenopzicht om het werk aan te pakken. In plaats van afzonderlijke mails worden alle alerts mooi samengevoegd in een cleane interface. Een groot deel van het manueel werk wordt nu opgevangen door de tool, nakijken op duplicaten, tot welke collectie behoort deze publicatie, semiautomatische invoer. Hierdoor wordt ook veel tijd bespaard.}\\
        gebruiker 2&\enquote{Een systeem die de papers na voldoende controle op relevantie en aanwezigheid in onze catalogus IMIS binnentrekt. Hier zit een eerste tijdsbesparing voor ons. We moeten niet meer (altijd) nagaan of de match klopt, en we moeten de papers dus niet (altijd) meer openen. In het vervolgtraject worden de DOIs via cross-ref geïmporteerd, waardoor we reeds gedeeltelijk ingevoerde publicaties kunnen verwerken in ons invoerscherm (IMIS-input) die bovendien reeds de juiste labels kregen toegewezen. Ook deze tweede stap houdt een enorme tijdsbesparing in. Op jaarbasis gaat het over 1500 tot 2000 publicaties over de verschillende projecten heen die anders volledig manueel moeten worden ingevoerd. Ruim geschat is dit een arbeidsinvestering van ongeveer 5-6 maand VTE op een jaar die nu sterk gereduceerd wordt. Binnen dit proces verschuift de klemtoon bij de bibliotheek nu naar validatie, en het invoeren van de lastigere papers die geen DOI hebben.}\\
        gebruiker 3&\enquote{Laat het me houden op een hoopvolle, positieve ‘ja’.}\\
        \hline
    \end{tabularx}
    \label{table:gebruikersfeedback}
\end{table}

%%=============================================================================
%% Conclusie
%%=============================================================================

\chapter{Conclusie}%
\label{ch:conclusie}

% TODO: Trek een duidelijke conclusie, in de vorm van een antwoord op de
% onderzoeksvra(a)g(en). Wat was jouw bijdrage aan het onderzoeksdomein en
% hoe biedt dit meerwaarde aan het vakgebied/doelgroep? 
% Reflecteer kritisch over het resultaat. In Engelse teksten wordt deze sectie
% ``Discussion'' genoemd. Had je deze uitkomst verwacht? Zijn er zaken die nog
% niet duidelijk zijn?
% Heeft het onderzoek geleid tot nieuwe vragen die uitnodigen tot verder 
%onderzoek?
De IMIS collecties van het VLIZ kunnen voortaan semi-automatisch uitgebreid worden door het onderzoek van deze bachelorproef.\\
Google Scholar alerts zijn een automatische en incrementele bron van gegevens die zullen blijven stromen zolang de melding bestaat.\\ 
Web scraping technieken met LLMs verdienen bijzondere aandacht. Online modellen laten toe om een SERP te parsen en bovendien zijn ze bestand tegen interne veranderingen van de SERP. Na dit onderzoek moet er verder getest worden met verschillende lokale modellen op verschillende systemen. Het succes van de online modellen doet toch verwachten dat er ook lokaal een slaagkans moet zijn. Tot dan wordt er gewerkt met Beautiful Soup. Aangezien de structuur van de SERP vast ligt, is dit zeker geen slechte oplossing. De code van Beautiful Soup en de HTML structuur gaan hand in hand zodat wijzigingen van die laatste naar verwachting vrij eenvoudig op te vangen zijn in de code. Belangrijk daarbij is dat er een notificatiesysteem voorzien wordt dat afgaat wanneer er zo een wijziging zou optreden.\\
Het berekenen van een relevantiescore op basis van de frequentie van de zoekopdracht in het zoekresultaat geeft zeker aanleiding tot een nieuwe indicator, maar de mate waarin die effectief ook voorspelt of een publicatie interessant is voor IMIS is zeker vatbaar voor discussie. Verder onderzoek moet dus ook blijven inzetten op de ``Natural Language Understanding'' om de waarde van een publicatie voor IMIS beter te berekenen.\\
Het opsporen van de DOI van elke publicatie volgens een stapsgewijze procedure is efficiënt doch niet feilloos. Er blijft een minderheid van publicaties waarvoor de DOI niet bestaat of niet gevonden kan worden. Verder onderzoek moet vooral inzetten op het vinden van de juiste DOI op de webpagina van de publicatie wanneer daar meerdere DOIs aanwezig zijn. Dat is een eenvoudige taak voor een mens, maar dat is het niet voor een computer.\\
Tenslotte helpt het opzoeken van de gelijkenis tussen de titel van een publicatie en alle titels in IMIS aan de hand van ``Semantic search'' wanneer een manuele beslissing moet genomen worden om een publicatie toe te voegen aan IMIS. Bij een volledige match zal er geen twijfel zijn, maar bij een gedeeltelijke match, bijvoorbeeld in het geval van variaties in de titel, moet er beter onderzocht worden wat de threshold is om te beslissen om een publicatie aan IMIS toe te voegen.



% Voeg hier je eigen hoofdstukken toe die de ``corpus'' van je bachelorproef
% vormen. De structuur en titels hangen af van je eigen onderzoek. Je kan bv.
% elke fase in je onderzoek in een apart hoofdstuk bespreken.

%\input{...}
%\input{...}
%...



%---------- Bijlagen -----------------------------------------------------------

\appendix
%%=============================================================================
%% Methodologie
%%=============================================================================

\chapter{\IfLanguageName{dutch}{Google Scholar zoekopdracht}{Google Scholar search}}%
\label{ch:googlescholarzoekopdracht}

%% TODO: In dit hoofstuk geef je een korte toelichting over hoe je te werk bent
%% gegaan. Verdeel je onderzoek in grote fasen, en licht in elke fase toe wat
%% de doelstelling was, welke deliverables daar uit gekomen zijn, en welke
%% onderzoeksmethoden je daarbij toegepast hebt. Verantwoord waarom je
%% op deze manier te werk gegaan bent.
%% 
%% Voorbeelden van zulke fasen zijn: literatuurstudie, opstellen van een
%% requirements-analyse, opstellen long-list (bij vergelijkende studie),
%% selectie van geschikte tools (bij vergelijkende studie, "short-list"),
%% opzetten testopstelling/PoC, uitvoeren testen en verzamelen
%% van resultaten, analyse van resultaten, ...
%%
%% !!!!! LET OP !!!!!
%%
%% Het is uitdrukkelijk NIET de bedoeling dat je het grootste deel van de corpus
%% van je bachelorproef in dit hoofstuk verwerkt! Dit hoofdstuk is eerder een
%% kort overzicht van je plan van aanpak.
%%
%% Maak voor elke fase (behalve het literatuuronderzoek) een NIEUW HOOFDSTUK aan
%% en geef het een gepaste titel.

\section{Zoeken in Google Scholar}
\subsection{Basis zoeken}
Google Scholar biedt een vertrouwde gebruikersinterface met een inputveld waar de relevante zoektermen ingevuld moeten worden. Default wordt er in elke taal gezocht, maar de gebruiker kan dit beperken tot zijn eigen taal.

\begin{figure}
    \centering
    \includegraphics[width=0.8\textwidth]{./1_google_scholar_zoekopdracht/1_basis_zoeken.PNG}
    \caption[Google Scholar basis zoeken.]{\label{fig:Google Scholar basis zoeken}Google Scholar user interface voor het basis zoeken van publicaties op basis van de ingevoerde zoektermen.}
\end{figure}

Wanneer een zoekopdracht verzonden wordt, dan is er een antwoord binnen de 3 seconden. Het resultaat kan vervolgens verder gefilterd worden:
\begin{itemize}
    \item \textbf{Elke periode}: Dit is de default filter zodat alle resultaten getoond worden ongeacht hun publicatiedatum.
    \item \textbf{Sinds jaar}: Hierbij worden enkel resultaten gefilterd die sinds het gespecifiëerde jaar gepubliceerd werden.
    \item \textbf{Aangepast bereik}: Hierbij worden enkel resultaten gefilterd waarvan de publicatiedatum binnen het gespecifiëerde bereik ligt.
    \item \textbf{Sorteren op relevantie}: Dit is de default filter tezamen met 'Elke periode' die de resultaten sorteert op basis van hun belangrijkheid.\footnote{De relevantie van elke publicatie wordt in de eerste plaats bepaald door het aantal citaties (\autocite{Beel2009})}
    \item \textbf{Sorteren op datum}: Hierbij worden de resultaten gesorteerd op publicatiedatum.
    \item \textbf{Reviewartikelen}: Hierbij worden enkel state of the art publicaties gefilterd.\footnote{Een reviewartikel ondergaat een systematische review door een groep van experten volgens de op dat moment geldende 'State of the art' (\autocite{Sataloff2021})}
\end{itemize}

\begin{figure}
    \centering
    \includegraphics[width=0.8\textwidth]{./1_google_scholar_zoekopdracht/2_zoekresultaten.PNG}
    \caption[Google Scholar zoekresultaten.]{\label{fig:Google Scholar zoekresultaten}Google Scholar zoekresultaten op basis van een zoekopdracht.}
\end{figure}

Elk resultaat kan verder uitgediept worden:
\begin{itemize}
    \item \textbf{Geciteerd door}: Een oplijsting van publicaties die zelf het artikel citeren. Dit kan leiden tot andere relevante artikels.
    \item \textbf{Verwante artikelen}: Andere artikels in hetzelfde thema. Dit kan leiden tot andere relevante artikels.
    \item \textbf{Alle versies}: Alternatieve locaties waar dezelfde informatie kan teruggevonden worden. Dit kan leiden tot een breder beeld van organisaties, instituten en uitgevers.
\end{itemize}
\subsection{Geävanceerd zoeken}
Google Scholar heeft ook een meerdere geävanceerde zoekopties:
\begin{itemize}
    \item \textbf{Zoek artikels met alle termen}: Combineert zoektermen. Zoekt publicaties die alle termen bevatten.
    \item \textbf{Zoek artikels met de exacte zoekterm}: Zoekt publicaties waar de zoekterm of zin exact in terug te vinden is.
    \item \textbf{Zoek artikels met op zijn minst 1 van de zoektermen}: Zoekt publicaties waar alle of minstens 1 van de zoektermen in voorkomen.
    \item \textbf{Zoek publicaties zonder de zoektermen}: Matcht publicaties waar geen enkele van de zoektermen in voorkomen.
\end{itemize}
Voor alle bovenstaande filters kan ingesteld worden of er enkel in de titel of overal in de tekst mag gezocht worden.

\begin{figure}
    \centering
    \includegraphics[width=0.8\textwidth]{./1_google_scholar_zoekopdracht/3_geavanceerd_zoeken.PNG}
    \caption[Google Scholar geävanceerd zoeken.]{\label{fig:Google Scholar geävanceerd zoek}Google Scholar user interface voor het geävanceerd zoeken van publicaties op basis van de ingevoerde zoektermen en filters.}
\end{figure}

\linebreak
Daarnaast zijn er 3 bijkomende geävanceerde filters:
\begin{itemize}
    \item \textbf{Zoek artikels op basis van auteurs}: Zoekt publicaties die geschreven zijn door een bepaalde auteur.
    \item \textbf{Zoek artikels op basis van de uitgever}: Zoekt publicaties die uitgegeven zijn door een bepaalde uitgever.
    \item \textbf{Zoek artikels op basis van publicatiedatum}: Zoekt publicaties die gepubliceerd zijn tussen 2 opgegeven datums.
\end{itemize}

Alle geävanceerde filters kunnen verder gespecifieerd worden door middel van logische operatoren: (AND, OR, NOT, AROUND).
\begin{itemize}
    \item \textbf{AND}: Zoekt beide zoektermen in de publicatie.
    \item \textbf{OR}: Zoekt 1 of beide zoektermen in de publicatie.
    \item \textbf{NOT}: Sluit ongewenste tekst uit van het zoekresultaat.
    \item \textbf{AROUND}: Zoekt zoektermen in de ingestelde nabijheid van de opgegeven zoekterm.
\end{itemize}

Alle geävanceerde filters kunnen verder gespecifieerd worden door middel van hulpwoorden:
\begin{itemize}
    \item \textbf{intitle}: De zoekresultaten bevatten de opgegeven zoekterm in de titel.
    \item \textbf{intext}: De zoekresultaten bevatten de opgegeven zoekterm in de tekst.
    \item \textbf{author}: De zoekresultaten bevatten de opgegeven auteur.
    \item \textbf{source}: De zoekresultaten bevatten de opgegeven uitgever.
\end{itemize}

Alle geävanceerde filters kunnen verder gespecifieerd worden door middel van enkele leestekens:
\begin{itemize}
    \item \textbf{aanhalingstekens (``'')}: De zoekresultaten bevatten de exacte tekst tussen aanhalingstekens.
    \item \textbf{liggend streepje (A-B)}: Om aan te tonen dat 2 zoektermen sterk verbonden zijn.
    \item \textbf{liggend streepje (A -B)}: Om de tweede zoekterm uit te sluiten van de resultaten.
\end{itemize}

\section{E-mail alerts}
Het is mogelijk om de zoekresultaten te personaliseren. Voor elke zoekopdracht die wordt aangemaakt, kan een overeenkomstige alert ingesteld worden.
Het volstaat om het e-mailadres in te vullen naar waar de alerts verstuurd moeten worden. Dit genereert een verificatie e-mail en na bevestiging is de alert geäctiveerd.

\begin{figure}
    \centering
    \includegraphics[width=0.8\textwidth]{./1_google_scholar_zoekopdracht/4_melding_maken.PNG}
    \caption[Google Scholar melding maken.]{\label{fig:Google Scholar melding maken}Google Scholar user interface voor het aanmaken van een e-mail alert voor de ingevoerde zoekopdracht.}
\end{figure}

Vanaf dan worden nieuwe publicaties die voldoen aan de filtercriteria systematisch doorgestuurd naar het e-mailadres.
Overeenkomstig werd een nieuw account \begin{verbatim}
    google-scholar@vliz.be
\end{verbatim} aangemaakt, als een gedeeld account waar meerdere gebruikers toegang tot hebben. Vervolgens werd een Google account aangemaakt met hetzelfde e-mailadres.
\section{Opstellen van een zoekopdracht}
Om relevante publicaties over het VLIZ te vinden, worden volgende zoektermen gebruikt:
\begin{itemize}
    \item Vlaams Instituut voor de Zee
    \item Vlaams Instituut van de Zee
    \item Flanders Marine Institute
    \item VLIZ
    \item Simon Stevin
    \item R/V Simon Stevin
    \item RV Simon Stevin
    \item Marine Station Ostend
    \item Mariene Station Oostende
\end{itemize}
De zoekopdracht moet resultaten geven wanneer minstens 1 of meerdere zoektermen voorkomen in de titel of de tekst van het artikel. Dit kan bereikt worden door de zoektermen tussen aanhalingstekens te schrijven en door ze te verbinden met de OR operator.
\linebreak
\begin{lstlisting}
``Vlaams Instituut voor de Zee'' 
               OR 
``Vlaams Instituut van de Zee'' 
               OR 
``Flanders Marine Institute'' 
               OR 
``VLIZ'' 
               OR 
``Simon Stevin'' 
               OR 
``R/V Simon Stevin'' 
               OR 
``RV Simon Stevin'' 
               OR 
``Marine Station Ostend'' 
               OR 
``Mariene Station Oostende''
\end{lstlisting}

\begin{figure}
    \centering
    \includegraphics[width=0.8\textwidth]{./1_google_scholar_zoekopdracht/5_email.PNG}
    \caption[Google Scholar email alert.]{\label{fig:Google Scholar email alert}Google Scholar e-mail alert met de nieuwe resultaten sinds het aanmaken van de melding en sinds de vorige melding.}
\end{figure}



\chapter{Onderzoeksvoorstel}

Het onderwerp van deze bachelorproef is gebaseerd op een onderzoeksvoorstel dat vooraf werd beoordeeld door de promotor. Dat voorstel is opgenomen in deze bijlage.

%% TODO: 
%\section*{Samenvatting}

% Kopieer en plak hier de samenvatting (abstract) van je onderzoeksvoorstel.

% Verwijzing naar het bestand met de inhoud van het onderzoeksvoorstel
\input{../voorstel/voorstel-inhoud}

%%---------- Andere bijlagen --------------------------------------------------
% TODO: Voeg hier eventuele andere bijlagen toe. Bv. als je deze BP voor de
% tweede keer indient, een overzicht van de verbeteringen t.o.v. het origineel.
%\input{...}

%%---------- Backmatter, referentielijst ---------------------------------------

\backmatter{}

\setlength\bibitemsep{2pt} %% Add Some space between the bibliograpy entries
\printbibliography[heading=bibintoc]

\end{document}
