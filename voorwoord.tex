%%=============================================================================
%% Voorwoord
%%=============================================================================

\chapter*{\IfLanguageName{dutch}{Woord vooraf}{Preface}}%
\label{ch:voorwoord}

%% TODO:
%% Het voorwoord is het enige deel van de bachelorproef waar je vanuit je
%% eigen standpunt (``ik-vorm'') mag schrijven. Je kan hier bv. motiveren
%% waarom jij het onderwerp wil bespreken.
%% Vergeet ook niet te bedanken wie je geholpen/gesteund/... heeft

Aan de basis van deze bachelorproef ligt een JIRA-ticket dat al een tijdje oud is. Dat ticket was aangemaakt door het team van IMIS met de bedoeling om een bestaande procedure te verbeteren, doch zonder daar enige prioriteit aan te koppelen. Bij mijn zoektocht naar een onderwerp kwam ik al snel bij mijn werkgever, het VLIZ, terecht. JIRA werd erbij gehaald en het ticket waarvan sprake stak met kop en schouders boven de andere uit omwille van de toepasbaarheid, de mate van uitdaging en de haalbaarheid.\\
Wat volgde was een boeiende verdieping in de wereld van de academische literatuur, in het gigantische net van de web scraping, en niet in het minst van het eindeloze universum van de LLMs.\\
Ik wil in het bijzonder mijn manager Bart Vanhoorne bedanken die het mogelijk maakt om een omgeving te creëren waar deze vorm van werken kan bestaan.\\
Bijzondere dank ook voor mijn collega Milan Lamote voor die onuitputtelijke positiviteit en omdat die zonder aarzelen het co-promotorschap van deze bachelorproef aanvaardde.\\
Bijzondere dank ook voor mijn promotor Jan Claes voor het opzetten van een transparant kader voor deze bachelorproef.\\
Geen bijzondere maar niet minder oprechte dank voor mijn collega Fons Verheyde voor zijn enthousiasme bij het bespreken van deze bachelorproef.\\
Tenslotte aan het einde van deze bachelorproef, maar vooral aan het einde van deze opleiding, een eindeloze dank aan mijn echtgenote Lies Knockaert. Veel van de punten op mijn curiculum zijn onrechtstreeks ook haar verdienste.

Bart De Paepe,
Sint-Baafs-Vijve, 1 mei 2025