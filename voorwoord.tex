%%=============================================================================
%% Voorwoord
%%=============================================================================

\chapter*{\IfLanguageName{dutch}{Woord vooraf}{Preface}}%
\label{ch:voorwoord}

%% TODO:
%% Het voorwoord is het enige deel van de bachelorproef waar je vanuit je
%% eigen standpunt (``ik-vorm'') mag schrijven. Je kan hier bv. motiveren
%% waarom jij het onderwerp wil bespreken.
%% Vergeet ook niet te bedanken wie je geholpen/gesteund/... heeft

Aan de basis van deze bachelorproef ligt een JIRA ticket dat al een tijdje oud is. Dat ticket was aangemaakt door het team van IMIS met de bedoeling om een bestaande procedure te verbeteren, doch zonder daar enige prioriteit aan te koppelen. Bij mijn zoektocht naar een onderwerp kwam ik al snel bij mijn werkgever, het VLIZ, terecht. JIRA werd erbij gehaald en het ticket waarvan sprake stak met kop en schouders boven andere onderwerpen uit omwille van de toepasbaarheid, de mate van uitdaging en de haalbaarheid.\\
Wat volgde was een boeiende verdieping in de wereld van de academische literatuur, in het gigantische net van de web scraping, en niet in het minst van het onmetelijke universum van de LLMs.\\\\
Ik wil eerst en vooral mijn manager Bart Vanhoorne bedanken die een omgeving creëert waarin het mogelijk is om aan een bachelorproef te werken.\\\\
Bijzondere dank ook voor mijn collega Milan Lamote voor zijn aanstekelijke positiviteit en omdat hij zonder aarzelen het co-promotorschap van deze bachelorproef aanvaardde.\\\\
Bijzondere dank ook voor mijn promotor Jan Claes voor het opzetten van een transparant kader voor deze bachelorproef en voor zijn kritische feedback.\\\\
Oprechte dank voor mijn collega Fons Verheyde voor zijn enthousiasme bij het bespreken van deze bachelorproef.\\\\
Oprechte dank voor mijn collega Cedric Decruw voor zijn inspiratie over LLMs.\\\\
Tenslotte aan het einde van deze bachelorproef, maar vooral aan het einde van deze opleiding, eeuwige dank aan mijn echtgenote Lies Knockaert. Veel van de resultaten in mijn studies zijn onrechtstreeks ook haar verdienste.
\\\\
Bart De Paepe,\\
Sint-Baafs-Vijve, 1 mei 2025