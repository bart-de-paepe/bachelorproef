%%=============================================================================
%% Samenvatting
%%=============================================================================

% TODO: De "abstract" of samenvatting is een kernachtige (~ 1 blz. voor een
% thesis) synthese van het document.
%
% Een goede abstract biedt een kernachtig antwoord op volgende vragen:
%
% 1. Waarover gaat de bachelorproef?
% 2. Waarom heb je er over geschreven?
% 3. Hoe heb je het onderzoek uitgevoerd?
% 4. Wat waren de resultaten? Wat blijkt uit je onderzoek?
% 5. Wat betekenen je resultaten? Wat is de relevantie voor het werkveld?
%
% Daarom bestaat een abstract uit volgende componenten:
%
% - inleiding + kaderen thema
% - probleemstelling
% - (centrale) onderzoeksvraag
% - onderzoeksdoelstelling
% - methodologie
% - resultaten (beperk tot de belangrijkste, relevant voor de onderzoeksvraag)
% - conclusies, aanbevelingen, beperkingen
%
% LET OP! Een samenvatting is GEEN voorwoord!

%%---------- Nederlandse samenvatting -----------------------------------------
%
% TODO: Als je je bachelorproef in het Engels schrijft, moet je eerst een
% Nederlandse samenvatting invoegen. Haal daarvoor onderstaande code uit
% commentaar.
% Wie zijn bachelorproef in het Nederlands schrijft, kan dit negeren, de inhoud
% wordt niet in het document ingevoegd.

\IfLanguageName{english}{%
\selectlanguage{dutch}
\chapter*{Samenvatting}
\lipsum[1-4]
\selectlanguage{english}
}{}

%%---------- Samenvatting -----------------------------------------------------
% De samenvatting in de hoofdtaal van het document

\chapter*{\IfLanguageName{dutch}{Samenvatting}{Abstract}}

Zoekresultaten afkomstig van Google Scholar moeten aan het Integrated Marine Information System (IMIS) toegevoegd worden. Dit onderzoek bekijkt hoe dit proces volledig of grotendeels geautomatiseerd kan worden. Daarbij komen verschillende aspecten kijken die allemaal stuk voor stuk afzonderlijk behandeld worden.\\
Zo worden Google Scholar alerts gebruikt om continu nieuwe zoekresultaten te ontvangen. Er wordt uitvoerig toegelicht hoe een zoekopdracht aangemaakt moet worden en hoe daarvoor een melding ingesteld kan worden.\\
Google Scholar zoekresultaten komen in de vorm van een HTML pagina. HTML aan IMIS toevoegen is niet interessant. Er wordt dieper ingegaan hoe de lijst met zoekresultaten van de HTML pagina gescraped kan worden. Daarbij worden verschillende technieken uitgeprobeerd waaronder Large Language Models (LLMs). Uiteindelijk wordt een klassieke benadering gekozen die de HTML parset aan de hand van Beautiful Soup.\\
Na het scrapen zou de gevonden informatie in IMIS opgeslagen kunnen worden. Er zijn echter nog bijkomende stappen nodig om ervoor te zorgen dat enkel kwalitatieve gegevens naar IMIS vloeien.
Behalve het feit dat Google Scholar de zoekresultaten selecteert voor de zoekopdracht en rangschikt volgens een bepaalde volgorde, zijn er verder geen criteria die aangeven hoe relevant een zoekresultaat is voor IMIS. Natural Language Processing (NLP) biedt daar oplossingen voor. Er wordt een relevantiescore berekend van de mate waarin de zoekopdracht aanwezig is in het zoekresultaat. Algemeen wordt daarvoor aangenomen dat de frequentie van de zoekopdracht in het zoekresultaat recht evenredig is met de relevantie van het resultaat voor IMIS.\\
Er mogen ook geen duplicaten in IMIS opgeslagen worden. Duplicaten kunnen gedetecteerd worden aan de hand van de Digital Object Identifier (DOI) van de publicatie. Die kan stapsgewijs opgezocht worden in de link van de publicatie, in Crossref, of op de webpagina van de publicatie. Maar er is geen garantie dat de DOI gevonden zal worden. Voor die gevallen moet er minstens een score gegeven worden of een publicatie al dan niet een duplicaat is. Dat kan gedaan worden aan de hand van ``Semantic search''. Daarvoor worden embeddings berekend voor alle titels in IMIS. Vervolgens wordt de embedding van de titel van het zoekresultaat daarmee vergeleken. De mate van gelijkenis is een score voor duplicaten van de publicatie.\\
Uiteindelijk worden al deze stappen geïntegreerd in een pijplijn voor het semi-automatisch toevoegen van publicaties aan IMIS. In het geval de relevantiescore goed is en indien de DOI gevonden wordt, kan de publicatie zonder tussenkomst aan IMIS toegevoegd worden. Wanneer er geen DOI gevonden wordt, is er nog steeds een manuele stap nodig om te beslissen op basis van de duplicatenscore of de publicatie toegevoegd mag worden.
